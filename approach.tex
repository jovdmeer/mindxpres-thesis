% vim:ts=1:et:nospell:spelllang=en_gb:ft=tex

 \chapter{Approach}

  \section{Compilation process}

   The first part of the approach is fairly straightforward in its basic
   explanation: we had to convert \ppt presentations into \mxp presentations.
   This involves finding out how \ppt presentations are structured, getting the
   parts wee need out of that structure, and then putting those parts together
   in de \mxp structure. Since the author of this thesis has a small background
   in compilers \citep{vandermeersch-1} it did not take long to see the
   resemblance of this process to that of a compiler.

   A compiler takes source code and transforms it into a working program with
   the semantics described by that source code. The compilation process
   consists of several steps. First the source code is tokenized, which means
   the symbols in the code are identified one by one and classified in certain
   categories.

   Then the tokens are processed by a parser into an intermediary form called a
   parse tree. A parser looks for certain predefined patterns in the source
   code. These patterns are part of the source code's language syntax. As such,
   these two steps analyse and validate the source code's syntax. If part of
   the code does not match any pattern, the parser and the compilation process
   stop and the user gets a message saying the code's syntax is invalid.

   When a parse tree is constructed, the compilation process can alter it, to
   improve it. Certain patterns in the parse tree may be replaceable by
   different patterns with the same outcome, but with more optimal execution.
   This part of the compilation process is optional, and is called compiler
   optimization. Optimizations can consist of many things, depending on the
   language. For example, some languages guarantee tail call optimization,
   where infinite loops can be constructed by letting a function call itself as
   its last statement without causing a stack overflow. This is something the
   compiler (or interpreter) can optimize during this part of the compilation
   process.

   After this, the parse tree can be written out to produce the desired output.
   Every node in the tree has a well-defined equivalent in the target
   language's syntax. The target language can be Assembly, which consists of
   the exact instructions a CPU needs to carry out a program, or it can be
   another programming language. Many compilers of higher-level languages
   translate their language into C, for several reasons: the C compilers that
   translate C into Assembly have been optimized so much that it is easier to
   rely on them than to put an enormous amount of effort into optimizing
   another language; C compilers exist for most --- if not all --- CPU
   architectures, which means translating a language into C makes it compatible
   with all those architectures, while it would cost a lot more effort to write
   different compilers for every architecture you would want to make your
   language available on.

   The conversion tool that is the purpose of this thesis, can be described in
   a similar succession of steps. First, we take a \ppt presentation and
   tokenize and parse it into an intermediary structure that allows us to
   perform other operations, or `optimizations', on it. The intermediary form
   consists of a `parse tree' containing the components of the original
   presentation --- a component tree, if you will.

   With this structure, we can construct a \mxp presentation containing the
   same components in the same place, essentially creating a `program' with the
   same semantic meaning as the original `source'.

  \section{Compiler optimizations}

   Since the conversion process resembles that of a compiler, it seemed logical
   at first to make automatic layout a part of that process, as some kind of
   `compiler optimization'.

   At first, we tried to traverse the component tree, giving its objects new
   coordinates and sizes so that they would fit together on every slide as well
   as possible. This seemed an easy solution, but the results were sub-optimal.
   On top of that, we soon realised that we were in essence creating another
   template out of which a presentation would be made, which was exactly the
   opposite of what we were trying to do. As such, we abandoned this approach.

   We then switched to a different method: defining constraints for every
   component, in the form of margins, maximum sizes and other limits, and then
   calculating a way to satisfy all constraints while fitting content together
   on each slide. While this is clearly a better method, it turned out the
   compiler optimization phase was not the best place in the process to take
   care of this.

   In the end, we decided to take a different approach, relying on the layout
   engine of \mxp itself and enhancing that engine to create the automatic
   layout we wanteD.

  \section{Using \mxp}

   One of the primary goals of \mxp is to separate content from layout,
   allowing the author of a presentation to focus on the content while \mxp
   takes care of the layout. The way it does this is currently mostly through
   the compiler, which decides the width, height and coordinates of content,
   relative to the container the content belongs to. The plug-ins responsible
   for handling components and containers currently don't mess with those
   settings, but technically, they could. The compiler decides the measurements
   and coordinates based on templates. The solution we were looking for was a
   layout engine that could take any content and put it in an appropriate
   layout without any directions from the user. As such, we had to enhance
   \mxp's layout engine to use constraints, based on the size of the content,
   and try to find an optimal position for every component it is given.

   We did this by creating an invisible container plug-in. Containers are a way
   of grouping components, other containers, etc. in \mxp. This means they have
   control over their child elements, which gives us the opportunity to
   override the layout of those elements. A container plugin thus allows us to
   implement our own layout system. Since it's a new element, it doesn't
   override existing elements as it would have done if we had, for example,
   rewritten the `slide' plugin. The user can decide for themself whether or
   not to use it, and it can be used anywhere in the presentation: wrap the
   whole presentation in it, or just a small part, whichever works best for
   your purposes. It also won't break existing presentations that don't use it,
   while those presentations can very easily be altered to take advantage of
   it.

   An important aspect of this is that containers can be nested. This means we
   can create slide-based presentations, which can contain our auto-layout
   container, which then contains the slide's contents, thus creating an
   optimal layout of the content per-slide. Another way of using it could be
   without slides, throwing all content together in one auto-layout container,
   and letting it take care of the layout for the whole presentation at once.
   It should be noted here that the auto-layout container makes each of its
   child nodes focusable separately, to compensate for arbitrary resizing it
   may perform on large objects in order to fit them next to other content, by
   using the focus functionality to automatically zoom into these components
   when necessary.
  
   We call it an \emph{invisible} container plug-in because it does not
   introduce any visual content, shape or indication for itself. Compare with
   the \emph{slide} plug-in which obviously puts some kind of slide-look around
   the content it encompasses, and it becomes clear what we mean by this:
   although the content within is obviously affected by our plug-in, there is
   no visible indication of its presence to the audience.

   The plug-in uses the compiler's numbers to decide relative locations between
   components, as well as size ratios, and then finds a way to display those
   components in a way that the display order makes sense (or at least matches
   the intended order as closely as possible), that no overlapping occurs
   (since we don't have the animations that \ppt might have used to display one
   piece of information and then another on top of it), and resizing everything
   if necessary in order to fit within the specified container. While this may
   seem like a bad idea since content can get illegibly small this way, keep in
   mind that we can rely on the ZUI to focus on each component separately, or
   on groups of components, while \ppt obviously can only display the whole
   slide at once.

%   TODO we need more content here

%   15:50 <omega> zeg, ik zit nu al een hele tijd thesis te schrijven en de laatste paar dagen vooral te zeveren over layout, maar intussen doe ik nog ni echt iets van layout, met t gedacht van ik schrijf daar binnenkort ne mindxpres plug-in voor en klaar
%   15:51 <omega> maar wordt layout momenteel eig ni mostly door de compiler gedaan?
%   15:52 <omega> ben zo eens naar de presentation.js libs en code gaan kijken, en ik zie ni direct een manier om ne plug-in layout te laten doen, aangezien plug-ins mostly component-specifiek zijn en ni alle componenten kunnen aansturen
%   15:53 <omega> dus klopt het dat ik ofwel de compiler moet aanpassen, ofwel presentation.js hacken om dat soort plug-ins toe te laten?
%   15:53 <omega> of laat het dat soort plug-ins al toe maar zijn er gewoon nog geen?
%   16:30 <omega> de 'structured' plug-in doet wel layout van slides, maar binnen die slides zie ik niet meteen een systeem dat layout regelt, met templates of otherwise, het pakt gewoon de coordinaten en afmetingen die de compiler bepaald heeft
%   16:31 <omega> al zou die slide plug-in wel *kunnen* prutsen met die layout... dus mss moet ik gwn de slide plug-in uitbreiden/hacken/vervangen
%   13:22 <reinout> ik zou een container plug-in maken
%   13:22 <reinout> gelijk de slide
%   13:22 <reinout> maar dan onzichtbaar
%   13:23 <reinout> want containers kan je nesten
%   13:23 <reinout> dus een slide kan bv uw layout container bevatten, die dan de children een layout geeft
%   13:23 <reinout> maar op die manier is uw layout ding bruikbaar buiten slides
%   13:24 <reinout> (alternatief was uw layout stuff in de slide plug-in steken)
%   13:31 <omega> oeh, cool idee indeed, beter dan de slide plug-in abusen
