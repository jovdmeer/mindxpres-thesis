% vim:ts=1:et:nospell:spelllang=en_gb:ft=tex

 \chapter{Approach}

  \section{Compilation process}

   The first part of the approach is fairly straightforward in its basic
   explanation: we had to convert \ppt presentations into \mxp presentations.
   This involves finding out how \ppt presentations are structured, getting the
   parts wee need out of that structure, and then putting those parts together
   in de \mxp structure. Since the author of this thesis has a small background
   in compilers \citep{vandermeersch-1} it did not take long to see the
   resemblance of this process to that of a compiler.

   A compiler takes source code and transforms it into a working program with
   the semantics described by that source code. The compilation process
   consists of several steps. First the source code is tokenized, which means
   the symbols in the code are identified one by one and classified in certain
   categories.

   Then the tokens are processed by a parser into an intermediary form called a
   parse tree. A parser looks for certain predefined patterns in the source
   code. These patterns are part of the source code's language syntax. As such,
   these two steps analyse and validate the source code's syntax. If part of
   the code does not match any pattern, the parser and the compilation process
   stop and the user gets a message saying the code's syntax is invalid.

   When a parse tree is constructed, the compilation process can alter it, to
   improve it. Certain patterns in the parse tree may be replaceable by
   different patterns with the same outcome, but with more optimal execution.
   This part of the compilation process is optional, and is called compiler
   optimization. Optimizations can consist of many things, depending on the
   language. For example, some languages guarantee tail call optimization,
   where infinite loops can be constructed by letting a function call itself as
   its last statement without causing a stack overflow. This is something the
   compiler (or interpreter) can optimize during this part of the compilation
   process.

   After this, the parse tree can be written out to produce the desired output.
   Every node in the tree has a well-defined equivalent in the target
   language's syntax. The target language can be Assembly, which consists of
   the exact instructions a CPU needs to carry out a program, or it can be
   another programming language. Many compilers of higher-level languages
   translate their language into C, for several reasons: the C compilers that
   translate C into Assembly have been optimized so much that it is easier to
   rely on them than to put an enormous amount of effort into optimizing
   another language; C compilers exist for most --- if not all --- CPU
   architectures, which means translating a language into C makes it compatible
   with all those architectures, while it would cost a lot more effort to write
   different compilers for every architecture you would want to make your
   language available on.

   The conversion tool that is the purpose of this thesis, can be described in
   a similar succession of steps. First, we take a \ppt presentation and
   tokenize and parse it into an intermediary structure that allows us to
   perform other operations, or `optimizations', on it. The intermediary form
   consists of a `parse tree' containing the components of the original
   presentation --- a component tree, if you will.

   With this structure, we can construct a \mxp presentation containing the
   same components in the same place, essentially creating a `program' with the
   same semantic meaning as the original `source'.

  \section{Compiler optimizations}

   Since the conversion process resembles that of a compiler, it seemed logical
   at first to make automatic layout a part of that process, as some kind of
   `compiler optimization'.

   At first, we tried to traverse the component tree, giving its objects new
   coordinates and sizes so that they would fit together on every slide as well
   as possible. This seemed an easy solution, but the results were sub-optimal.
   On top of that, we soon realised that we were in essence creating another
   template out of which a presentation would be made, which was exactly the
   opposite of what we were trying to do. As such, we abandoned this approach.

   We then switched to a different method: defining constraints for every
   component, in the form of margins, maximum sizes and other limits, and then
   calculating a way to satisfy all constraints while fitting content together
   on each slide. While this is clearly a better method, it turned out the
   compiler optimization phase was not the best place in the process to take
   care of this.

   In the end, we decided to take a different approach, relying on the layout
   engine of \mxp itself and enhancing that engine to create the automatic
   layout we wanteD.

  \section{Using \mxp}

   \mxp already takes care of layout for you, since that is one of its primary
   goals. The way it does this currently is however heavily based on templates,
   while we wanted a layout engine that could take any content and put in in an
   appropriate layout without any directions from the user. As such, we had to
   enhance \mxp's layout engine to use constraints, based on the size of the
   content, and try to find an optimal position for every component it is
   given.

   TODO we need more content here

