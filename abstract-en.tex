% vim:ts=1:et:nospell:spelllang=en_gb:ft=tex

 \chapter*{Abstract}

  Classic slide-based presentations are used worldwide to share and transfer
  knowledge. \ppt* is the most popular and well-known software package in this
  area. Unfortunately, little evolution has taken place since its inauguration
  back in the 1980s, despite an incredible amount of innovations in almost
  every other area of software and computer technology. The main concept of
  digital presentations has always remained the same, based on the original
  physical presentations using overhead or slide projectors. The limitations in
  size often force a less than optimal display of information, making it
  difficult for the presenter to explain and for the audience to understand the
  information that is being presented.

  Presenters often, if not usually, struggle to put their content into this
  format in a way that is clear to understand, aesthetically pleasing and
  well-structured. Most of the time spent creating a presentation is wasted on
  finding a proper layout for the content provided, frequently with suboptimal
  results. Professional designers spend years designing templates for these
  presentation tools, trying to automatically provide a one-size-fits-all
  solution for content that unsurprisingly mostly does not conform to the
  restrictions imposed upon it.
 
  \mxp is a presentation tool that brings a shift of paradigms in authoring and
  delivering presentations. It provides an extensible platform that allows a
  presenter to focus on the content of their presentation, while \mxp takes
  care of the visualisation. It consists almost entirely of plug-ins to process
  and visualise various content types, and allows the addition of new plug-ins
  to introduce new functionality as needed. The only limit in this system is
  the plug-in developer's imagination.

  In this thesis, we propose a tool and an approach to convert existing \ppt
  presentations into \mxp presentations, in the hopes of convincing \ppt users
  to switch to \mxp by showing them the possibilities of \mxp using their own
  content. As a second goal, we try to replace the default template-based
  layout system of \mxp with a layout engine that generates an ideal layout
  based on the content of a presentation.

% maybe TODO is this long enough?

