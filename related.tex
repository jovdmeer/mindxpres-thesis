% vim:ts=1:et:nospell:spelllang=en_gb:ft=tex

 \chapter{Related work}

  \emph{This chapter's content is largely based on ``MindXpres: An Extensible
  Content-driven Cross-Media Presentation Platform'' \citep{roels-1}.}

  \section{Background}

   The importance of digital presentations in this day and age cannot be
   understated. Millions of presentations are created every day, supporting the
   oral transfer of knowledge and playing an important role in educational
   settings. Their origins as tools for creating physical media such as
   photographic slides or transparencies for overhead projectors are still
   reflected in the underlying concepts and principles of slide-based
   presentation tools. The rectangular boundaries of a slide, and the linear
   navigation between slides, are still restrictions we face today in digital
   presentations. Tufte argues that these concepts of slideware have a negative
   impact on the effectiveness of knowledge transfer \citep{tufte-1}.  While
   the presenter is compelled to squeeze complex ideas into a linear sequence
   of slides, those ideas are rarely sequential by nature, resulting in a loss
   of relations, overview and details. An initial approach to address these
   issues might involve creating minimalistic presentations or introducing some
   structure via a table of contents. Sadly, when complex knowledge or other
   pieces of rich information need to be presented “as is” \citep{farkas-1} ---
   as in the domain of learning --- this does not work.

   One of the main issues with traditional slideware presentations is their
   monolithic nature, especially when content is spread over many
   self-contained presentation files. ``Reusing'' previous work involves either
   switching between files while giving a presentation or duplicating some
   slides in the new presentation. It should be noted that this issue is not
   limited to the reuse of single slides: there is an ever increasing wealth of
   resources available for reuse, spread over a wide spectrum of distribution
   channels and formats. The possibility to include content by reference or
   transclusion \citep{nelson-1} may contribute in crossing the boundaries
   between different types of media and prove beneficial in the context of
   modern cross-media presentation tools.

   The difference in functionality between the authoring of content and its
   visualisation is striking as well. The primary editors consist of mostly
   toolbars and buttons used for selecting and specifying the way content
   should be visualised, while support for authoring the content itself is not
   quite as extensive. Modern slideware has grown to include basic multimedia
   types such as videos, but most content is still rather static. It is, for
   example, not possible during a presentation to easily switch from a bar
   chart to a pie chart data visualisation, or to dynamically change some
   values in the represented data and immediately see the effect in the graph,
   which could be beneficial for knowledge transfer \citep{holzinger-1}. The
   audience could also be more actively involved in the presentation, through
   audience response and classroom connectivity systems providing multi-device
   interfaces allowing to share knowledge and results during as well as after a
   presentation. The evolution of presentations is reminiscent of the Web2.0
   movements where users have switched roles from purely consuming content to
   contributing as well, content has become more dynamic and interactive, and
   service-oriented architectures (``The Cloud'') have ensured decentralisation
   of content.

   In order to move a step towards the next generation of cross-media
   presentation tools, it is essential to allow the rapid prototyping and
   evaluation of new concepts for the representation, visualisation and
   interaction with content.
   
   %<TODO rework>
   After introducing existing slideware solutions, we discuss the requirements
   for next generation presentation tools. This is followed by a description of
   the extensible \mxp architecture and its plug-in mechanism. The web
   technology-based implementation of \mxp is validated based on a number of
   use cases and \mxp plug-ins and followed by a discussion of future work.
   %</TODO rework>

   A specific issue with slideware we'd like to focus on in this thesis, is the
   trouble with layout in presentations. It can be hard to display the content
   you want in a way that's clear, informative and nice to look at. The vast
   majority of layouts created today is mostly done by hand: a human graphic
   designer or ``layout expert'' makes most, if not all, of the decisions about
   the position and size of the objects to be presented \citep{lok-1}. Most
   software offers some templates, allowing you to drop pictures and text into
   predefined slots and places on a slide, but then those templates have been
   defined by someone else too. Computer-generated layout is rare and usually
   not quite up to the task.
  
   \mxp is among the software packages offering templates, in that layout is
   handled by whichever plug-in you choose, but so far no plug-ins have defined
   dynamic layout algorithms, rather sticking to predefined ways to put text
   and pictures on slides. But as \mxp does not constrain us to the limits of
   slides, this should be seen as an opportunity to offer dynamic layout as
   well. After all, if we're not limited to a certain area within which our
   content should fit, it should be much easier to put content next to each
   other in a way that makes sense.

  \section{Existing solutions}

   %<TODO rework>
   The impact, benefits and issues of slideware have been studied ever since
   digital slideware has been introduced. While some studies acknowledge
   slideware as a teaching aid \citep{holzinger-1}, Tufte \citep{tufte-1}
   heavily criticises slideware for sticking to outdated concepts. He addresses
   the many consequences of spatial limitations or linear navigation and
   relates them to the human mind which works differently. One of Tufte's
   conclusions, which is also confirmed by Adams \citep{adams-1}, is that
   slide-based presentations are not suitable for all kinds of knowledge
   transfer and in particular not in scientific settings. Recent work shows
   that it is important for the learning process that content is well
   integrated in the greater whole, both structurally and visually
   \citep{gross-1}, which is influenced by the navigation and visualisation.
   There have been a number of different approaches to offer non-linear
   navigation. Zoomable User Interfaces (ZUIs) as used by CounterPoint
   \citep{good-1}, Fly \citep{lichtschlag-1} or Prezi, offer virtually
   unlimited space. Also Microsoft has experimented with zoomable interfaces in
   pptPlex. While ZUIs are one way to escape the boundaries of the slide, we
   have seen other approaches such as MultiPresenter \citep{lanir-1} or tiling
   slideshows \citep{chen-1}. PaperPoint \citep{signer-1} and Palette
   \citep{nelson-2} further enable the non-linear navigation of digital
   presentations based on a slide selection via augmented paper-based
   interfaces. Finally, there is a category of authoring tools that use
   hypermedia to enable different paths through a set of slides.
   NextSlidePlease \citep{spicer-1} allows users to create a weighted graph of
   slides and may suggest navigational paths based on the link weights and the
   remaining presentation time. Microsoft follows this trend with their
   HyperSlides \citep{edge-1} project. The potential of \ppt* as an authoring
   tool for hypermedia-based presentations has further been investigated by
   Garcia \citep{garcia-1}.

   %<TODO this is my stuff!>
   \ppt* was officially released in 1990, with Windows 3.0 \citep{austin-1}. It
   had originally been developed as Presenter, but trademark issues caused a
   name change early on. It was also originally build for the Macintosh, which
   may seem surprising nowadays but was actually common practice back then
   since the Macintosh was widely regarded as a better development environment,
   more mature, more stable and capable of far better performance and
   visualisations. Some may argue this still rings true today.

   Since then, it has grown to be the world's most popular slide show
   presentation program, alledgedly having been installed on over 1 billion
   computers worldwide, and being used on average 350 times \emph{per second}
   \citep{parks-1}. In 2012, it had a market share of 95\%, leaving the other
   5\% to be shared by alternatives such as Apple's Keynote, Prezi, SlideRocket
   and others. While this number is declining, it may not be going as fast as
   many people think. As most readers of this thesis have heard before, over 30
   million \ppt presentations are created every day, for all kinds of purposes,
   with good and bad results both presentation-wise and goal-wise.
   %</TODO this is my stuff!>

   Existing presentation tools require content to be duplicated for reuse,
   resulting in multiple redundant copies that need to be kept up to date.
   Even though some attempts have been made to address this issue, there is
   room for improvement. When it comes to document formats for more general
   educational purposes, there are formats such as the Learning Material Markup
   Language (LMML) \citep{suss-1}, the Connexions Markup Language (CNXML) and
   the eLesson Markup Language (eLML) \citep{fisler-1}. The common factor of
   these formats is their focus on the reuse of content, but always at a
   relatively high granularity level. Content is organised in lessons or
   modules and users are encouraged to use these, as a whole, in their
   teaching. When we examined the formats in more detail, we noticed that they
   support outgoing links to external content but not the inclusion of content
   via references (transclusion). In the context of presentations, Microsoft's
   Slide Libraries are central repositories for the storage of slides in order
   to facilitate slide sharing and reuse within a company. However, one needs
   to set up a SharePoint server which might represent a hurdle for some users.
   Slides still need to be searched and manually copied into presentations.
   Furthermore, users are responsible to push back updates to the repository or
   update slides when they have been modified on the server side. Other
   commercial tools with similar intentions and functionality for content reuse
   are SlideRocket or SlideShare. The SliDL \citep{canos-1} research framework
   provides a service-oriented architecture for storing and tagging slides in a
   database for reuse. However, it shares some of the shortcomings of
   Microsoft's Slide Libraries. The ALOCOM \citep{verbert-1} framework for
   flexible content reuse consists of a content ontology and a (de)composition
   framework for legacy documents including \ppt documents, Wikipedia pages and
   SCORM content packages. While ALOCOM succeeds in the decomposition of legacy
   documents, it might be too rigid for evolving presentation formats and the
   tool is furthermore only supporting the authoring phase.

   There is not only a similarity in the evolution of the Web and presentation
   environments, but a number of the issues presented in this section have
   solutions in the setting of the Web. It is therefore not surprising that
   more recently we see the use of web technologies for realising presentation
   solutions. The Simple Standards-based Slide Show System
   (S5)\footnote{http://meyerweb.com/eric/tools/s5/} is an XHTML-based file
   format for slideshows which enforces the classical slideware model. The
   Slidy \citep{raggett-1} initiative by the W3C introduces another
   presentation format which is based on the standard slideware model. While
   these two formats are too limited for our needs, they have some interesting
   properties. Both formats show a clean separation of content and presentation
   via CSS themes. The visualisation is resolution independent and the layout
   and font size are adapted to the available screen real estate. Finally, we
   would like to mention recent HTML5-based presentation solutions, including
   projects such as impress.js, deck.js, Shower or reveal.js. A major benefit
   of applying a widely used open standard such as HTML is the cross-device
   support. Nevertheless, also these solutions show some restrictions in terms
   of visualisation, navigation and cross-media support.

   Most of the tools and projects presented in this section focus on specific
   novel ideas for presentations. However, there is no interoperability between
   the concepts introduced in different tools. While one project might focus on
   the authoring, another one focusses on novel content types and a third
   solution introduces radically new navigation mechanisms. Some slideware
   tools can be extended via third-party plug-ins but the functionality that is
   exposed to the developers is often limited by the tool's underlying model.
   For example, \ppt allows plug-ins to interact with the presentation model,
   but the model dictates that a presentation must consist of a sequence of
   slides. This lack of freedom is also a shortcoming of existing web-based
   presentation formats. We therefore see a need for an open presentation
   platform such as \mxp which supports innovation by providing the necessary
   modularity and interoperability \citep{bush-1}.
   %</TODO rework>

   It is perhaps surprising that, to our knowledge, currently no tools exist to
   calculate dynamic layouts of content in slideware. Existing solutions
   include template systems, sometimes very fine-grained like \latex allowing
   you to define templates for every single layout choice, usually more coarse
   like \ppt* or Apple Keynote using Master Slides to define different layouts
   on a per-slide basis, and always with the option of letting the user
   customize the layout by hand, literally manually moving the content to the
   exact place where we want it, unhindered by style guides, good practices or
   common sense. This has resulted in mindboggling layout choices involving
   enormous amounts of tiny text crammed onto one slide, or pictures strewn
   across a slide overwhelming the audience with too much information at once. 

  \section{New solutions}

   %<TODO rework>
   We now introduce a number of requirements to support a broad range of
   presentation styles and visualisations which have been compiled based on a
   review of the more recent presentation solutions presented in the previous
   section.

     \paragraph{Non-linear Navigation} As outlined earlier, the linear
      traversal of slides is a concept that has been taken over from early
      photographic slides. Nowadays, users are accustomed to this form of
      navigation even if it might come with some disadvantages. Any navigation
      outside of the predefined linear path (e.g. to answer a question from the
      audience) is rather complicated, since the presenter either needs
      substantial time to scroll forwards or backwards to the desired slide or
      has to switch to the slide sorter view. It is further impossible to
      include a single slide multiple times in the navigational path without
      any duplication. There are different ways how this lack of flexible
      navigation can be addressed, including presentation tools that allow the
      presenter to define non-linear navigation paths \citep{spicer-1}
      \citep{edge-1} or zoomable user interfaces (ZUIs) \citep{good-1}
      \citep{lichtschlag-1} \citep{haller-1}.

     \paragraph{Separation of Content and Presentation} In order to facilitate
      experimentation with different visualisations, there should be a clear
      separation between content and presentation. This allows the authors of a
      presentation to focus on the content while the visualisation is handled
      by the presentation tool. Note that this approach is similar to the
      \latex typesetting system where content is written in a standardised
      structured way and the visualisation is automatically handled by the
      typesetting system. There is also a \latex document class for
      presentations called Beamer and we were inspired by its structured and
      content-driven approach. However, the content-related functionality and
      the visualisation are too limited to be considered as a basis for an
      extensible presentation tool.

     \paragraph{Extensibility} In order for a presentation tool to be
      successful as an experimental platform for new presentation concepts, it
      should be easy to rapidly prototype new content types and presentation
      formats as well as innovative navigation and visualisation techniques. It
      has to be possible to add or replace specific components without
      requiring changes in the core. In order to be truly extensible, a
      presentation tool should provide a modular architecture with loosely
      coupled components. Note that this type of extensibility should not only
      be offered on the level of content types but also for the visualisation
      engine or content structures.

     \paragraph{Cross-Media Content Reuse} In the introduction we briefly
      mentioned the lack of content reuse in existing presentation tools. There
      is a wealth of open education material available but it is rather
      difficult to use this content in presentations. On the other hand, the
      concept of transclusion works well for digital documents and parts of the
      Web (e.g. via the HTML img tag). A modern presentation tool should also
      support the seamless integration of external cross-media content. This
      includes various mechanisms for including parts of other presentations
      (e.g. slides), transcluding content from third-party document formats as
      well as including content from open learning repositories.

     \paragraph{Connectivity} With the rise of social and mobile technologies,
      connectivity for multi-device input and output becomes more relevant in
      the context of presentation tools. Support for multi-directional
      connectivity is required for a number of reasons. First, it is necessary
      for the previously mentioned cross-media transclusion from external
      resources. Second, multi-directional connectivity forms the backbone for
      audience feedback via real-time response or voting systems
      \citep{dufresne-1} as well as other forms of multi-device interfaces.

     \paragraph{Interactivity} We mentioned that content might be more
      interactive and the extensibility requirement addresses this issue since
      the targeted architecture should support dynamic or interactive content
      and visualisations. Nevertheless, the use of mouse and keyboard might not
      be sufficient for components offering a high level of interaction.
      Therefore, a presentation tool should enable the integration of other
      forms of input such a gesture-based interaction based on Microsoft's
      Kinect controller or digital pen interaction \citep{signer-2} as offered
      by the PaperPoint \citep{signer-1} presentation tool.

     \paragraph{Post-Presentation Phase} Even if it was never the original goal
      of slide decks, they often play an important role as study or reference
      material. While the sharing of traditional slide decks after a
      presentation is trivial, this changes when the previously mentioned
      requirements are taken into account. For instance, the nonlinear
      navigation allows presenters to go through their content in a non-obvious
      order or input from the audience might drive parts of a presentation.
      Special attention should therefore be paid to the post-presentation
      phase. It should not only be easy to play back a presentation with the
      original navigational path, annotations and audience input, but its
      content should also be made discoverable and reusable. In accordance with
      the Web 2.0, we see potential for the social aspect in a
      post-presentation phase via a content discussion mechanism.

  \section{\mxp Platform}
   \label{mxp-platform}

   In this section, we present the general architecture of our
   \mxp\footnote{http://mindxpres.com} crossmedia presentation platform which
   is outlined in \figref{roels-1-fig-1} and addresses the requirements
   presented in the previous section.

   \fig{roels-1-fig-1}{\mxp architecture}

   \subsection{Document Format and Authoring Language}

    Content is stored, structured and referenced in a dedicated \mxp document
    format. An individual \mxp document contains the content itself and may
    also refer to some external content to be included. A new \mxp document can
    be written manually similar to the \latex approach introduced earlier or in
    the near future it can also be generated via a graphical authoring tool. In
    contrast to other presentation formats such as Slidy, S5 or OOXML, the
    authoring language eliminates unnecessary HTML and XML specifics and
    focusses on a semantically more meaningful vocabulary. The vocabulary of
    the authoring language is almost completely defined by plug-ins that
    provide support for various media types and structures. In order to give
    users some freedom in the way they present their information, the core \mxp
    presentation engine only plays a supporting role for plug-ins and lets them
    define the media types (e.g. video or source code) as well as structures
    (e.g. slides or graph-based content layouts).

    This is also reflected in the document format as each plug-in extends the
    vocabulary that can be used. Any visual styling including different fonts,
    colours or backgrounds is achieved by applying specific themes to the
    underlying content.

   \subsection{Compiler}

    The compiler transforms a \mxp document into a self-contained portable \mxp
    presentation bundle. While a \mxp document could be directly interpreted at
    visualisation time, for a number of reasons we decided to have this
    intermediary step. First, the compiler allows different types of
    presentations to be created from the same \mxp document instance. This
    means that we can not only create dynamic and interactive presentations but
    also more static output formats such as PDF documents for printing.
    Similarly, we cannot always expect that there will be an Internet
    connection while giving a presentation. For this case, the compiler might
    create an offline version of a presentation with all necessary content
    pre-downloaded and included in the \mxp presentation bundle.  Last but not
    least, the compiler might resolve incompatibility issues by, for instance,
    converting unsupported video formats.

   \subsection{\mxp Presentation Bundle}

    The dynamic \mxp presentation bundle consists of the compiled content
    together with a portable cross-platform presentation runtime engine which
    allows more interactive and networked presentations. Similar to the
    original document, the compiled presentation content still consists of
    both, integrated content and references to external resources such as
    online content that will be retrieved when the presentation is visualised.
    Note that the content might have been modified by the compiler and, for
    example, been converted or extracted from other document formats that the
    runtime engine cannot process.  References to external content may have
    been dereferenced by the compiler for offline viewing.

    A presentation bundle's core runtime engine consists of the three modules
    shown in \figref{roels-1-fig-1}. The \emph{content engine} is responsible
    for processing the content and linking it to the corresponding
    visualisation plug-ins. The \emph{graphics engine} abstracts all
    rendering-related functionality. For instance, certain presenters prefer a
    zoomable user interface in order to provide a better overview of their
    content \citep{reuss-1}. This graphical functionality is also exposed to
    the plug-ins, which can make use of the provided abstractions. The
    \emph{communication engine} exposes a communication API that can be used by
    plug-ins. It provides some basic functionality for fetching external
    content but also offers the possibility to form networks between multiple
    \mxp presentation instances as well as to connect to third-party hardware
    such as digital pens or clicker systems.

    In addition to the presentation content and core modules, the presentation
    bundle contains a set of \emph{themes} and \emph{plug-ins} that are
    referenced by the content. Themes may contain visual styling on a global as
    well as on a plug-in level. When the content engine encounters different
    content types, they are handed over to the specific plug-in which uses the
    graphics engine to visualise the content.

   \subsection{Plug-in Types}

    \fig{roels-1-fig-2}{Structure plug-in examples}

    In order to provide the necessary flexibility, all non-core modules are
    implemented as plug-ins. Even the basic content types such as text, images
    or bullet lists have been realised via plug-ins and three major categories
    of plug-ins have to be distinguished:

    \begin{itemize}

     \item \emph{Components} form the basic building blocks of a presentation.
     They are represented by plug-ins that handle the visualisation for
     specific content types such as text, images, bullet lists, graphs or
     videos. The content engine invokes the corresponding plug-ins in order to
     visualise the content.

     \item \emph{Containers} are responsible for grouping and organising
     components of a specific type. An example of such a container is a slide
     with each slide containing different content but also some reoccurring
     elements.  Every slide of a presentation may for example contain elements
     such as a title, a slide number and the author's name, which can be
     abstracted in a higher level container. Another example is an image
     container that visualises its content as a horizontally scrollable list of
     images. Note that we are not restricted to the slide format and content
     can be laid out in alternative ways.

     \item \emph{Structures} are high-level structures and layouts for
     components and containers. For example, content can be scattered in a
     graph-like structure or it can be clearly grouped in sections like in a
     book.  Both are radically different ways of visualising and navigating
     content but by abstracting them as plug-ins, the user can easily switch
     between different presentation styles as the ones shown in
     \figref{roels-1-fig-2}. Structures differ from containers by the fact that
     they do not impose restrictions on the media types of their child elements
     and may also influence the default navigational path through the content.

    \end{itemize}

   \subsection{Implementation}

    HTML5 and its related web technologies were chosen as the backbone for
    the \mxp presentation platform. Other options such as JavaFX, Flash or game
    engines have also been investigated, but HTML5 seemed to be the best choice.
    The widely accepted HTML5 standard makes \mxp presentations highly portable and
    runnable on any device with a recent web browser, including smartphones and
    tablets. Furthermore, HTML5 provides rich visualisation functionality out of
    the box and the combination with Cascading Style Sheets
    (CSS) and third-party JavaScript libraries forms a potent visualisation
    platform.

    \subsubsection{Document Format and Authoring Language}

     The \mxp document format which allows us to easily express a
     presentation's content, structure and references is based on the
     eXtensible Markup Language (XML). A simple example of a presentation
     defined in our XML-based authoring language is shown in \lstref{xml-1}.
     The set of valid tags and their structure, except the \code{presentation}
     root tag, is defined by the available plug-ins.

     \begin{figure}[h!]
      \begin{lstxml}{xml-1}{Authoring a simple \mxp presentation}
<presentation>
  <slide title="Vannevar Bush">
    <bulletlist>
      <item>March 11, 1890 - June 28, 1974</item>
      <item>American Engineer, founder of Raytheon</item>
    </bulletlist>
    <image source="bush.jpg"/>
  </slide>
</presentation>
      \end{lstxml}
     \end{figure}

    \subsubsection{Compiler}

     The compiler has been realised as a Node.js application. This not only
     allows the compiler to be used via a web interface or as a web service,
     but projects such as node-webkit also enable the compiler to run as a
     local offline desktop application. The choice of using server-side
     JavaScript was influenced by the fact that Node.js is capable of bridging
     web and desktop technologies. On the one hand, the framework makes it easy
     to interact with other web services and to work with HTML, JSON, XML and
     JavaScript visualisation libraries at compile time. On the other hand, the
     framework can also perform tasks which are usually not suited for web
     technologies, including video conversion, legacy document format access,
     file system access or TCP/IP connectivity.

     In order to validate a \mxp document in the XML format described above,
     there is an XML Schema which is augmented with additional constraints
     provided by the plug-ins. After validation, the document is parsed and
     discovered tags might trigger preprocessor actions by the plug-ins such as
     the extraction of data from referenced legacy document formats (e.g.  \ppt
     or Excel) or the conversion of an unsupported video format. The tag is
     then converted to HTML5 by simply encoding the information in the
     attributes of a \code{div} element. The HTML5 standard allows custom
     attributes if they start with a \code{data-} prefix. \lstref{xml-2} shows
     parts of the transformed XML document shown in \lstref{xml-1}. Note that
     the transformation does not include visualisation-specific information but
     merely results in a valid HTML5 document which is bundled into a
     self-contained package together with the presentation engine.

     \begin{figure}[h!]
      \begin{lstxml}{xml-2}{Transformed HTML5 presentation content}
<div data-type="presentation">
  <div data-type="slide" data-title="Vannevar Bush">
    <div data-type="bulletlist">
      ...
      \end{lstxml}
     \end{figure}

    \subsubsection{Presentation Engine}

     The presentation engine's task is to turn the compiled HTML content into a
     visually appealing and interactive presentation. As highlighted in
     \figref{roels-1-fig-1}, the presentation engine consists of several
     smaller components which help plug-ins to implement powerful features with
     minimal effort. The combination of these components enables the rapid
     prototyping and evaluation of innovative visualisation ideas. A resulting
     \mxp presentation combining various structure, container and component
     plug-ins is shown in \figref{roels-1-fig-3}.

     \fig{roels-1-fig-3}{A \mxp presentation}

     \paragraph{Content Engine} When a presentation is loaded, the content
      engine is the first component that is activated. It processes the content
      of the HTML presentation by making use of the well-known jQuery
      JavaScript library. Whenever a \code{div} element is discovered, the
      \code{data-type} attribute is read and the corresponding plug-ins are
      notified in order to visualise the content.

     \paragraph{Graphics Engine} The graphics engine provides support for
      interesting new visualisation and navigation styles. Next to some basic
      helper functions, it offers efficient panning, scaling and rotation via
      CCS3 transformations and supports zoomable user interfaces as well as the
      more traditional navigation approaches.

     \paragraph{Communication Engine} The communication engine implements
      abstractions that allow plug-ins to retrieve external content at run
      time. It further provides the architectural foundation to form networks
      between different \mxp instances or to integrate third-party hardware
      \citep{roels-2}. For our \mxp prototype, we used a small Intel Next Unit
      of Computing Kit (NUC) with high-end WiFi and Bluetooth modules to act as
      a central access point and provide the underlying network support. \mxp
      instances use WebSockets to communicate with other \mxp instances via the
      access point. The access point further acts as a container for data
      adapters which translate input from third-party input and output devices
      into a generic representation that can be used by the \mxp instances in
      the network. In order to go beyond simple broadcast-based communication,
      we have implemented a routing mechanism based on the publish-subscribe
      pattern where plug-ins can subscribe to specific events or publish
      information. The communication engine provides the basis for audience
      response systems \citep{roels-2} or even full classroom communication
      systems where functionality is only limited by the creativity of plug-in
      developers.

     \paragraph{Plug-ins} Plug-ins are implemented as JavaScript bundles which
      consist of a folder containing JavaScript files and other resources such
      as CSS files, images or other JavaScript libraries. As a first
      convention, a plug-in should provide a manifest file with a predefined
      name. The manifest provides metadata such as the plug-in name and version
      but also a list of tags to be used in a presentation. The plug-in claims
      unique ownership for these tags and is in charge for their visualisation
      if they are encountered by the content engine. As a second convention, a
      plug-in must provide at least one JavaScript file implementing certain
      methods, one of them being the \code{init()} method which is called when
      the plug-in is loaded by the presentation engine. It is up to the plug-in
      to load additional JavaScript or CSS via the provided dependency
      injection functionality. A second method to be implemented is the
      \code{visualise()} method which is invoked with a pointer to the
      corresponding DOM node as a parameter when the content engine encounters
      a tag to be visualised. A plug-in is free to modify the DOM tree and may
      also register callbacks to handle future interaction with the content.

     \paragraph{Themes} We currently use CSS to provide a basic templating
      system. These themes offer styling either on a global or on a plug-in
      level. However, we see this as a temporary solution as it is not
      well-suited for alternative compiler outputs (e.g. PDF) and a more
      generic templating scheme is planned for the future.

   \subsection{Use Cases}

    In order to validate the architectural and technological choices, we
    demonstrate the extensibility and feasibility of \mxp as a rapid
    prototyping platform by presenting a number of content- and
    navigation-specific plug-ins that have been developed so far. Additional
    plug-ins for audience-driven functionality such as real-time polls, screen
    mirroring and navigational takeover can be found in \citep{roels-2}.

    \subsubsection{Structured Overview Plug-in}

     In Section \ref{mxp-platform} we have explained how structure plug-ins may
     change the way presentations are visualised and navigated. In order to
     illustrate this, we have implemented a structure plug-in called
     \emph{structured layout} which combines a zoomable user interface with the
     ability to group content into sections. The resulting visualisation of the
     \emph{structured layout} plug-in is shown in \figref{roels-1-fig-3}.
     Whenever a new section is reached, the view is zoomed out to provide an
     overview of the content within the section and communicate a sense of
     progress.

    \subsubsection{Slide Plug-in}

     In order to also support the traditional slide concept, we created a
     slide-like container plug-in. While the benefits and issues of using
     slides with a fixed size are debatable, we implemented this plug-in as a
     proof of the framework's versatility. The main function of the slide
     plug-in is to provide a rectangular styleable component container with a
     title and some other information. Containers may also offer functionality
     to layout their content. In this case, the slide plug-in offers a quick
     and easy layout mechanism which allows the presenter to partition the
     slide into rows and columns. Content is then assigned to these slots in
     the order that it is discovered. The use of the slide plug-in together
     with the resulting visualisation is exemplified in \lstref{roels-1-fig-4}.
     It also shows the use of the image plug-in (a component plug-in) which
     enables a simple form of cross-media transclusion. A visualised external
     image can be cropped and filters (e.g. colour correction) may be applied
     without duplicating or modifying the original source.

     \begin{figure}[h!]
      \begin{subfigure}{0.53\textwidth}
       \vspace{-1em}
       \begin{lstxml*}
<slide layout="\60\40" title="Vannevar Bush">
 <bulletlist>
  <item>About
   <item>March 11, 1890 - June 28, 1974</item>
   <item>American Engineer</item>
   <item>Founder of Raytheon</item>
  </item>
  <item>...
  </item>
 </bulletlist>
 <image source="http://example.com/bush.jpg">
  <crop bounds="10%, 5%, -10%, -20%" />
 </image>
</slide>
       \end{lstxml*}
      \end{subfigure}
      \hfill
      \begin{subfigure}{0.43\textwidth}
       \fignl{roels-1-fig-4}
      \end{subfigure}
      \renewcommand{\figurename}{Listing}
      \renewcommand{\figureshortname}{Lst.}
      \cl{roels-1-fig-4}{Slide plug-in}
     \end{figure}

    \subsubsection{Enhanced Video Plug-in}

     When videos are used in educational settings, we often need more
     functionality than what is offered by the average video player
     \citep{reuss-1}. \mxp provides the enhanced video plug-in shown in
     \lstref{roels-1-fig-5} with the possibility to overlay a video with text
     or arbitrary shapes. This overlay functionality can be used as a basic
     captioning system as well as to highlight items of interest during
     playback.

     Furthermore, we added the option to trigger certain events at specified
     times. One can define that a video should automatically pause at a certain
     point, highlight an object and continue playing after a specified amount
     of time. Additional features include the bookmarking of certain positions
     in a video for direct access or the possibility to display multiple videos
     in a synchronised manner. Our enhanced video plug-in injects the default
     HTML5 video player and overlays it with a transparent \code{div} element
     for augmentation. Currently we make use of the HTML5 video API to
     synchronise the creation and removal of overlays but a SMIL-based
     implementation might be used in the future.

     \begin{figure}[h!]
      \begin{subfigure}{0.53\textwidth}
       \vspace{-1em}
       \begin{lstxml*}
<video source="vid.mp4">
 <caption start="0:00" duration="1500ms">
  Lecture 3 - Butterfly Species
 </caption>
 <pause start="0:43" duration="5s">
  <caption>
   The peacock butterfly (aglais io) ...
  </caption>
  <highlight x="30%" y="9%"
             width="35%" height="40%" />
 </pause>
</video>
       \end{lstxml*}
      \end{subfigure}
      \hfill
      \begin{subfigure}{0.43\textwidth}
       \fignl{roels-1-fig-5}
      \end{subfigure}
      \renewcommand{\figurename}{Listing}
      \renewcommand{\figureshortname}{Lst.}
      \cl{roels-1-fig-5}{Enhanced video plug-in}
     \end{figure}

 \newpage

    \subsubsection{Source Code Visualisation Plug-in}

     Earlier, we mentioned the difficulty of visualising complex resources such
     as source code. Our \mxp source code plug-in exports a \code{code} tag
     which allows the presenter to paste their code into a presentation and
     have \mxp visualise it nicely by making use of syntax highlighting via the
     SyntaxHighlighter\footnote{http://alexgorbatchev.com/SyntaxHighlighter/}
     JavaScript library. Whenever the content engine encounters a \code{code}
     tag, it invokes the code plug-in to beautify the code and automatically
     adds vertical scrollbars for larger pieces of source code as shown in
     \lstref{roels-1-fig-6}.

     \begin{figure}[h!]
      \begin{subfigure}{0.53\textwidth}
       \vspace{-1em}
       \begin{lstxml*}
<code>
 <publications>
  <publication type="inproceedings">
   <title>An Architecture for Open Cross-Media
          Annotation Services</title>
   <author>
     <surname>Signer</surname>
     <forename>Beat</forename>
   </author>
   <author>
     <surname>Norrie</surname>
     <forename>Moira</forename>
   ...
</code>
       \end{lstxml*}
      \end{subfigure}
      \hfill
      \begin{subfigure}{0.43\textwidth}
       \fignl{roels-1-fig-6}
      \end{subfigure}
      \renewcommand{\figurename}{Listing}
      \renewcommand{\figureshortname}{Lst.}
      \cl{roels-1-fig-6}{Source code visualisation}
     \end{figure}

   \subsection{Discussion and Future Work}

    \mxp currently supports transclusion and cross-media content reuse on the
    plug-in level. For instance, the image or video plug-in can visualise (and
    enhance) external resources, a dictionary plug-in might retrieve
    definitions on demand via a web service or we might create a plug-in that
    allows us to import content (e.g. \ppt slides) from legacy documents at
    compile time. Nevertheless, we are currently investigating the introduction
    of generic reuse tags in our document format which would allow the
    presenter to transclude arbitrary parts of other \mxp presentations. While
    our focus has been on the cross-media aspect of resources that can be used
    in a presentation, we might also investigate the cross-media publishing
    aspect via alternative compiler output formats.

    We are aware that the current authoring of \mxp presentations has some
    usability issues. The average presenter cannot be expected to construct an
    XML document or any CSS themes. In order to tackle this issue and further
    evaluate \mxp in real-life settings, we are currently developing a
    graphical \mxp authoring tool. We further intend to provide a central
    plug-in repository which would make it easy for novice users to find,
    install and use new plug-ins via the authoring tool. In the long run, we
    intend to revise the use of monolithic documents and move towards
    repositories of semantically linked information based on the RSL hypermedia
    metamodel \citep{signer-3}. This would not only promote content reuse and
    sharing, but also create opportunities for context-aware as well as
    semi-automatic presentation authoring where relevant content is recommended
    by the authoring tool.

   \subsection{Conclusion}

%    TODO do we need this?
   
    We have presented \mxp, an HTML5 and web technology-based presentation
    platform addressing the lacking extensibility of existing slideware
    solutions. The extensibility of the \mxp platform heavily relies on a
    plug-in mechanism and we have outlined how the tool can be extended on the
    content, visualisation as well as on the interaction level. By providing
    different forms of cross-media content transclusion, our solution further
    avoids the redundant storage and replication of slides. The presented \mxp
    document format in combination with a compiler and the corresponding
    plug-ins offers the opportunity to have compile- or run-time cross-media
    content transclusion from third-party resources. At the same time, the
    flexible and extensible document model in combination with a zoomable user
    interface allows us to escape the boundaries of traditional slide formats.
    The presented \mxp solution represents a promising platform for the
    unification of existing presentation concepts as well as for the rapid
    prototyping and investigation of new ideas for next generation cross-media
    presentation solutions.
   %</TODO rework>

% --- BEGIN my crappy stuff
%    TODO ... or this?

    \mxp tries to be an alternative unlike other alternatives, stepping away
    from the classic slide format and introducing a plug-in system to allow
    literally anything you can think of. The obvious choices are
    visualisations, animations, layout and the embedding of various media such
    as video, but it can also provide interaction with the audience, allow
    different presenters to take control from their own device, arbitrarily and
    ad-hoc change the way data is presented allowing to play into the
    audience's reactions unhindered by design decisions made beforehand, and so
    many other novel ways of presenting that captivate the audience instead of
    lulling them to sleep.

    \mxp was first created by my advisor, Reinout Roels, in 2012 as part of his
    Master thesis, and has since been built upon by himself and other students.

% --- END my crappy stuff

  \section{Layout}

   Proper layout is incredibly important when trying to transfer knowledge and
   information through written and visual media. Layout can help clarify
   boundaries and relations between pieces of information, by grouping and
   separating them appropriately. Layout is one component of a presentation's
   design, that --- combined with other decisions --- determines the number and
   nature of the visual representations of the information the creator wants to
   communicate, along with its format\footnote{The way the visual objects are
   realized (e.g. as text, graphics, UI widgets\ldots), and their attributes
   (e.g. color, texture, font\ldots)}. The layout of a presentation can have a
   tremenduous influence on its effectiveness in communicating information to,
   and obtaining information from, the audience it is meant to interact with.
   The importance of individual objects can be emphasized or minimized, and the
   connection between obects can be clarified or blurred. A well laid out
   presentation can provide a narrative for the viewer to discover, inferring
   correct links between the objects along the way, and to accomplish tasks
   quickly and correctly, increasing the presentation's effectiveness.
  
   Creating a good layout is almost never easy. People often spend more time on
   the layout of their presentation than the content. Most, if not all,
   decisions in layout are made by human beings. Some of them are professional
   designers who spend years learning and figuring out how to create effective
   layouts, and even then they may take hours or days to create even a single
   screen of a presentation. In fact, the more someone knows about proper
   layout and design, the more time they may spend perfectioning their work.
   However, sometimes time-critical information must be communicated and the
   layout process is too expensive and too slow to address these situations.
   This can be a serious problem (see also section \ref{nasa}). Many software
   packages have been developed to make this process easier, to get better
   results, to give more or less control to the creators. Many different
   approaches have been taken, and yet most of them still involve having a
   human being make the final decisions on the layout.

   There are tools like \ppt, which give you some guidelines and some templates
   but generally let you do your own thing. If your own thing is entirely
   different from any best practices on layout, nothing will stop you. Other
   tools like \latex give you complete control over every aspect of the layout,
   while setting some sensible defaults so that you can get a good-looking
   layout without much effort, while still letting you do whatever you want
   once you overcome the steep learning curve that separates the casual users
   from the experts. There are tools that combine the power of \latex with the
   comfort of WYSIWYG editors, bringing the casuals a bit closer to the
   experts. But all of those tools have one thing in common: every aspect of
   every layout they create has, at some point, been designed and decided upon
   by a human being.
  
   Aesthetics are a natural phenomenon, and the creation of aesthetically
   pleasing layouts is therefore a manifestation of our instincts. As with most
   instincts, it has proven difficult to translate this into a concept that can
   be understood by a computer. Moreover, it is still difficult to explain it
   in human terms, which --- according to a popular quote often attributed to
   Albert Einstein --- proves we don't fully understand it ourselves.

   When we look to other technologies, we do find some automatic layout
   implementations. For example, the web has had to adapt to mobile devices
   with small screens over the past few years, and has done this gracefully by
   creating the concept of responsive design. In short, this allows websites to
   adapt their layout to any screen, no matter the size. While this is often a
   hard-coded difference, where effectively two or more versions of the same
   webpage are created aimed at different screen sizes, some websites take a
   more dynamic approach based on constraints. As the space the page is to be
   displayed on gets smaller, the layout algorithm may decide to display
   content below other content instead of side-by-side, it may scale images to
   fit the screen, it may even switch fonts and font sizes if necessary.

   This constraint-based technique is described in a few papers
   \citep{lok-1,hurst-1}, but has --- to our knowledge --- not been applied in
   any presentation software so far. This is surprising, as presentations often
   look like they could use some of this magic. A proper constraint-based
   layout algorithm could allow any user to drop content onto a slide, without
   worrying about clarity or even legibility, and the algorithm could take care
   of the rest. Of course, there are some limits in traditional slideware that
   may hinder this approach: if a user decides to put more content on a slide
   than there is physical room available, the algorithm could either make the
   content smaller or split it across several slides, but either solution may
   bring its own problems up. An advantage of ZUI's is that no matter how small
   the content gets, we can still zoom in to make it clear again\footnote{It
   should be noted that \mxp in its current form does not support this level of
   zooming. While the software can zoom out to provide an overview of the
   presentation while zooming in on the separate components, it is not yet
   possible to zoom in or out extremely to reveal 'hidden' parts of the
   presentation. This is something we encourage to look into and change,
   because it can greatly improve both our layout solution as well as the whole
   \mxp experience in general.}.

%   TODO more stuff from papers about automatic layout goes here

   \subsection{Algorithms}

    When researching layout algorithms, one will often come across the very
    active field of graph layout \citep{battista-1}. We will not go into the
    specifics of this field, as most of the issues with which it is concerned
    are specific to problems caused by the explicit visual representation of
    graph edges --- for example, the minimization of edge crossing
    \citep{battista-2, shahrokhi-1}. The same applies to automated layout as
    referring to automated circuit layout for VLSI chip fabrication
    \citep{hu-1, lengauer-1} as well as automated placement of pieces to be cut
    from a bolt of cloth used to produce clothing \citep{milenkovic-1}.
    Contrary to presentation layouts (including graph layouts), these layouts
    are designed to meet the requirements of a fabrication process, rather than
    to make them understandable to humans. While some techniques used therein
    definitely apply to our more general problem of automated presentation
    layout (e.g. general constraint solvers) others decidedly do not (e.g.
    bin-packing techniques \citep{hofri-1} that result in minimal area layouts
    at the expense of maintaining visually obvious relationships between
    objects).

