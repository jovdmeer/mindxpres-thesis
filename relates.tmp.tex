% vim:ts=1:et:nospell:spelllang=en_gb:ft=tex

		In the remainder of this paper, we first discuss some of the simple
		approaches to presentation layout used in current commercial software in
		Section \ref{simple-techniques}. Next, we focus on the two methods that
		have been adopted by research systems that address automated presentation
		layout: constraints and learning. As described in Section
		\ref{constraint-satisfaction}, most research layout systems assume that a
		layout can be described by a set of constraints. Much of this work
		addresses applying constraint solvers and developing new ones. In addition,
		many researchers have focused on how to determine or extract the right
		constraints as well as how to to express them to a computer program.
		Systems that are not constraint-based generally fall into the category of
		learning systems, which we discuss in Section \ref{learning-techniques}.
		These systems are sometimes trained by experts and other times by the user
		as the system assists in the layout process. In Section
		\ref{evaluation-techniques}, we introduce some of the issues involved in
		evaluating presentation layouts. Finally, Section
		\label{conclusions-layout-paper} presents our conclusions and ideas for
		future work.

		\section{Simple techniques}
		\label{simple-techniques}

		Almost all contemporary user interfaces are built with a UI toolkit (e.g.,
		Sun JFC/Swing \citep{sun-1}, Microsoft Foundation Classes
		\citep{microsoft-1}, and their ancestors, such as Xtk \citep{mccormack-1}
		or Tk \citep{ousterhout-1}) that provides basic functionality, such as
		creating buttons and windows. Toolkits include layout managers (also known
		as geometry managers) to assist the UI designer (programmer) in designing
		the layout of objects in a managed container, without having to specify the
		absolute position and size of each object. Layout managers allow the
		programmer to say, in effect, “add button” or “add button to this part of
		the window,” and optionally specify additional numeric constraints. This
		makes it possible to create layouts that adapt to changes in the
		container’s size.

		A layout manager chooses positions and sizes at run time for the objects
		that it controls, governed by a set of constraints imposed by a simple
		layout policy built into the manager and parameters specified by the
		programmer. Typical layout policies include strict horizontal (row) or
		vertical (column) layout; row-major or column-major layout in which objects
		wrap to the next row or column to avoid exceeding the managed container’s
		bounds; border layout in which objects can be specified to reside in the
		north, south, east, west, or center of the managed container; and grid
		layout in which objects are specified to reside at one position (or
		straddle multiple positions) in a programmer-specified 2D grid.
		Programmer-specified parameters indicate preferred, minimum, and maximum
		widths and heights of objects; and spacing, both between objects and
		between objects and the container’s edges. Managed containers can be nested
		inside of other managed containers, and treated just like any other
		objects.

		A programmer designs a parameterized layout as a hierarchy of managed
		containers, chosen for their layout policies, and further constrained by
		programmer-specified parameters. Thus, a layout manager does not actually
		design a layout, but rather instantiates a layout at run time from the
		structure and parameters specified by the programmer. Designing a simple
		layout (e.g., four buttons displayed in row-major order) is easy for a
		programmer, but designing a complex hierarchical layout, while possible, is
		tedious and difficult, especially if it needs to behave robustly when
		resized. The popularity of the layout-manager approach stems primarily from
		the ease of implementing the managers themselves and the relative ease with
		which they can be used by programmers to define simple parameterized
		layouts. In addition, because the final layout is determined at run time,
		the presentation can work well under a wide range of possible window sizes,
		with lower bounds on usability imposed by the minimum effective sizes of
		the objects being laid out.

		Commercial word-processing and presentation systems intended primarily for
		sequential presentations (e.g., Microsoft Word, Publisher, and PowerPoint,
		Quark Express, and LaTeX), provide a set of preauthored style “templates”
		(and the ability to create new ones) that can be applied to existing
		material; for example, by matching “markup” tags present in the material to
		be processed with corresponding tags in the template. In comparison with
		the parameterized layouts of UI toolkit layout managers, most of these
		template-based systems are simpler. They emphasize the format of the
		material to be presented, relying on the order in which objects are
		specified as input to directly determine the order in which formatted
		objects appear in the presentation, with the exception of their handling of
		floating objects. Floating objects, such as tables or figures, can move
		relative to surrounding objects (typically text). Typically, a set of rules
		is used to determine the final position of the floating object; for
		example, “If the height of the object is more than 60\% of the height of
		the page, then put the object on a separate page; otherwise, put the object
		at the nearest paragraph break.” However, users of these systems often move
		floating objects around by hand to guide or overrule simplistic placement
		policies.

		In the following sections, we discuss the two techniques that have been
		explored in automated layout systems: constraint satisfaction and learning.

    \section{Constraint satisfaction}
    \label{constraint-satisfaction}

		The vast majority of research in automated layout to date focuses on
		constraint-based methods (e.g.,
		\citep{vanderzanden-1,borning-1,graf-1,hudson-3,kochhar-1,hudson-2,weitzman-2,myers-2}).
		The idea that layouts can be represented as a set of constraints is very
		intuitive. For instance, consider the constraint relationships depicted in
		Figure 1. The goal of a constraint-based automated layout system is to take
		such a constraint network and generate a set of positions and sizes for
		each of the components in the network. Any constraint-based automated
		layout system can be characterized by the kinds of constraints it uses; how
		they are described, obtained, and resolved; and how the system addresses
		constraint inconsistencies, loops and other hazards that might prevent the
		solution from converging. In the rest of this section, we explore these
		issues.

    \subsection{Types of constraints}

		In a constraint-based automated layout system, most constraints can be
		classified as either abstract or spatial. By abstract, we mean that the
		constraint describes a high-level relationship between two components that
		are to be included in the layout (e.g., “TEXT1 REFERENCES PIC1”). In
		contrast, spatial constraints enforce position or size restrictions on the
		components (e.g., “CAPTION1 BELOW PIC1”). Spatial constraints can be fed
		directly into a constraint solver for resolution, whereas abstract
		constraints must be processed and reduced to spatial constraints before the
		layout can be realized. The process of generating spatial constraints from
		the abstract constraints is often the most challenging part of creating an
		automated layout system.

		The choice of employing abstract or spatial constraints depends on the
		nature of the layout that the system is designed to generate. Many research
		systems employ both abstract and spatial constraints because they are
		general multimedia presentation tools \citep{feiner-1,weitzman-2,graf-1}.
		Other research initiatives are embodied in the form of libraries such as
		subArctic \citep{hudson-3,hudson-1} and the Garnet toolkit \citep{myers-2}
		that include extensions of “layout manager” type functionality with simple
		spatial constraints. Although automated layout systems for more limited
		environments can create effective layouts using only spatial constraints
		\citep{kosak-1}, the same can not be said for attempting to employ abstract
		constraints without spatial constraints. Figure 2 depicts what the
		difference in output might be between a system that considers only abstract
		constraints versus a system that takes into account both abstract and and
		spatial constraints. This issue is discussed further in Section
		\ref{spatialcon}.

    \subsubsection{Abstract constraints}

		Abstract constraints are descriptions of high-level relationships between
		the various components that are to be placed into the layout. Abstract
		constraints such as “TEXT1 REFERENCES PIC1” and “TITLE IS IMPORTANT” are
		sufficiently high-level that content authors can easily specify them and
		are particularly effective for use in interactive systems because the
		author of the content needs no additional technical or artistic skill to
		specify them.

		Although one might think that “TEXT1 REFERENCES PIC1” implies that B and C
		are relatively close to each other in the generated layout, abstract
		constraints in and of themselves do not specify the position and size of
		the components of a layout. This is because the mapping between abstract
		constraints and spatial constraints is performed by a translation component
		before spatial constraints are passed to the numeric constraint solver. The
		abstract– spatial constraint translator can choose to map the abstract
		constraint “TEXT1 REFERENCES PIC1” into any set of spatial constraints.
		This might mean that the PIC1 is placed right next to text TEXT1, or that
		PIC1 is placed on a different page and some visual cue is left to guide the
		end user to it from TEXT1.

    \subsubsection{Spatial constraints}
		\label{spatialcon}

		Spatial constraints are relations that directly express the geometric
		structure of the presentation. For instance, we may wish to force a certain
		block of text to appear beneath another block of text that the user is
		assumed to have read first. Another instance of spatial constraints would
		be to force all objects to occupy a space that is of a certain size or an
		integral multiple of that size.

		There are a number of reasons why we would want to impose spatial
		constraints. Perhaps the most prevalent reason is to improve upon the
		visual quality and aesthetics of the presentation. Many early automated
		layout systems are created from the perspective of computer science and
		mathematics alone \citep{beach-1}. Such systems tend to treat the problem
		as a purely theoretical question of tiling and use optimization techniques
		to find a solution \citep{luders-1}. These kinds of systems will often not
		take into account simple legibility rules (e.g., text should be placed into
		columns that run down the entire page rather than having blocks of random
		size packed onto the screen) and style guidelines (e.g., all captions go
		beneath their associated figures and spacing between a figure and
		surrounding text block should be the same everywhere). Figure 2 exemplifies
		what might happen in a system that employs abstract constraints without
		spatial constraints. A system that considers only abstract constraints will
		not be able to generate layouts with the same aesthetic appeal as systems
		that consider both because the system has no visual restrictions on where
		components of the layout are placed.

		The method by which the components are represented may place some kind of
		limitation on what kinds of spatial constraints may be used. One such issue
		that may arise is sometimes referred to as the “Cousins Problem,” an
		example of which is shown in Figure 3. This problem may arise if the data
		structure in which the components are being stored does not allow
		referencing one component’s children from a different component’s child.

		It seems intuitive that we would want to use concepts from graphic design
		to create legible and pleasing output. Some systems enforce the
		presentation to conform to a grid system, similar to those used to lay out
		newspapers \citep{muller-1,hurlburt-1}. In a grid system, every screen or
		page of the presentation is divided into an array of upright rectangles.
		Each object must occupy one or more complete rectangles. Figure 4 is an
		example of output from Feiner’s GRIDS system \citep{feiner-1} which designs
		layout grids that enforce a consistency between screens or pages of a
		presentation. One complication with employing grid systems is that a
		graphical component may need to be cropped, padded with a border or have
		its aspect ratio changed This is because uniform scaling of the object may
		not be sufficient to make the object occupy an integral number of grid
		rectangles.

		Automated layout systems for well-defined environments, such as network
		diagrams or label placement, often employ spatial constraints exclusively
		\citep{kosak-1,christensen-1}. These systems consider issues such as the
		proximity of the components being placed, the distance between a label and
		its target, and the possibility of confusing the end user by placing
		multiple labels that are sufficiently near the same target that the end
		user doesn’t know which label is associated with it. Abstract constraints
		are often used for formatting labels (larger cities have bigger names), but
		are generally not used for layout directly.

		Some systems allow the user to specify abstract data constraints separately
		from spatial constraints. This allows for a logical single presentation to
		have many different “skins,” opening the door for a single presentation to
		be displayed using different media \citep{weitzman-2}. This is particularly
		effective for interactive layout systems that might want to maintain a
		separation between content authors and “layout experts.” A similar approach
		that leverages this concept is to build the spatial constraints into the
		system, thereby eliminating the need for human intervention to specify
		spatial constraints for each layout to be generated. Feiner’s GRIDS is an
		example of a such system \citep{feiner-1}.

		Spatial constraints are sometimes used to increase the efficiency of the
		constraint solver. For instance, a constraint might be placed on all
		objects of a certain type that permits them to be resized in only one
		direction \citep{linton-1}. Similarly, imposing the constraint of a
		quantized display permits the use of fast fixed point and integer
		programming techniques when resolving the set of constraints.

    \subsection{Expressing constraints}

		Intuitively it would seem best to define a formal grammar to describe the
		method by which the constraints are expressed. This approach benefits from
		being able to leverage a rich body of existing research for manipulating
		and parsing constraints. A rich grammar might be very flexible and
		expressive and translate into better layouts \citep{weitzman-1}. An example
		of a grammar for use in a layout system is shown in Figure 6. However,
		powerful and expressive grammars may also be difficult to use. This is
		especially true of grammars or ontologies that attempt to be extremely
		general and all-encompassing. In addition, a very complex solver may be
		necessary to process the information present in such a system. As one might
		imagine, it is extremely difficult to create a system that describes the
		set of all possible high-level relationships between components of a
		presentation, although this has been explored for the use of automatic
		graphics generation \citep{zhou-1}.

		Another extremely powerful approach is to express the constraints in terms
		of Boolean predicates \citep{graf-1}. This approach alleviates some of the
		concerns that arise from the more expressive grammar and relational grammar
		approaches by limiting the space of what can be expressed. The use of
		Boolean predicates also eases the process of solving the set of constraints
		as the input needs little or no translation before being passed to the
		solver.

		\subsection{Obtaining constraints}

		One of the most important practical issues in implementing a
		constraint-based automated layout system is determining where to obtain the
		constraints. Approaches that have been tried range from fully automated to
		the computer making suggestions.

		Many automated layout systems implement abstract constraints by gathering
		them from structured input data
		\citep{mackinlay-1,casner-1,borning-1,beach-1}. These systems take tables
		of numeric data and automatically create presentations. The structure of
		the data provides all the relationships that are needed to generate the
		layout. Other systems that are designed for multimedia layout have
		languages to explicitly specify the abstract constraints to describe
		relationships such as “author-of,” “description-of” and “precedes” between
		the components \citep{weitzman-2,graf-1}. The assumption that input data is
		readily available in a structured form is becoming increasingly valid
		because the information that we create is beginning to be stored in more
		structured formats \citep{bray-1}. Such work is prevalent in the layout
		modules of automated graphics generation systems \citep{zhou-2}.

    \subsubsection{Interactive specification}

		Interactive constraint specification systems are also extremely popular,
		but suffer from the obvious limitation that they require user input. Some
		systems are designed to help graphically naive content authors create
		professional quality layouts. Others are meant to reduce the amount of time
		a graphic designer needs to spend on a layout.

		Most systems that take interactive input do so for spatial constraints
		\citep{singh-2,hudson-2,borning-1}. This is because it is easy to create
		graphical user interfaces that allow the user to interactively place or
		adjust components on the screen. Although providing a graphical user
		interface to specify abstract constraints is not unheard of, abstract
		constraints tend not to need adjustment. Roth’s SAGE system \citep{roth-1}
		allows a user to associate database records with visual elements.

		Some of the interactive systems require the user to specify the high-level
		design of the document and then automatically generate the final result
		\citep{kim-1}. This approach is very useful in situations where the goal is
		to enable a content author to create layouts without the need for
		intervention by a “layout expert.” Some other systems take the opposite
		approach where the system produces the initial layout and allows the user
		to refine it \citep{singh-1}. These systems are more applicable in
		situations where the goal is to save the amount of time that a graphic
		designer needs to spend to create the layout.

    \subsubsection{Automated extraction}

		Fully automated constraint extraction is the least explored method of
		obtaining constraints. Some work has been done in integrating natural
		language techniques with automated layout \citep{roth-2}. This is
		particularly effective if natural language generation is being employed to
		create the content. Since the content generators are computer programs, it
		is much easier to have the generator send abstract that describes
		relationships between components as well as markup the text with flags
		denoting which parts are particularly important.

		Another method that has been explored is to extract abstract constraints
		from the entity relationships found in SQL databases \citep{pizano-1}.
		Unlike the natural language generation system that passes additional
		information to the layout system, in this approach abstract constraints are
		derived from data structures that were originally intended for use
		elsewhere.

    \subsection{Constraint solvers}

		A constraint-based automatic layout system must have some way to resolve
		the constraints with which it is presented. Formally, automated layout
		techniques all solve a form of the constraint satisfaction problem (CSP)
		\citep{mackworth-2,mackworth-1}. Both randomized and deterministic
		algorithms have been applied to solve the problems in this field. In
		general, the constraint solvers employed can be categorized as applying
		either a local or a global methodology.

    \subsubsection{Local techniques}

		Local constraint solvers attack the constraint satisfaction problem
		bottom-up. This might be compared to inductive reasoning, where a small
		subset of the universe is first solved. Two routes can be taken to solv the
		rest of the constraints and create the final presentation. The first
		approach is to solve many small subsets of the constraints independently
		and then run a second resolution phase to combine the results. An
		alternative approach is to iteratively resolve constraints at the border
		between the constraints that have already been solved and those that have
		yet to be considered \citep{nilsson-1}.

		Using a local-resolution technique can be problematic if the solver
		encounters local minima \citep{borning-2}. By resolving small subsets
		first, the solver may make decisions that bring the system to a suboptimal
		final solution. The advantage, however, is that local-resolution techniques
		usually execute much faster than global techniques.

    \subsubsection{Global techniques}

		Global constraint solvers attack the constraint satisfaction problem from
		the top down. Unlike local techniques, global techniques generally do not
		suffer from the problem of local minima, but require more computation time.
		To address the issue of additional computation time, numeric solvers that
		use iterative approximation techniques have been applied
		\citep{kurlander-1}. Randomized computation techniques (e.g., genetic
		algorithms and simulated annealing) have also been applied for label
		placement \citep{christensen-2}.

    \subsection{Inconsistency policies}

		If the set of constraints is sufficiently large, there is a strong
		likelihood that there will be some problems. In particular, some
		constraints may contradict others and possibly make the system of
		constraints unsolvable. A resolution policy must be specified to generate a
		layout in these cases.

		Some systems take a very simple approach to inconsistency by avoiding it.
		Rather than bogging the system down with inconsistency handling, the
		grammar used to articulate the constraints is designed in such a way that
		cycles cannot occur \citep{weitzman-2}.

		Another popular method for handling inconsistency is to apply priorities to
		the constraints \citep{graf-1}. By permitting each constraint to have an
		inherent priority, the system can make intelligent decisions about which
		constraints to drop should a contradiction be encountered. A problem can
		still occur here if two conflicting constraints have the same priority. In
		this case, the system can use the AI technique of applying a tie-breaking
		strategy (e.g., first-come first-served or pick one randomly) so that the
		layout can be generated \citep{nilsson-1}. Priorities are critical for
		generating effective layouts in systems where there are complex networks of
		both abstract and spatial constraints. For example, the enforcement of the
		grid in a system that employs design grids must take precedence over all
		other constraints.

    \section{Learning techniques}
    \label{learning-techniques}

		Machine learning has been applied to many automated multimedia authoring
		systems, including speech synthesis \citep{pan-1} and natural language
		generation \citep{kamimura-1}, as well as to graph layout \citep{masui-1}.
		However, most automated layout systems do not leverage the large body of
		existing work by the AI community in machine learning.

		Automated layout systems that do have some form of learning tend to use it
		during the interactive specification of constraints
		\citep{myers-1,borning-1}. These systems do not implement full machine
		learning systems. Rather, they try to “learn” based on interacting with the
		user. The Marquise system \citep{myers-1} allows a user to set the system
		into a training mode where the relative locations of components are
		demonstrated to the system. Spatial constraints that will be used to
		generate the layout are then extracted from this interaction with the user.
		If the constraints cannot be extracted, the system falls back to having the
		user specify the position explicitly as a LISP function. Borning’s
		\citep{borning-1} ThingLab is similar in that it allows for demonstration
		of constraints, but also adds the ability to demonstrate animation.

    Some recent work in automated graphics generation has also explored the use of learning techniques \citep{zhou-3}. Zhou divides the space of rules that need to be acquired for presentation generation into three categories: information learning space, visual learning space and rule learning space. Visual learning space is directly related to spatial constraints, and thus is similar to Myer’s and Borning’s work. Unlike Marquise and ThingLab however, Zhou’s system employs full-strength machine learning that can be fully automated by providing the system with a large dataset of presentations designed by a “layout expert” in addition to the interactive methods seen in other work that employs learning techniques.

    \section{Evaluation techniques}
    \label{evaluation-techniques}

		Whenever a piece of software is used to perform a task that is
		traditionally believed to be reserved for human “experts,” the question
		that will always be asked is whether or not the computer is as “good” as
		the human. In the field of layout, “good” may refer to the usability (e.g.,
		whether tasks can be accomplished more quickly, with fewer errors, with
		greater user satisfaction), as well as the aesthetics of the presentation
		(e.g., whether end users or graphic designers think that the results look
		as good as ones produced by humans).

		In some sense, all interactive layout generation systems have a module that
		handles the evaluation of how “good” the layout is: the human user. By
		using a computer-based evaluation mechanism, we could evaluate the layout
		automatically without relying (as much) on the user, ideally feeding back
		the results to redesign a layout that is not deemed good enough.

		Evaluation of user interfaces, independent of who or what designs them, has
		been explored by a number of researchers
		\citep{tullis-1,sears-1,sears-3,comver-1,jeffries-1}. The focus of this
		research has primarily been divided between creating heuristic inferences
		and quantitative metrics. Heuristics are of course more flexible, whereas
		metrics are easier to incorporate into computer systems and hence more
		common in automated evaluation mechanisms.

		Sears has explored the application of metrics to automated layout systems
		\citep{sears-2}. He employed metrics to evaluate how usable an interface is
		based on the amount of mouse movement that is needed between button clicks.
		Figure 7 shows screen shots from Sears’s system. The additional information
		provided by these metrics are embodied in the form of spatial constraints.
		Fitts’ Law \citep{mackenzie-1} implies that buttons that are often clicked
		sequentially should be placed close to each other spatially, since this
		reduces the time needed to move between them.

		Other research has addressed adding evaluation to the automated layout
		process in the context of user modeling \citep{schlungbaum-1}. This work
		proposes that the interface not only include or exclude certain elements
		based on the type of user, but that it also change the layout. This
		information may be gathered at run time—as the frequency of use of
		different parts of the system change, the layout could be modified to
		reflect this. An appropriate user model could make it possible to adapt the
		output for the user. Note, however, the potential for change to create a
		less, rather than more, effective UI by clashing with the user’s mental map
		of the user interface.

    \section{Conclusions and future work}
    \label{conclusions-layout-paper}

		We have illustrated the range of research that has been accomplished in the
		field of automated layout, from simple techniques to research systems. As
		data presentation needs rapidly increase, the field of automated layout
		will become increasingly important. In time, we feel it is inevitable that
		the more powerful techniques found in the research systems will make their
		way into popular software packages. Reviewing current research, we see a
		number of rich possibilities for future work.

		Integration of natural language techniques with automated layout systems
		has been explored to some extent, particularly with natural language
		generation. However, similar work has not been pursued with image and video
		understanding or speech recognition. It may be possible to extract abstract
		or spatial constraints for automated layout by applying well known vision
		or speech recognition techniques to multimedia components of a layout.

		Another interesting possibility is considering how to handle constraints
		that are “wrong.” Most systems have a user specifying the constraints in
		some manner at some point in the layout pipeline. The problem is that the
		user might simply make a mistake and not really mean what what he or she
		specified. In a system that has support for ranking constraints by
		priority, the user might also assign incorrect priorities. Constraints that
		are automatically extracted opens up even more doors for feeding incorrect
		information to the system. The use of natural language understanding, image
		understanding or speech recognition to extract constraints by definition
		means that there will be some probability of error in the constraints.
		Enabling an automated layout system that would be capable of handling these
		kinds of problems might involve applying AI techniques from adversarial
		game playing \citep{nilsson-1}. Algorithms from computational biology and
		genomics may also be applicable because this kind of problem is encountered
		during gene sequencing \citep{salzberg-1}. Constraints extracted by an
		automated process can be verified by running multiple extraction algorithms
		and having them “vote.” A similar approach is employed by an
		object-recognition technique called the Hough transform \citep{horn-1}. By
		applying algorithms to defend against constraint error, a system might be
		made robust enough that errors in the constraints could cause little or no
		loss of effectiveness in the presentation.

		There are other kinds of constraints that might be worthwhile to consider,
		some of which be obtained through hardware capture of information for user
		models. For example, real-time systems that automatically generate a user
		interface would benefit from knowing the user’s distance from the display
		medium (information that was taken into account at the beginning of the
		layout design process of \citep{feiner-1}, but not computed automatically).
		This could be determined from any of a variety of head-tracking
		technologies. Eye tracking could also be used to extract additional
		constraints based on where the user is looking. What parts of the display
		the user has actually seen would be useful information to pass to the
		automated layout system.

