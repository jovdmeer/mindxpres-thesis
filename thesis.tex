% vim:ts=1:et:spelllang=en

\documentclass[a4paper,12pt,nocenter]{thesis}
%\documentclass[a4paper,12pt]{report}
%\documentclass[a4paper,12pt]{book}

% The following makes latex use nicer postscript fonts.
%\usepackage{times}
\usepackage[english]{babel}
\usepackage{subcaption}
\usepackage[usenames]{color}
\usepackage{multicol}
%\usepackage[colorlinks,urlcolor=blue,linkcolor=blue]{hyperref}
\usepackage[%ps2pdf,
            bookmarks=true,
            bookmarksnumbered=false,
            bookmarksopen=false,
            colorlinks=false,
%            colorlinks=true,
            linkcolor=webred]{hyperref}
\definecolor{webgreen}{rgb}{0, 0.5, 0} % less intense green
\definecolor{webblue}{rgb}{0, 0, 0.5} % less intense blue
\definecolor{webred}{rgb}{0.5, 0, 0} % less intense red
\usepackage[round,comma,authoryear]{natbib}

%\hyphenation{administrative argument arguments assignments complex evaluates functions happening however machine understand unreliable variable variables whenever}

\usepackage{vubtitlepage}
\author{Joris Vandermeersch}
\title{Content Migration and Layout for the \mxp Presentation Tool}

%\promotortitle{Promotor/Promotors}
\promotor{Prof. Dr. Beat Signer}
\advisors{Reinout Roels}
\advisortitle{Begeleider}
\faculty{Faculteit Wetenschappen}
\department{Departement Informatica\\
            en Toegepaste Informatica}
\reason{Proefschrift ingediend met het oog op het behalen\\
        van de graad van Master in de Toegepaste Informatica}
\date{Augustus 2015}

% vim:ts=1:et:nospell:spelllang=en_gb:ft=tex

\usepackage{xspace}
\usepackage{xparse}

\usepackage{listings}
\usepackage{color}

\newcommand\code{\lstinline[basicstyle=\ttfamily\selectfont, breaklines=true, breakatwhitespace=true]}
\NewDocumentCommand\ppt{s}{\IfBooleanTF#1{{Microsoft PowerPoint}\xspace}{{PowerPoint}\xspace}}
\newcommand\mxp{{MindXpres}\xspace}
\newcommand\latex{\LaTeX\xspace}
\newcommand\lstref[1]{Listing \ref{lst:#1}}
\newcommand\figref[1]{Figure \ref{fig:#1}}
\newcommand\fig[2]{
 \begin{figure}[h!]
  \centering
  \includegraphics[width=0.85\textwidth]{img/#1.png}
  \caption{#2}
  \label{fig:#1}
 \end{figure}
}

%\lstloadlanguages{C}

\definecolor{maroon}{rgb}{0.6,0,0}
\definecolor{darkgreen}{rgb}{0,0.65,0.1}
\definecolor{darkblue}{rgb}{0,0,0.7}
\definecolor{grey}{rgb}{0.90,0.90,0.90}
\definecolor{pblue}{rgb}{0.13,0.13,1}
\definecolor{pgreen}{rgb}{0,0.5,0}
\definecolor{pred}{rgb}{0.9,0,0}
\definecolor{pgrey}{rgb}{0.46,0.45,0.48}
\lstdefinelanguage{XML} {
  morestring=[s]{"}{"},
  morecomment=[s]{<?}{?>},
  morecomment=[s]{<!--}{-->},
  commentstyle=\color{blue},
  moredelim=[s][\color{maroon}]{\ }{=},
  moredelim=[s][\color{black}]{>}{<},
  moredelim=[s][\color{black}]{\ />}{<},
  stringstyle=\color{darkgreen},
  identifierstyle=\color{darkblue}
}

\lstnewenvironment{lstjava}[2]
{
  \csname lst@SetFirstLabel\endcsname
  \lstset{language=Java,
  showspaces=false,
  showtabs=false,
  breaklines=true,
  showstringspaces=false,
  breakatwhitespace=true,
  commentstyle=\color{pgreen},
  keywordstyle=\color{pblue},
  stringstyle=\color{pred},
  moredelim=[il][\textcolor{pgrey}]{$$},
  moredelim=[is][\textcolor{pgrey}]{\%\%}{\%\%}
  }
}
{
  \csname lst@SaveFirstLabel\endcsname
  \centering
  \renewcommand{\figurename}{Listing}
  \renewcommand{\figureshortname}{Lst.}
  \caption{#2}
  \label{lst:#1}
}

\lstnewenvironment{lstxml}[2]
{
  \csname lst@SetFirstLabel\endcsname
  \lstset{language=XML}
}
{
  \csname lst@SaveFirstLabel\endcsname
  \centering
  \renewcommand{\figurename}{Listing}
  \renewcommand{\figureshortname}{Lst.}
  \caption{#2}
  \label{lst:#1}
}

\lstnewenvironment{lstxml*}
{
 \csname lst@SetFirstLabel\endcsname
 \lstset{language=XML,
         xrightmargin=0pt
         }
}
{
 \csname lst@SaveFirstLabel\endcsname
}

\lstnewenvironment{lstxmlnoref}[0]
{
  \csname lst@SetFirstLabel\endcsname
  \lstset{language=XML}
}
{
  \csname lst@SaveFirstLabel\endcsname
}

\newcommand\fignl[1]{
 \centering
 \includegraphics[width=\textwidth]{img/#1.png}
}

\newcommand\cl[2]{
 \centering
 \caption{#2}
 \label{lst:#1}
}

\lstset{
         basicstyle=\ttfamily\fontsize{7pt}{8pt}\selectfont,
         numbers=left,                   % where to put the line-numbers
         numberstyle=\fontsize{7pt}{8pt}\selectfont,      % the size of the fonts that are used for the line-numbers
         stepnumber=1,                   % the step between two line-numbers. If it is 1 each line will be numbered
         numbersep=1em,                  % how far the line-numbers are from the code
         backgroundcolor=\color{grey},   % choose the background color. You must add \usepackage{color}
         showspaces=false,               % show spaces adding particular underscores
         showstringspaces=false,         % underline spaces within strings
         showtabs=false,                 % show tabs within strings adding particular underscores
         frame=single,           % adds a frame around the code
         frameround=tttt,
         flexiblecolumns=false,
         basewidth={0.5em,0.45em},
         linewidth=\textwidth,
         xleftmargin=2em,
         xrightmargin=1em
         }


\begin{document}

% First dutch TitlePage
 \maketitlepage

 \faculty{Faculty of Science}
 \advisortitle{Advisor}
 \department{Department of Computer Science\\
             and Applied Computer Science}
 \reason{Graduation thesis submitted in partial fulfillment of the\\
         requirements for the degree of Master in Applied Computer Science}

 \date{August 2015}

 \setcounter{page}{1}
% Then english TitlePage
 \maketitlepage

 \include{abstract}

 % vim:ts=1:et:nospell:spelllang=en_gb:ft=tex

 \chapter*{Acknowledgements}

  \emph{``Simplicity is a great virtue,\\
  but it requires hard work to achieve it and education to appreciate it.\\
  And to make matters worse: complexity sells better.''}

  \hfill\emph{--- Edsger W. Dijkstra}

%  I'd like to thank:
%  \begin{itemize}
%   \item Tania, my fianc\'ee, \emph{for all of the patience and sex she continues to have with me.}
%   \item Peter, my brother, \emph{for helping me realize life should be fun.}
%   \item Reinout, my advisor and friend, \emph{for giving me the opportunity to get this over with.}
%   \item my friends \emph{for frequently providing the necessary distraction from this awful ordeal.}
%   \item my grandparents \emph{for the moral and financial support which gave me the opportunity to start and finish this, even though I'd have liked to give up ages ago.}
%   \item my parents \emph{for the genes, the upbringing and the moral support, even if they generally have no clue what I'm talking about.}
%   \item my employers and colleagues at Roots Software \emph{for putting up with my quirks, my seemingly never-ending studies and all of the inconveniences it has brought along through the years, as well as for the steady paycheck that allows me to have a life in which the academic world has nor needs a place.}
%   \item all professors, assistants and other academic personnel at the VUB \emph{for making me realize I never want to be a part of their world. Seriously.}
%  \end{itemize}



 \tableofcontents

% TODO
%14:52 <reinout> thesis is wel een ok begin, but it all needs a lot of padding :p
%14:52 <reinout> bijvoorbeeld, you should spend ~5pg on mindxpres
%14:52 <reinout> waarom bestaat het, hoe werk dat plug-in gedoe, hoe werkt dat xml gedoe
%14:55 <reinout> basically alles wat in deze paper staat:
%14:55 <reinout> https://www.academia.edu/4186970/An_Extensible_Presentation_Tool_for_Flexible_Human-Information_Interaction
%14:55 <reinout> nee wacht
%14:55 <reinout> deze:
%14:56 <reinout> https://www.academia.edu/7719770/MindXpres_An_Extensible_Content-driven_Cross-Media_Presentation_Platform
%14:56 <reinout> explain ALL the MindXpres
%14:57 <reinout> en dan kan je verdergaan, "mensen hebben nu hun content, maar nu willen we dat ze die content in MindXpres kunnen gebruiken"

 % vim:ts=1:et:nospell:spelllang=en_gb:ft=tex

 \chapter{Introduction}

  For over 25 years, \ppt* has been the market leader in digital prsentations.
  Admittedly, it was a revolutionary software package when it was first
  introduced, and its ease-of-use combined with its supreme graphical
  capabilities --- at least compared to other software in the same era --
  quickly made it one of the most popular software packages in history. 25
  years later, \ppt* can claim over 90\% market share in presentation software,
  and on average 30 million \ppt presentations are created every day.

  In this time, \ppt* has gotten many new features, and certainly improved and
  grew with every new version, but it never really changed its core approach.
  It started out mimicking the then-popular and widespread use of dia and
  overhead projection slides, which was at the time a good way to convince
  people of its purpose, allowing them to feel comfortable with a familiar
  format instead of alienating potential customers with a new and potentially
  confusing interface.

  However, this interface is quite restricting, and in recent years different
  approaches have seen the light of day. The zoomable user interface of Prezi
  is probably the most well-known, but apart from abandoning the traditional
  slide format it does little to improve or extend the concept of presenting
  information to an audience.

  This is where \mxp comes in. Its extensible plug-in system allows anyone with
  some knowledge of programming to create new functionality to use in
  presentations. Examples are interactivity with the audiencer through various
  means, controlling the presentation from another device --- or several! ---
  and (re)modelling data while presenting it, based on feedback from the
  audience.

  While this is obviously a big improvement on the traditional presentation
  model of \ppt* and the likes, it remains hard to convince the general public
  of its merits. People are generally afraid of change, and it is important to
  make the transition as smooth as possible. On top of that, people are often
  worried that the work they did in the past may be lost --- or worse,
  irrelevant --- after switching to something new. This alone may be a huge
  factor in deciding wether or not to start using new software, or to stick
  with what they know.

  That is where the subject of this thesis comes in. We aim to provide a way
  for people to convert their existing \ppt presentations into \mxp
  presentations, allowing them to take their previous work with them in their
  switch to \mxp. This way, we lower the treshold for them to make the decision
  to start using \mxp as their presentation software of choice. Once all their
  existing \ppt content is available, usable and editable in \mxp, it should be
  obvious to anyone why \mxp is the better option for their presentations.

  Another common problem with \ppt presentations is the way they look. This is
  not necessarily the fault of the software; most people just are not trained
  in graphical design, and as such they know very little about proper layout,
  color choices, or slide content limits. Everyone has probably encountered
  slides with full paragraphs of text, too small to read and / or too much to
  process in the short time the slide is visible --- (too) many people have
  made those slides themselves.

  When we say this is not the fault of the software, that is mostly true, as
  the creators of these slides obviously made a conscious choice to make their
  content appear like that. It could be said however that \ppt* and other
  presentation tools are guilty through inaction. We believe it is possible to
  have software either warn its users against these choices and practices, or
  --- even better --- have the software fix these problems automatically.

  One of the primary purposes of \mxp is to provide automated layout, much like
  \latex does, ensuring that the content creator only has to worry about the
  actual content, while the software takes care of layout. In practice, both
  \latex and \mxp currently use template-based layouts, where the contents'
  position is predefined in the template and not related to or based on its
  size, shape or nature. In the end, everyone who has ever used \latex knows
  that sooner or later you will struggle to get a certain image incorporated in
  the text correctly, ending up doing the layout yourself anyway, because the
  predefined template just doesn't work properly for your specific content.

  Our goal is to eradicate those situations. Automated layout should
  dynamically adjust to any content it is given, no matter the size or aspect
  ratio. This may seem hard, if you consider the limits of slides and the fact
  that you can only fit so much content on them before they are full. This is
  where another important aspect of \mxp comes into play: we are not
  necessarily bound to the limits of slides. If we don't have to consider the
  boundaries of traditional slides, we can fit content together in an
  aesthetically pleasing way much easier, without having to scale anything.

  As such, the second part of this thesis focuses on implementing true
  automated layout in \mxp. Again with the goal to convince \ppt* users to
  switch, showing that their presentations actually could look better in \mxp,
  but at the same time we also provide new functionality to other \mxp users.

  We believe this functionality will improve the aesthetic aspect as well as
  the effectiveness of presentations. If content is not scaled down to fit the
  articifial confines of a slide, but can instead be shown and studied in
  detail, this should clearly increase the flow of information towards the
  audience. Providing an overview of the information in a presentation becomes
  easier and more effective too: where traditional presentations rely on a
  boring table of contents, in which the presenter announces the subjects
  they'll be talking about one by one while the audience forgets the first
  thing in the list by the time they get to the last, \mxp allows the presenter
  to just show all of the presentation's content at once just by zooming out.
  Here automated layout can help as well: content can be arranged in such a way
  that an overview effectively hilights the important subjects, different parts
  or keywords of a presentation.

  Last but not least, we hope this functionality improves the experience of
  creating a presentation. Everyone who ever created a presentation knows, and
  research has shown \citep{lok-1} that often more time is spent on creating
  and fine-tuning the layout than actually putting in the content. Most
  presenters however have not had any significant training in creating
  effective layout, which means this time is often wasted on a layout that ends
  up not actually benefiting the presentation as a whole. We want to eliminate
  this problem by taking control over the layout away from the presenter and
  instead providing them with a programatically generated layout that presents
  the information provided by the presenter in the best, clearest way possible.


 % vim:ts=1:et:nospell:spelllang=en_gb:ft=tex

 \chapter{Related work}

  \emph{This chapter's content is largely based on ``MindXpres: An Extensible
  Content-driven Cross-Media Presentation Platform'' \citep{roels-1}.}

  \section{Background}

   The importance of digital presentations in this day and age cannot be
   understated. Millions of presentations are created every day, supporting the
   oral transfer of knowledge and playing an important role in educational
   settings. Their origins as tools for creating physical media such as
   photographic slides or transparencies for overhead projectors are still
   reflected in the underlying concepts and principles of slide-based
   presentation tools. The rectangular boundaries of a slide, and the linear
   navigation between slides, are still restrictions we face today in digital
   presentations. Tufte argues that these concepts of slideware have a negative
   impact on the effectiveness of knowledge transfer \citep{tufte-1}.  While
   the presenter is compelled to squeeze complex ideas into a linear sequence
   of slides, those ideas are rarely sequential by nature, resulting in a loss
   of relations, overview and details. An initial approach to address these
   issues might involve creating minimalistic presentations or introducing some
   structure via a table of contents. Sadly, when complex knowledge or other
   pieces of rich information need to be presented “as is” \citep{farkas-1} ---
   as in the domain of learning --- this does not work.

   One of the main issues with traditional slideware presentations is their
   monolithic nature, especially when content is spread over many
   self-contained presentation files. ``Reusing'' previous work involves either
   switching between files while giving a presentation or duplicating some
   slides in the new presentation. It should be noted that this issue is not
   limited to the reuse of single slides: there is an ever increasing wealth of
   resources available for reuse, spread over a wide spectrum of distribution
   channels and formats. The possibility to include content by reference or
   transclusion \citep{nelson-1} may contribute in crossing the boundaries
   between different types of media and prove beneficial in the context of
   modern cross-media presentation tools.

   The difference in functionality between the authoring of content and its
   visualisation is striking as well. The primary editors consist of mostly
   toolbars and buttons used for selecting and specifying the way content
   should be visualised, while support for authoring the content itself is not
   quite as extensive. Modern slideware has grown to include basic multimedia
   types such as videos, but most content is still rather static. It is, for
   example, not possible during a presentation to easily switch from a bar
   chart to a pie chart data visualisation, or to dynamically change some
   values in the represented data and immediately see the effect in the graph,
   which could be beneficial for knowledge transfer \citep{holzinger-1}. The
   audience could also be more actively involved in the presentation, through
   audience response and classroom connectivity systems providing multi-device
   interfaces allowing to share knowledge and results during as well as after a
   presentation. The evolution of presentations is reminiscent of the Web2.0
   movements where users have switched roles from purely consuming content to
   contributing as well, content has become more dynamic and interactive, and
   service-oriented architectures (``The Cloud'') have ensured decentralisation
   of content.

   In order to move a step towards the next generation of cross-media
   presentation tools, it is essential to allow the rapid prototyping and
   evaluation of new concepts for the representation, visualisation and
   interaction with content.
   
   Before discussing the requirements for a new generation of presentation
   tools, we briefly introduce existing slideware solutions. Afterwards, we
   describe the architecture of \mxp, its extensible nature and its plug-in
   mechanism. The HTML5-based implementation of \mxp is then discussed through
   demonstration of several use cases and \mxp plug-ins.

   A specific issue with slideware we'd like to focus on in this thesis, is the
   trouble with layout in presentations. It can be hard to display the content
   you want in a way that's clear, informative and nice to look at. The vast
   majority of layouts created today is mostly done by hand: a human graphic
   designer or ``layout expert'' makes most, if not all, of the decisions about
   the position and size of the objects to be presented \citep{lok-1}. Most
   software offers some templates, allowing you to drop pictures and text into
   predefined slots and places on a slide, but then those templates have been
   defined by someone else too. Computer-generated layout is rare and usually
   not quite up to the task.
  
   \mxp is among the software packages offering templates, in that layout is
   handled by whichever plug-in you choose, but so far no plug-ins have defined
   dynamic layout algorithms, rather sticking to predefined ways to put text
   and pictures on slides. But as \mxp does not constrain us to the limits of
   slides, this should be seen as an opportunity to offer dynamic layout as
   well. After all, if we're not limited to a certain area within which our
   content should fit, it should be much easier to put content next to each
   other in a way that makes sense.

  \section{Existing solutions}

   Since digital slideware was first introduced, their influence, advantages
   and disadvantages have been studied extensively. There have been studies
   acknowledging the benefits of slideware as a teaching asset
   \citep{holzinger-1}, while others have been less positive. Tufte
   \citeyearpar{tufte-1} heavily criticises slideware for its infatuation with
   outdated concepts. He discusses the many ramifications of dimensional and
   structural limitations as well as linear navigation, and points out the
   discrepancy with how the human mind works. Amongst Tufte's conclusions, and
   also confirmed by Adams \citep{adams-1}, is the suggestion that slide-based
   presentations are not appropriate for every kind of knowledge transfer and
   especially not in a scientific context. Recent work shows the importance
   towards the learning process of integrating content into the bigger picture,
   both structurally and visually \citep{gross-1}, which is affected by the
   navigation and visualisation.

   Several approaches have been proposed to offer non-linear navigation.
   CounterPoint \citep{good-1}, Fly \citep{lichtschlag-1} and Prezi, provide
   Zoomable User Interfaces (ZUIs) which offer virtually unlimited space.
   Microsoft has experimented with this concept as well in pptPlex. Other
   approaches to escape the confines of the slide have been noticed, like
   MultiPresenter \citep{lanir-1} or tiling slideshows \citep{chen-1}.
   PaperPoint \citep{signer-1} and Palette \citep{nelson-2} additionally
   facilitate the non-linear navigation of digital presentations consisting of
   slide selection through augmented paper-based interfaces. Lastly, a category
   of authoring tools exists which use hypermedia to implement varying paths
   through a set of slides. NextSlidePlease \citep{spicer-1} enables users to
   define a weighted graph of slides, and tries to suggest navigational paths
   based on the link weights and the remaining presentation time. Microsoft
   cultivates this idea in their HyperSlides \citep{edge-1} project. Garcia
   \citep{garcia-1} has additionally explored the potential of \ppt* as an
   authoring tool for hypermedia-based presentations.

   \ppt* was officially released in 1990, with Windows 3.0 \citep{austin-1}. It
   had originally been developed as Presenter, but trademark issues caused a
   name change early on. It was also originally build for the Macintosh, which
   may seem surprising nowadays but was actually common practice back then
   since the Macintosh was widely regarded as a better development environment,
   more mature, more stable and capable of far better performance and
   visualisations. Some may argue this still rings true today.

   Since then, it has grown to be the world's most popular slide show
   presentation program, alledgedly having been installed on over 1 billion
   computers worldwide, and being used on average 350 times \emph{per second}
   \citep{parks-1}. In 2012, it had a market share of 95\%, leaving the other
   5\% to be shared by alternatives such as Apple's Keynote, Prezi, SlideRocket
   and others. While this number is declining, it may not be going as fast as
   many people think. As most readers of this thesis have heard before, over 30
   million \ppt presentations are created every day, for all kinds of purposes,
   with good and bad results both presentation-wise and goal-wise.

   To reuse content in existing presentation tools, that content needs to be
   duplicated, which results in a multitude of redundant copies that need to be
   kept consistent with each other: if one copy is changed, all the others must
   be changed in the same way to prevent inconsistencies and mistakes. While
   some attempts have been made to solve this problem, there is still a long
   way to go. When looking for document formats designed to server more general
   educational purposes, we find formats such as the Learning Material Markup
   Language (LMML) \citep{suss-1}, the Connexions Markup Language (CNXML) and
   the eLesson Markup Language (eLML) \citep{fisler-1}.  All of these formats
   share their focus on the reuse of content, but all of them attempt this at a
   relatively high granularity level. Content can be organised in lessons or
   modules, and users are encouraged to use these, as a whole, in their
   teaching. When we investigated the formats more closely, we observed that
   outgoing links to external content were supported, but
   transclusion\footnote{The inclusion of content via references} was not.  In
   relation to presentations, Microsoft's Slide Libraries exist as central
   repositories that store slides to enable slide sharing and reuse within an
   organisation. The dependency on SharePoint might represent a hurdle for some
   users, as not everyone has the ability and opportunity to set up such a
   server. A more significant issue is the fact that slides still need to be
   searched and manually copied into presentations. Keeping slides in the
   repository and in other presentations is the responsibility of the authors
   of those slides and those presentations, as no automatic update system is
   provided. SlideRocket and SlideShare are both similar tools showing
   intentions and providing functionality for content reuse. The SliDL
   \citep{canos-1} research framework works much like Microsoft's Slide
   Libraries, in that it allows for storage and tagging of slides in a database
   for reuse, but also in that it shares the same shortcomings. The ALOCOM
   \citep{verbert-1} framework aimed at flexible content reuse is built upon a
   content ontology and a (de)composition framework for legacy documents
   including \ppt documents, Wikipedia pages and SCORM content packages.
   However, ALOCOM may be too rigid for evolving presentation formats, and it
   currently only supports the authoring phase, although the tool does succeed
   in decomposing legacy documents as advertised.

   Aside from the similarities in the Web's and presentation environments'
   evolution, some of the problems mentioned in this section can find their
   solutions in the context of the Web. It should not come as a surprise then,
   that web technologies are being used more often recently in the realisation
   of presentation solutions.  The Simple Standards-based Slide Show System
   (S5)\footnote{http://meyerweb.com/eric/tools/s5/} is an XHTML-based
   slideshow file format that enforces the standard slideware model. The W3C's
   Slidy \citep{raggett-1} initiative offers another presentation format based
   on the classical slideware model. Both of these formats have some valuable
   properties. They encourage a clean separation of content and visualisation
   through the use of CSS themes. The design is resolution independent, and the
   layout and font size adjust to the available screen real estate. Last but
   not least, some more recent HTML5-based presentation solutions such as
   impress.js, deck.js, Shower or reveal.js. Cross-device support is one of the
   most important advantages to leverage when using a well-known open standard
   such as HTML. However, as all of these solutions display some restrictions
   in terms of visualisation, navigation, and cross-media support, they are
   unfortunately too limited for our needs.

   The tools and projects discussed in this section mostly focus on
   distinguishing novel ideas for presentations. Nevertheless, the different
   concepts introduced in these tools don't offer interoperability between
   them. One project may focus on the authoring, another one fixates on novel
   content types and a third solution supplies radically new navigation
   mechanisms. Slideware tools may often allow third-party extensions but the
   API exposed to plug-in developers is usually limited by the software's
   underlying model. As an illustration, \ppt supports interaction from
   plug-ins with the presentation model, but the model dictates that a
   presentation consists of a sequence of slides. Many existing web-based
   presentation formats share this flaw. Because of this, we see a need for an
   open presentation platform such as \mxp to support innovation by
   contributing the necessary modularity and interoperability \citep{bush-1}.

   It is perhaps surprising that, to our knowledge, currently no tools exist to
   calculate dynamic layouts of content in slideware. Existing solutions
   include template systems, sometimes very fine-grained like \latex allowing
   you to define templates for every single layout choice, usually more coarse
   like \ppt* or Apple Keynote using Master Slides to define different layouts
   on a per-slide basis, and always with the option of letting the user
   customise the layout by hand, literally manually moving the content to the
   exact place where we want it, unhindered by style guides, good practices or
   common sense. This has resulted in mindboggling layout choices involving
   enormous amounts of tiny text crammed onto one slide, or pictures strewn
   across a slide overwhelming the audience with too much information at once. 

  \section{New solutions}

   Here we propose a set of requirements to establish a wide range of
   presentation styles and visualisations. This set has been compiled based on
   a review of the more recent presentation solutions discussed in the previous
   section.

     \paragraph{Non-linear Navigation} As we mentioned before, traversing
      slides in a linear fashion is a remnant of the way early photographic
      slides worked. Over the years, people have grown used to this form of
      navigation despite the inconveniences. If the presenter unexpectedly
      needs to show anything other than the next or previous slide (e.g. to
      answer a question from the audience), they either need a considerable
      amount of time to scroll forwards or backwards, or they have to switch to
      the slide sorter view, to find the desired slide. Also troubling is the
      lack of any functionality allowing a single slide to be included multiple
      times throughout the presentation without duplicating the slide in
      question, meaning if any change has to be made to that slide the same
      change has to be performed on all copies. This poses the risk of
      overlooking some copies, introducing inconsistencies and facilitating
      mistakes. There are several manners in which this lack of flexible
      navigation might be addressed, including the possibility to define
      non-linear navigation paths \citep{spicer-1,edge-1} or zoomable user
      interfaces (ZUIs) \citep{good-1,lichtschlag-1,haller-1}.
     
   %<TODO rework>

     \paragraph{Separation of Content and Presentation} In order to facilitate
      experimentation with different visualisations, there should be a clear
      separation between content and presentation. This allows the authors of a
      presentation to focus on the content while the visualisation is handled
      by the presentation tool. Note that this approach is similar to the
      \latex typesetting system where content is written in a standardised
      structured way and the visualisation is automatically handled by the
      typesetting system. There is also a \latex document class for
      presentations called Beamer and we were inspired by its structured and
      content-driven approach. However, the content-related functionality and
      the visualisation are too limited to be considered as a basis for an
      extensible presentation tool.

     \paragraph{Extensibility} In order for a presentation tool to be
      successful as an experimental platform for new presentation concepts, it
      should be easy to rapidly prototype new content types and presentation
      formats as well as innovative navigation and visualisation techniques. It
      has to be possible to add or replace specific components without
      requiring changes in the core. In order to be truly extensible, a
      presentation tool should provide a modular architecture with loosely
      coupled components. Note that this type of extensibility should not only
      be offered on the level of content types but also for the visualisation
      engine or content structures.

     \paragraph{Cross-Media Content Reuse} In the introduction we briefly
      mentioned the lack of content reuse in existing presentation tools. There
      is a wealth of open education material available but it is rather
      difficult to use this content in presentations. On the other hand, the
      concept of transclusion works well for digital documents and parts of the
      Web (e.g. via the HTML img tag). A modern presentation tool should also
      support the seamless integration of external cross-media content. This
      includes various mechanisms for including parts of other presentations
      (e.g. slides), transcluding content from third-party document formats as
      well as including content from open learning repositories.

     \paragraph{Connectivity} With the rise of social and mobile technologies,
      connectivity for multi-device input and output becomes more relevant in
      the context of presentation tools. Support for multi-directional
      connectivity is required for a number of reasons. First, it is necessary
      for the previously mentioned cross-media transclusion from external
      resources. Second, multi-directional connectivity forms the backbone for
      audience feedback via real-time response or voting systems
      \citep{dufresne-1} as well as other forms of multi-device interfaces.

     \paragraph{Interactivity} We mentioned that content might be more
      interactive and the extensibility requirement addresses this issue since
      the targeted architecture should support dynamic or interactive content
      and visualisations. Nevertheless, the use of mouse and keyboard might not
      be sufficient for components offering a high level of interaction.
      Therefore, a presentation tool should enable the integration of other
      forms of input such a gesture-based interaction based on Microsoft's
      Kinect controller or digital pen interaction \citep{signer-2} as offered
      by the PaperPoint \citep{signer-1} presentation tool.

     \paragraph{Post-Presentation Phase} Even if it was never the original goal
      of slide decks, they often play an important role as study or reference
      material. While the sharing of traditional slide decks after a
      presentation is trivial, this changes when the previously mentioned
      requirements are taken into account. For instance, the nonlinear
      navigation allows presenters to go through their content in a non-obvious
      order or input from the audience might drive parts of a presentation.
      Special attention should therefore be paid to the post-presentation
      phase. It should not only be easy to play back a presentation with the
      original navigational path, annotations and audience input, but its
      content should also be made discoverable and reusable. In accordance with
      the Web 2.0, we see potential for the social aspect in a
      post-presentation phase via a content discussion mechanism.

  \section{\mxp Platform}
   \label{mxp-platform}

   In this section, we present the general architecture of our
   \mxp\footnote{http://mindxpres.com} crossmedia presentation platform which
   is outlined in \figref{roels-1-fig-1} and addresses the requirements
   presented in the previous section.

   \fig{roels-1-fig-1}{\mxp architecture}

   \subsection{Document Format and Authoring Language}

    Content is stored, structured and referenced in a dedicated \mxp document
    format. An individual \mxp document contains the content itself and may
    also refer to some external content to be included. A new \mxp document can
    be written manually similar to the \latex approach introduced earlier or in
    the near future it can also be generated via a graphical authoring tool. In
    contrast to other presentation formats such as Slidy, S5 or OOXML, the
    authoring language eliminates unnecessary HTML and XML specifics and
    focusses on a semantically more meaningful vocabulary. The vocabulary of
    the authoring language is almost completely defined by plug-ins that
    provide support for various media types and structures. In order to give
    users some freedom in the way they present their information, the core \mxp
    presentation engine only plays a supporting role for plug-ins and lets them
    define the media types (e.g. video or source code) as well as structures
    (e.g. slides or graph-based content layouts).

    This is also reflected in the document format as each plug-in extends the
    vocabulary that can be used. Any visual styling including different fonts,
    colours or backgrounds is achieved by applying specific themes to the
    underlying content.

   \subsection{Compiler}

    The compiler transforms a \mxp document into a self-contained portable \mxp
    presentation bundle. While a \mxp document could be directly interpreted at
    visualisation time, for a number of reasons we decided to have this
    intermediary step. First, the compiler allows different types of
    presentations to be created from the same \mxp document instance. This
    means that we can not only create dynamic and interactive presentations but
    also more static output formats such as PDF documents for printing.
    Similarly, we cannot always expect that there will be an Internet
    connection while giving a presentation. For this case, the compiler might
    create an offline version of a presentation with all necessary content
    pre-downloaded and included in the \mxp presentation bundle.  Last but not
    least, the compiler might resolve incompatibility issues by, for instance,
    converting unsupported video formats.

   \subsection{\mxp Presentation Bundle}

    The dynamic \mxp presentation bundle consists of the compiled content
    together with a portable cross-platform presentation runtime engine which
    allows more interactive and networked presentations. Similar to the
    original document, the compiled presentation content still consists of
    both, integrated content and references to external resources such as
    online content that will be retrieved when the presentation is visualised.
    Note that the content might have been modified by the compiler and, for
    example, been converted or extracted from other document formats that the
    runtime engine cannot process.  References to external content may have
    been dereferenced by the compiler for offline viewing.

    A presentation bundle's core runtime engine consists of the three modules
    shown in \figref{roels-1-fig-1}. The \emph{content engine} is responsible
    for processing the content and linking it to the corresponding
    visualisation plug-ins. The \emph{graphics engine} abstracts all
    rendering-related functionality. For instance, certain presenters prefer a
    zoomable user interface in order to provide a better overview of their
    content \citep{reuss-1}. This graphical functionality is also exposed to
    the plug-ins, which can make use of the provided abstractions. The
    \emph{communication engine} exposes a communication API that can be used by
    plug-ins. It provides some basic functionality for fetching external
    content but also offers the possibility to form networks between multiple
    \mxp presentation instances as well as to connect to third-party hardware
    such as digital pens or clicker systems.

    In addition to the presentation content and core modules, the presentation
    bundle contains a set of \emph{themes} and \emph{plug-ins} that are
    referenced by the content. Themes may contain visual styling on a global as
    well as on a plug-in level. When the content engine encounters different
    content types, they are handed over to the specific plug-in which uses the
    graphics engine to visualise the content.

   \subsection{Plug-in Types}

    \fig{roels-1-fig-2}{Structure plug-in examples}

    In order to provide the necessary flexibility, all non-core modules are
    implemented as plug-ins. Even the basic content types such as text, images
    or bullet lists have been realised via plug-ins and three major categories
    of plug-ins have to be distinguished:

    \begin{itemize}

     \item \emph{Components} form the basic building blocks of a presentation.
     They are represented by plug-ins that handle the visualisation for
     specific content types such as text, images, bullet lists, graphs or
     videos. The content engine invokes the corresponding plug-ins in order to
     visualise the content.

     \item \emph{Containers} are responsible for grouping and organising
     components of a specific type. An example of such a container is a slide
     with each slide containing different content but also some reoccurring
     elements.  Every slide of a presentation may for example contain elements
     such as a title, a slide number and the author's name, which can be
     abstracted in a higher level container. Another example is an image
     container that visualises its content as a horizontally scrollable list of
     images. Note that we are not restricted to the slide format and content
     can be laid out in alternative ways.

     \item \emph{Structures} are high-level structures and layouts for
     components and containers. For example, content can be scattered in a
     graph-like structure or it can be clearly grouped in sections like in a
     book.  Both are radically different ways of visualising and navigating
     content but by abstracting them as plug-ins, the user can easily switch
     between different presentation styles as the ones shown in
     \figref{roels-1-fig-2}. Structures differ from containers by the fact that
     they do not impose restrictions on the media types of their child elements
     and may also influence the default navigational path through the content.

    \end{itemize}

   \subsection{Implementation}

    HTML5 and its related web technologies were chosen as the backbone for
    the \mxp presentation platform. Other options such as JavaFX, Flash or game
    engines have also been investigated, but HTML5 seemed to be the best choice.
    The widely accepted HTML5 standard makes \mxp presentations highly portable and
    runnable on any device with a recent web browser, including smartphones and
    tablets. Furthermore, HTML5 provides rich visualisation functionality out of
    the box and the combination with Cascading Style Sheets
    (CSS) and third-party JavaScript libraries forms a potent visualisation
    platform.

    \subsubsection{Document Format and Authoring Language}

     The \mxp document format which allows us to easily express a
     presentation's content, structure and references is based on the
     eXtensible Markup Language (XML). A simple example of a presentation
     defined in our XML-based authoring language is shown in \lstref{xml-1}.
     The set of valid tags and their structure, except the \code{presentation}
     root tag, is defined by the available plug-ins.

     \begin{figure}[h!]
      \begin{lstxml}{xml-1}{Authoring a simple \mxp presentation}
<presentation>
  <slide title="Vannevar Bush">
    <bulletlist>
      <item>March 11, 1890 - June 28, 1974</item>
      <item>American Engineer, founder of Raytheon</item>
    </bulletlist>
    <image source="bush.jpg"/>
  </slide>
</presentation>
      \end{lstxml}
     \end{figure}

    \subsubsection{Compiler}

     The compiler has been realised as a Node.js application. This not only
     allows the compiler to be used via a web interface or as a web service,
     but projects such as node-webkit also enable the compiler to run as a
     local offline desktop application. The choice of using server-side
     JavaScript was influenced by the fact that Node.js is capable of bridging
     web and desktop technologies. On the one hand, the framework makes it easy
     to interact with other web services and to work with HTML, JSON, XML and
     JavaScript visualisation libraries at compile time. On the other hand, the
     framework can also perform tasks which are usually not suited for web
     technologies, including video conversion, legacy document format access,
     file system access or TCP/IP connectivity.

     In order to validate a \mxp document in the XML format described above,
     there is an XML Schema which is augmented with additional constraints
     provided by the plug-ins. After validation, the document is parsed and
     discovered tags might trigger preprocessor actions by the plug-ins such as
     the extraction of data from referenced legacy document formats (e.g.  \ppt
     or Excel) or the conversion of an unsupported video format. The tag is
     then converted to HTML5 by simply encoding the information in the
     attributes of a \code{div} element. The HTML5 standard allows custom
     attributes if they start with a \code{data-} prefix. \lstref{xml-2} shows
     parts of the transformed XML document shown in \lstref{xml-1}. Note that
     the transformation does not include visualisation-specific information but
     merely results in a valid HTML5 document which is bundled into a
     self-contained package together with the presentation engine.

     \begin{figure}[h!]
      \begin{lstxml}{xml-2}{Transformed HTML5 presentation content}
<div data-type="presentation">
  <div data-type="slide" data-title="Vannevar Bush">
    <div data-type="bulletlist">
      ...
      \end{lstxml}
     \end{figure}

    \subsubsection{Presentation Engine}

     The presentation engine's task is to turn the compiled HTML content into a
     visually appealing and interactive presentation. As highlighted in
     \figref{roels-1-fig-1}, the presentation engine consists of several
     smaller components which help plug-ins to implement powerful features with
     minimal effort. The combination of these components enables the rapid
     prototyping and evaluation of innovative visualisation ideas. A resulting
     \mxp presentation combining various structure, container and component
     plug-ins is shown in \figref{roels-1-fig-3}.

     \fig{roels-1-fig-3}{A \mxp presentation}

     \paragraph{Content Engine} When a presentation is loaded, the content
      engine is the first component that is activated. It processes the content
      of the HTML presentation by making use of the well-known jQuery
      JavaScript library. Whenever a \code{div} element is discovered, the
      \code{data-type} attribute is read and the corresponding plug-ins are
      notified in order to visualise the content.

     \paragraph{Graphics Engine} The graphics engine provides support for
      interesting new visualisation and navigation styles. Next to some basic
      helper functions, it offers efficient panning, scaling and rotation via
      CCS3 transformations and supports zoomable user interfaces as well as the
      more traditional navigation approaches.

     \paragraph{Communication Engine} The communication engine implements
      abstractions that allow plug-ins to retrieve external content at run
      time. It further provides the architectural foundation to form networks
      between different \mxp instances or to integrate third-party hardware
      \citep{roels-2}. For our \mxp prototype, we used a small Intel Next Unit
      of Computing Kit (NUC) with high-end WiFi and Bluetooth modules to act as
      a central access point and provide the underlying network support. \mxp
      instances use WebSockets to communicate with other \mxp instances via the
      access point. The access point further acts as a container for data
      adapters which translate input from third-party input and output devices
      into a generic representation that can be used by the \mxp instances in
      the network. In order to go beyond simple broadcast-based communication,
      we have implemented a routing mechanism based on the publish-subscribe
      pattern where plug-ins can subscribe to specific events or publish
      information. The communication engine provides the basis for audience
      response systems \citep{roels-2} or even full classroom communication
      systems where functionality is only limited by the creativity of plug-in
      developers.

     \paragraph{Plug-ins} Plug-ins are implemented as JavaScript bundles which
      consist of a folder containing JavaScript files and other resources such
      as CSS files, images or other JavaScript libraries. As a first
      convention, a plug-in should provide a manifest file with a predefined
      name. The manifest provides metadata such as the plug-in name and version
      but also a list of tags to be used in a presentation. The plug-in claims
      unique ownership for these tags and is in charge for their visualisation
      if they are encountered by the content engine. As a second convention, a
      plug-in must provide at least one JavaScript file implementing certain
      methods, one of them being the \code{init()} method which is called when
      the plug-in is loaded by the presentation engine. It is up to the plug-in
      to load additional JavaScript or CSS via the provided dependency
      injection functionality. A second method to be implemented is the
      \code{visualise()} method which is invoked with a pointer to the
      corresponding DOM node as a parameter when the content engine encounters
      a tag to be visualised. A plug-in is free to modify the DOM tree and may
      also register callbacks to handle future interaction with the content.

     \paragraph{Themes} We currently use CSS to provide a basic templating
      system. These themes offer styling either on a global or on a plug-in
      level. However, we see this as a temporary solution as it is not
      well-suited for alternative compiler outputs (e.g. PDF) and a more
      generic templating scheme is planned for the future.

   \subsection{Use Cases}

    In order to validate the architectural and technological choices, we
    demonstrate the extensibility and feasibility of \mxp as a rapid
    prototyping platform by presenting a number of content- and
    navigation-specific plug-ins that have been developed so far. Additional
    plug-ins for audience-driven functionality such as real-time polls, screen
    mirroring and navigational takeover can be found in \citep{roels-2}.

    \subsubsection{Structured Overview Plug-in}

     In Section \ref{mxp-platform} we have explained how structure plug-ins may
     change the way presentations are visualised and navigated. In order to
     illustrate this, we have implemented a structure plug-in called
     \emph{structured layout} which combines a zoomable user interface with the
     ability to group content into sections. The resulting visualisation of the
     \emph{structured layout} plug-in is shown in \figref{roels-1-fig-3}.
     Whenever a new section is reached, the view is zoomed out to provide an
     overview of the content within the section and communicate a sense of
     progress.

    \subsubsection{Slide Plug-in}

     In order to also support the traditional slide concept, we created a
     slide-like container plug-in. While the benefits and issues of using
     slides with a fixed size are debatable, we implemented this plug-in as a
     proof of the framework's versatility. The main function of the slide
     plug-in is to provide a rectangular styleable component container with a
     title and some other information. Containers may also offer functionality
     to layout their content. In this case, the slide plug-in offers a quick
     and easy layout mechanism which allows the presenter to partition the
     slide into rows and columns. Content is then assigned to these slots in
     the order that it is discovered. The use of the slide plug-in together
     with the resulting visualisation is exemplified in \lstref{roels-1-fig-4}.
     It also shows the use of the image plug-in (a component plug-in) which
     enables a simple form of cross-media transclusion. A visualised external
     image can be cropped and filters (e.g. colour correction) may be applied
     without duplicating or modifying the original source.

     \begin{figure}[h!]
      \begin{subfigure}{0.53\textwidth}
       \vspace{-1em}
       \begin{lstxml*}
<slide layout="\60\40" title="Vannevar Bush">
 <bulletlist>
  <item>About
   <item>March 11, 1890 - June 28, 1974</item>
   <item>American Engineer</item>
   <item>Founder of Raytheon</item>
  </item>
  <item>...
  </item>
 </bulletlist>
 <image source="http://example.com/bush.jpg">
  <crop bounds="10%, 5%, -10%, -20%" />
 </image>
</slide>
       \end{lstxml*}
      \end{subfigure}
      \hfill
      \begin{subfigure}{0.43\textwidth}
       \fignl{roels-1-fig-4}
      \end{subfigure}
      \renewcommand{\figurename}{Listing}
      \renewcommand{\figureshortname}{Lst.}
      \cl{roels-1-fig-4}{Slide plug-in}
     \end{figure}

    \subsubsection{Enhanced Video Plug-in}

     When videos are used in educational settings, we often need more
     functionality than what is offered by the average video player
     \citep{reuss-1}. \mxp provides the enhanced video plug-in shown in
     \lstref{roels-1-fig-5} with the possibility to overlay a video with text
     or arbitrary shapes. This overlay functionality can be used as a basic
     captioning system as well as to highlight items of interest during
     playback.

     Furthermore, we added the option to trigger certain events at specified
     times. One can define that a video should automatically pause at a certain
     point, highlight an object and continue playing after a specified amount
     of time. Additional features include the bookmarking of certain positions
     in a video for direct access or the possibility to display multiple videos
     in a synchronised manner. Our enhanced video plug-in injects the default
     HTML5 video player and overlays it with a transparent \code{div} element
     for augmentation. Currently we make use of the HTML5 video API to
     synchronise the creation and removal of overlays but a SMIL-based
     implementation might be used in the future.

     \begin{figure}[h!]
      \begin{subfigure}{0.53\textwidth}
       \vspace{-1em}
       \begin{lstxml*}
<video source="vid.mp4">
 <caption start="0:00" duration="1500ms">
  Lecture 3 - Butterfly Species
 </caption>
 <pause start="0:43" duration="5s">
  <caption>
   The peacock butterfly (aglais io) ...
  </caption>
  <highlight x="30%" y="9%"
             width="35%" height="40%" />
 </pause>
</video>
       \end{lstxml*}
      \end{subfigure}
      \hfill
      \begin{subfigure}{0.43\textwidth}
       \fignl{roels-1-fig-5}
      \end{subfigure}
      \renewcommand{\figurename}{Listing}
      \renewcommand{\figureshortname}{Lst.}
      \cl{roels-1-fig-5}{Enhanced video plug-in}
     \end{figure}

 \newpage

    \subsubsection{Source Code Visualisation Plug-in}

     Earlier, we mentioned the difficulty of visualising complex resources such
     as source code. Our \mxp source code plug-in exports a \code{code} tag
     which allows the presenter to paste their code into a presentation and
     have \mxp visualise it nicely by making use of syntax highlighting via the
     SyntaxHighlighter\footnote{http://alexgorbatchev.com/SyntaxHighlighter/}
     JavaScript library. Whenever the content engine encounters a \code{code}
     tag, it invokes the code plug-in to beautify the code and automatically
     adds vertical scrollbars for larger pieces of source code as shown in
     \lstref{roels-1-fig-6}.

     \begin{figure}[h!]
      \begin{subfigure}{0.53\textwidth}
       \vspace{-1em}
       \begin{lstxml*}
<code>
 <publications>
  <publication type="inproceedings">
   <title>An Architecture for Open Cross-Media
          Annotation Services</title>
   <author>
     <surname>Signer</surname>
     <forename>Beat</forename>
   </author>
   <author>
     <surname>Norrie</surname>
     <forename>Moira</forename>
   ...
</code>
       \end{lstxml*}
      \end{subfigure}
      \hfill
      \begin{subfigure}{0.43\textwidth}
       \fignl{roels-1-fig-6}
      \end{subfigure}
      \renewcommand{\figurename}{Listing}
      \renewcommand{\figureshortname}{Lst.}
      \cl{roels-1-fig-6}{Source code visualisation}
     \end{figure}

   \subsection{Discussion and Future Work}

    \mxp currently supports transclusion and cross-media content reuse on the
    plug-in level. For instance, the image or video plug-in can visualise (and
    enhance) external resources, a dictionary plug-in might retrieve
    definitions on demand via a web service or we might create a plug-in that
    allows us to import content (e.g. \ppt slides) from legacy documents at
    compile time. Nevertheless, we are currently investigating the introduction
    of generic reuse tags in our document format which would allow the
    presenter to transclude arbitrary parts of other \mxp presentations. While
    our focus has been on the cross-media aspect of resources that can be used
    in a presentation, we might also investigate the cross-media publishing
    aspect via alternative compiler output formats.

    We are aware that the current authoring of \mxp presentations has some
    usability issues. The average presenter cannot be expected to construct an
    XML document or any CSS themes. In order to tackle this issue and further
    evaluate \mxp in real-life settings, we are currently developing a
    graphical \mxp authoring tool. We further intend to provide a central
    plug-in repository which would make it easy for novice users to find,
    install and use new plug-ins via the authoring tool. In the long run, we
    intend to revise the use of monolithic documents and move towards
    repositories of semantically linked information based on the RSL hypermedia
    metamodel \citep{signer-3}. This would not only promote content reuse and
    sharing, but also create opportunities for context-aware as well as
    semi-automated presentation authoring where relevant content is recommended
    by the authoring tool.
   %</TODO rework>

  \section{Layout}

   Proper layout is incredibly important when trying to transfer knowledge and
   information through written and visual media. Layout can help clarify
   boundaries and relations between pieces of information, by grouping and
   separating them appropriately. Layout is one component of a presentation's
   design, that --- combined with other decisions --- determines the number and
   nature of the visual representations of the information the creator wants to
   communicate, along with its format\footnote{The way the visual objects are
   realised (e.g. as text, graphics, UI widgets\ldots), and their attributes
   (e.g. color, texture, font\ldots)}. The layout of a presentation can have a
   tremenduous influence on its effectiveness in communicating information to,
   and obtaining information from, the audience it is meant to interact with.
   The importance of individual objects can be emphasised or minimised, and the
   connection between obects can be clarified or blurred. A well laid out
   presentation can provide a narrative for the viewer to discover, inferring
   correct links between the objects along the way, and to accomplish tasks
   quickly and correctly, increasing the presentation's effectiveness.
  
   Creating a good layout is almost never easy. People often spend more time on
   the layout of their presentation than the content. Most, if not all,
   decisions in layout are made by human beings. Some of them are professional
   designers who spend years learning and figuring out how to create effective
   layouts, and even then they may take hours or days to create even a single
   screen of a presentation. In fact, the more someone knows about proper
   layout and design, the more time they may spend perfectioning their work.
   However, sometimes time-critical information must be communicated and the
   layout process is too expensive and too slow to address these situations.
   This can be a serious problem (see also section \ref{nasa}). Many software
   packages have been developed to make this process easier, to get better
   results, to give more or less control to the creators. Many different
   approaches have been taken, and yet most of them still involve having a
   human being make the final decisions on the layout.

   There are tools like \ppt, which give you some guidelines and some templates
   but generally let you do your own thing. If your own thing is entirely
   different from any best practices on layout, nothing will stop you. Other
   tools like \latex give you complete control over every aspect of the layout,
   while setting some sensible defaults so that you can get a good-looking
   layout without much effort, while still letting you do whatever you want
   once you overcome the steep learning curve that separates the casual users
   from the experts. There are tools that combine the power of \latex with the
   comfort of WYSIWYG editors, bringing the casuals a bit closer to the
   experts. But all of those tools have one thing in common: every aspect of
   every layout they create has, at some point, been designed and decided upon
   by a human being.
  
   Aesthetics are a natural phenomenon, and the creation of aesthetically
   pleasing layouts is therefore a manifestation of our instincts. As with most
   instincts, it has proven difficult to translate this into a concept that can
   be understood by a computer. Moreover, it is still difficult to explain it
   in human terms, which --- according to a popular quote often attributed to
   Albert Einstein --- proves we don't fully understand it ourselves.

   When we look to other technologies, we do find some automated layout
   implementations. For example, the web has had to adapt to mobile devices
   with small screens over the past few years, and has done this gracefully by
   creating the concept of responsive design. In short, this allows websites to
   adapt their layout to any screen, no matter the size. While this is often a
   hard-coded difference, where effectively two or more versions of the same
   webpage are created aimed at different screen sizes, some websites take a
   more dynamic approach based on constraints. As the space the page is to be
   displayed on gets smaller, the layout algorithm may decide to display
   content below other content instead of side-by-side, it may scale images to
   fit the screen, it may even switch fonts and font sizes if necessary.

   This constraint-based technique is described in a few papers
   \citep{lok-1,hurst-1}, but has --- to our knowledge --- not been applied in
   any presentation software so far. This is surprising, as presentations often
   look like they could use some of this magic. A proper constraint-based
   layout algorithm could allow any user to drop content onto a slide, without
   worrying about clarity or even legibility, and the algorithm could take care
   of the rest. Of course, there are some limits in traditional slideware that
   may hinder this approach: if a user decides to put more content on a slide
   than there is physical room available, the algorithm could either make the
   content smaller or split it across several slides, but either solution may
   bring its own problems up. An advantage of ZUI's is that no matter how small
   the content gets, we can still zoom in to make it clear again\footnote{It
   should be noted that \mxp in its current form does not support this level of
   zooming. While the software can zoom out to provide an overview of the
   presentation while zooming in on the separate components, it is not yet
   possible to zoom in or out extremely to reveal 'hidden' parts of the
   presentation. This is something we encourage to look into and change,
   because it can greatly improve both our layout solution as well as the whole
   \mxp experience in general.}.

%   maybe TODO more stuff from papers about automated layout goes here

   \subsection{Algorithms}
    \label{related-algorithms}

    When researching layout algorithms, one will often come across the very
    active field of graph layout \citep{battista-1}. We will not go into the
    specifics of this field, as most of the issues with which it is concerned
    are specific to problems caused by the explicit visual representation of
    graph edges --- for example, the minimisation of edge crossing
    \citep{battista-2, shahrokhi-1}. The same applies to automated layout as
    referring to automated circuit layout for VLSI chip fabrication
    \citep{hu-1, lengauer-1} as well as automated placement of pieces to be cut
    from a bolt of cloth used to produce clothing \citep{milenkovic-1}.
    Contrary to presentation layouts (including graph layouts), these layouts
    are designed to meet the requirements of a fabrication process, rather than
    to make them understandable to humans. While some techniques used therein
    definitely apply to our more general problem of automated presentation
    layout (e.g. general constraint solvers) others decidedly do not (e.g.
    bin-packing techniques \citep{hofri-1} that result in minimal area layouts
    at the expense of maintaining visually obvious relationships between
    objects).



 % vim:ts=1:et:nospell:spelllang=en_gb:ft=tex

 \chapter{Slideware and the importance of layout}

  Computers, software and digital content are everywhere. Everything we use
  nowadays is somehow related to computers and electronics, and if it isn't, it
  probably will be soon. This may be a bit of hyperbole, but there's a core of
  truth in it. If you think about it, more and more things have become and are
  becoming some kind of computer. Coffee machines used to be simple machines
  that heated water and let it drip over coffee grounds; now there are coffee
  machines that are connected to the internet, and can be turned on remotely
  from your smartphone. That smartphone itself is an incredible evolution as
  well: just 20 years ago, phones were analog devices, and you could use them
  to call people and nothing more. Today, our phone does a lot more than that,
  so much more that calling has actually become a minor feature to most people.

  Content is going the same way. Photos used to be on a special film, and could
  be `developed' onto special paper through a proces involving a dark room and
  several chemicals. Movies existed on a projection film, newspapers were
  actually made of paper and music was available on vinyl disks with grooves
  that matched the sound waves. All of this content has been digitized since.
  This means of course that you can see or hear it using a computer, like you
  would've seen it without a computer before, but on top of that it means the
  content can be much more dynamic. You can link it to other content, you can
  make it respond to your actions, you can discuss it with people around the
  world. Digitized content allows for interactivity, so that the audience is no
  longer a passive onlooker but an active participant.

  It is no surprise, then, that slideshow presentations have evolved from the
  original dia's or overhead projection slides into a digital form as well.
  Except, until recently the evolution stopped there. Slideshows did not become
  interactive, and the audience remained passive onlookers watching a series of
  images projected on a screen or a wall. The presenter told a story, and the
  audience listened. Often during or at the end of the presentation there would
  be a chance to ask questions, but those questions could only be answered
  vocally by the presenter. If the question needed any visual explanation, the
  slideshow would not be able to help. We had digital slides, but the
  difference with the physical slides was neglectable.

  In our eyes, the culprit for this is Microsoft's \ppt. This software package
  took the world by storm, making it possible for everyone with a computer to
  make digital slideshows, which was impressive at the time. However, \ppt*
  never really evolved beyond that. It did add features that fit within the
  slideware concept, but never went beyond that comfort zone. Since it was ---
  and still is! --- the dominant player in the world of slideware with over
  90\% market share, this apathy towards change firmly rooted slideware in the
  concepts of the past. Luckily, a few years ago some people realized this and
  decided to take matters into their own hands. They stepped away from the
  classic slide format, allowing for any kind of layout, combined with zoomable
  interfaces and other methods of displaying data. One such alternative is
  \mxp, created in the WISE lab at the VUB.

  \mxp is based on a plug-in architecture. Plug-ins can do anything from
  arranging data in a certain way to letting the audience control the
  slideshow. Virtually anything is possible if you only implement it, and
  implementing it is fairly simpel if you know a bit about web development as
  the whole thing is written in HTML5. Other software packages have plug-ins
  too of course, but they have a limited set of functionality available to
  them, they're not as easy to implement, and most importantly: they're bound
  by the same slide format used since overhead projections.

  However, even with the new alternatives, \ppt* remains the most-used slideshow
  software. People keep using it because it's familiar, they've used it
  hundreds of times before and as such all their existing work is viewable only
  through \ppt. Switching to a new software package is hard. This thesis aims
  to make the transition easier, by providing a way to convert existing \ppt
  presentations into \mxp. On top of that, we try to find a way to immediately
  release the transferred content from the confines of classic slides, by
  instead automatically figuring out the best possible layout for the content
  we extracted from the original \ppt file.

  \section{Terminology}

%   maybe TODO use fancy words to explain other fancy words

   The words \emph{slideshow} and \emph{presentation} are often used
   interchangeably throughout this report, although they do not quite cover the
   same meaning. By \emph{slideshow} we mean a presentation consisting of a set
   of slides, the kind \ppt* and many other presentation software provide us
   with. \emph{Presentation} then refers to the wider concept of material
   intended to be viewed and manipulated by people in order to convey
   information, usually but not necessarily from one or several presenter(s) to
   an audience.

   The term \emph{layout} refers to both the process of determining the
   position and size of each visual object that is to be displayed in a
   presentation, and the result of that process.

   \emph{Slideware} is a contraction of the words `slideshow' and `software',
   referring to software packages used to create slideshow presentations.

  \section{Problem statement}

   According to several sources \citep{parker-1, drucker-1, bajaj-1}, over 30
   million \ppt presentations are being made every day. That is an enormous
   amount. Creating a \ppt presentation is easy; creating a good \ppt
   presentation, however, is not. Slides have a fixed size, and you can only
   fit so much information on one slide before the effectiveness of
   transferring that information to one's audience starts deteriorating. Over
   the years, many people have created written and unwritten guidelines to
   creating effective slideshows, specifying how much text and how many images
   should fit on one slide. Over those years, many people have failed to follow
   those guidelines. But whether you choose to follow the guidelines or
   not, one thing remains true: people who create slideshow presentations spend
   most of their time not on the \emph{content} of their presentation, but on the
   \emph{layout} \citep{lok-1}.

   The layout of a presentation can have a significant impact on how well it
   communicates information to and obtains information from those who interact
   with it. The vast majority of layouts created today are done ``by hand'': a
   human graphic designer or ``layout expert'' makes most, if not all, of the
   decisions about the position and size of the objects to be presented.
   Designers typically spend years learning how to create effective layouts,
   and may take hours or days to create even a single screen of a presentation.
   Designing presentations by hand is too expensive and too slow to address
   situations in which time-critical information must be communicated.

   Since layout is such a hard skill to master, we propose to automate this
   task, letting the presenter focus on the content of the presentation and
   providing a proper layout fit for the content provided through software.

   \subsection{Real-life slideware problems}
    \label{nasa}

    It may seem like an overstatement to emphasize the significance of layout
    and formatting in presentations. One could assume these issues are
    irrelevant, or that only inexperienced presenters would make these mistakes.
    The real-life example of the space shuttle Columbia illustrates that this is
    not always the case. Leading up to the tragic incident in which the shuttle
    burned up during re-entry after spending 2 weeks in orbit, Boeing
    Corporation engineers delivered three reports to {NASA} totalling 28 \ppt
    slides, to help them assess the damage caused by a piece of debris hitting
    the wing of the shuttle during launch, and the threat this damage might have
    posed.  As Edward Tufte beautifully describes in his article ``\ppt Does
    Rocket Science'' \citep{tufte-2}, the reports existed only in those slides,
    and the slides were woefully inadequate for the task at hand. Although Tufte
    likes to suggest this proves that \ppt is an inherently bad tool, what it
    really proves is that \ppt makes it easy to create bad presentations, and a
    tool that either discourages this manner of presenting information or makes
    it altogether impossible would be a great improvement.



 % vim:ts=1:et:nospell:spelllang=en_gb:ft=tex

 \chapter{Approach}

  In this chapter we explain the different approaches we tried in order to
  reach our goal and find a solution for the problem we described. As you will
  see, this was not immediately a straightforward process but rather one of
  trial and error. The goal was clear, the starting point was clear as well,
  but as often in computer science, there is more than one way to get from
  point A to point B, and it is not always clear which way is the best,
  easiest, most efficient or most effective.
 
  Since we're talking about the approach here, and not the implementation (for
  that, see chapter \ref{implementation}), we start by describing in broad
  terms what needs to be done and how this should be done, then we refine until
  we have a full set of specifications ready for implementation, where the last
  details will be ironed out.

  Unfortunately it is possible to refine an approach until it is ready for
  implementation, and only find out during implementation that the approach
  you've chosen will not work. This happened during our work on creating an
  automated layout system. Luckily we still had time to go back to the drawing
  board, and we did not have to restart from scratch; large parts of our
  approach were correct, the basic layout process we thought out was still a
  viable part of the approach, but it turned out we would have to split up the
  conversion and layout parts into two separate processes, rather than
  implementing them as two steps of the same process.
  
  Specifically, we had thought at first to figure out the ideal layout during
  conversion, when we would have all the separate components, by immediately
  putting them in the right place. This idea was partly conceived after looking
  at the HTML code generated by the \mxp compiler, thinking we would generate
  the same HTML code in our conversion process. It turned out we could bypass
  the \mxp compiler this way, but that wouldn't be necessary: we could just as
  well generate \mxp XML and have the compiler take care of the rest for us.
 
  We also found during implementation that generating a layout in Java would
  not easily give us the results we were hoping for. However, at this point we
  had realized generating \mxp XML would be a better option, so we could have
  \mxp take care of the layout for us. Except \mxp didn't do fully automated
  layout yet, the layout system was mostly template-based, so we decided we
  would need to write our own \mxp plugin that would solve this problem for us.

  \section{Conversion process}

   The first part of the approach is fairly straightforward in its basic
   explanation: we had to convert \ppt presentations into \mxp presentations.
   This involved finding out how \ppt presentations are structured, getting the
   parts wee need out of that structure, and then putting those parts together
   in de \mxp structure.

   It appeared soon enough to us that the nature of this process resembled that
   of a compilation process. A compiler takes source code and transforms it
   into a working program with the semantics described by that source code. The
   compilation process consists of several steps. First the source code is
   tokenized, which means the symbols in the code are identified one by one and
   classified in certain categories.

   Then the tokens are processed by a parser into an intermediary form called a
   parse tree. A parser looks for certain predefined patterns in the source
   code. These patterns are part of the source code's language syntax. As such,
   these two steps analyse and validate the source code's syntax. If part of
   the code does not match any pattern, the parser and the compilation process
   stop and the user gets a message saying the code's syntax is invalid.

   When a parse tree is constructed, the compilation process can alter it, to
   improve it. Certain patterns in the parse tree may be replaceable by
   different patterns with the same outcome, but with more optimal execution.
   This part of the compilation process is optional, and is called compiler
   optimization. Optimizations can consist of many things, depending on the
   language. For example, some languages guarantee tail call optimization,
   where infinite loops can be constructed by letting a function call itself as
   its last statement without causing a stack overflow. This is something the
   compiler (or interpreter) can optimize during this part of the compilation
   process.

   After this, the parse tree can be written out to produce the desired output.
   Every node in the tree has a well-defined equivalent in the target
   language's syntax. The target language can be Assembly, which consists of
   the exact instructions a CPU needs to carry out a program, or it can be
   another programming language. Many compilers of higher-level languages
   translate their language into C, for several reasons: the C compilers that
   translate C into Assembly have been optimized so much that it is easier to
   rely on them than to put an enormous amount of effort into optimizing
   another language; C compilers exist for most --- if not all --- CPU
   architectures, which means translating a language into C makes it compatible
   with all those architectures, while it would cost a lot more effort to write
   different compilers for every architecture you would want to make your
   language available on.

   The conversion tool that is the purpose of this thesis, can be described in
   a similar succession of steps. As a first step, we take a \ppt presentation
   and take it apart into its components, effectively walking over each
   component, classifying them and registering their content type, original
   position and size, and any other specific properties. This can be seen as
   the tokenization phase, after which we end up with a series of `tokens' or,
   in our case, presentation components.
  
   We then turn this series of `tokens' into a `parse tree', an intermediary
   structure that reflects the relation between the components and the
   hierarchy of the presentation, which may consist of chapters, sections,
   slides and component groups. In \ppt this structure is fairly simple, so the
   creation of this `parse tree' is a straightforward process.
  
   However, in \mxp we are not limited to the rigid hierarchy of sections and
   slides, so at this point we can actually start manipulating our tree and
   improve upon it, for example by moving parts around, nesting components in
   different ways, grouping them in other ways than they originally were, etc.
   In compilation terms, this is the optimization phase, where the compiler can
   manipulate the program to run more efficiently, to replace parts of it with
   other functionality, or to add features the source didn't explicitly specify
   (e.g. garbage collection, but also spyware components \citep{scahill-1}). 

   As we discuss in section \ref{compiler-optimizations}, this seemed like the
   right time to incorporate automated layout generation into the conversion
   process. As we see later in section \ref{mxp-plugin}, it turned out it
   wasn't. In the end, no significant `optimizations' or manipilation of the
   tree structure were implemented. Perhaps in the future some other problem
   may be solved by utilizing this optimization phase, but in its current
   incarnation it does not affect the end result in any way.
  
   To finish the conversion process, we can traverse our component tree and
   generate a \mxp presentation from it. This can be done in several ways,
   since our intermediary form is in no way dependant on or bound to a specific
   format. Since the \mxp compiler was unavailable for a long time during our
   research and implementation, we decided it would be best to go straight to
   HTML5, so that we could test the conversion process without relying on the
   \mxp compiler. This worked out fairly well, although manually constructing
   HTML5 to work with the \mxp JavaScript library proved difficult. We ran into
   several issues, often mostly due to our lack of knowledge of the inner
   workings of \mxp, but we managed to get a presentable result that emulated
   the original \ppt presentation quite well.

   Afterwards, we altered our conversion tool to generate \mxp XML instead,
   which was a lot simpler since we would rely on \mxp to provide our layout
   and other things for us through the \mxp compiler. This approach allowed us
   to use the full power of \mxp, including our own plugin for automated
   layout. At this point, the optimization phase was also revisited, and
   leveraged to introduce specific XML tags around component groups that would
   trigger our automated layout plugin.

  \section{Compiler optimizations}
   \label{compiler-optimizations}

   Since the conversion process resembles that of a compiler, it seemed logical
   at first to make automatic layout a part of that process, as some kind of
   `compiler optimization'.

   At first, we tried to traverse the component tree, giving its objects new
   coordinates and sizes so that they would fit together on every slide as well
   as possible. This seemed an easy solution, but the results were sub-optimal.
   On top of that, we soon realised that we were in essence creating another
   template out of which a presentation would be made, which was exactly the
   opposite of what we were trying to do. As such, we abandoned this approach.

   We then switched to a different method: defining constraints for every
   component, in the form of margins, maximum sizes and other limits, and then
   calculating a way to satisfy all constraints while fitting content together
   on each slide. While this is clearly a better method, it turned out the
   compiler optimization phase was not the best place in the process to take
   care of this.

   In the end, we decided to take a different approach, relying on the layout
   engine of \mxp itself and enhancing that engine to create the automatic
   layout we wanteD.

  \section{Using \mxp}
   \label{mxp-plugin}

   One of the primary goals of \mxp is to separate content from layout,
   allowing the author of a presentation to focus on the content while \mxp
   takes care of the layout. The way it does this is currently mostly through
   the compiler, which decides the width, height and coordinates of content,
   relative to the container the content belongs to. The plug-ins responsible
   for handling components and containers currently don't mess with those
   settings, but technically, they could. The compiler decides the measurements
   and coordinates based on templates. The solution we were looking for was a
   layout engine that could take any content and put it in an appropriate
   layout without any directions from the user. As such, we had to enhance
   \mxp's layout engine to use constraints, based on the size of the content,
   and try to find an optimal position for every component it is given.

   We did this by creating an invisible container plug-in. Containers are a way
   of grouping components, other containers, etc. in \mxp. This means they have
   control over their child elements, which gives us the opportunity to
   override the layout of those elements. A container plugin thus allows us to
   implement our own layout system. Since it's a new element, it doesn't
   override existing elements as it would have done if we had, for example,
   rewritten the `slide' plugin. The user can decide for themself whether or
   not to use it, and it can be used anywhere in the presentation: wrap the
   whole presentation in it, or just a small part, whichever works best for
   your purposes. It also won't break existing presentations that don't use it,
   while those presentations can very easily be altered to take advantage of
   it.

   An important aspect of this is that containers can be nested. This means we
   can create slide-based presentations, which can contain our auto-layout
   container, which then contains the slide's contents, thus creating an
   optimal layout of the content per-slide. Another way of using it could be
   without slides, throwing all content together in one auto-layout container,
   and letting it take care of the layout for the whole presentation at once.
   It should be noted here that the auto-layout container makes each of its
   child nodes focusable separately, to compensate for arbitrary resizing it
   may perform on large objects in order to fit them next to other content, by
   using the focus functionality to automatically zoom into these components
   when necessary.
  
   We call it an \emph{invisible} container plug-in because it does not
   introduce any visual content, shape or indication for itself. Compare with
   the \emph{slide} plug-in which obviously puts some kind of slide-look around
   the content it encompasses, and it becomes clear what we mean by this:
   although the content within is obviously affected by our plug-in, there is
   no visible indication of its presence to the audience.

   The plug-in uses the compiler's numbers to decide relative locations between
   components, as well as size ratios, and then finds a way to display those
   components in a way that the display order makes sense (or at least matches
   the intended order as closely as possible), that no overlapping occurs
   (since we don't have the animations that \ppt might have used to display one
   piece of information and then another on top of it), and resizing everything
   if necessary in order to fit within the specified container. While this may
   seem like a bad idea since content can get illegibly small this way, keep in
   mind that we can rely on the ZUI to focus on each component separately, or
   on groups of components, while \ppt obviously can only display the whole
   slide at once.

%   TODO we need more content here

%   15:50 <omega> zeg, ik zit nu al een hele tijd thesis te schrijven en de laatste paar dagen vooral te zeveren over layout, maar intussen doe ik nog ni echt iets van layout, met t gedacht van ik schrijf daar binnenkort ne mindxpres plug-in voor en klaar
%   15:51 <omega> maar wordt layout momenteel eig ni mostly door de compiler gedaan?
%   15:52 <omega> ben zo eens naar de presentation.js libs en code gaan kijken, en ik zie ni direct een manier om ne plug-in layout te laten doen, aangezien plug-ins mostly component-specifiek zijn en ni alle componenten kunnen aansturen
%   15:53 <omega> dus klopt het dat ik ofwel de compiler moet aanpassen, ofwel presentation.js hacken om dat soort plug-ins toe te laten?
%   15:53 <omega> of laat het dat soort plug-ins al toe maar zijn er gewoon nog geen?
%   16:30 <omega> de 'structured' plug-in doet wel layout van slides, maar binnen die slides zie ik niet meteen een systeem dat layout regelt, met templates of otherwise, het pakt gewoon de coordinaten en afmetingen die de compiler bepaald heeft
%   16:31 <omega> al zou die slide plug-in wel *kunnen* prutsen met die layout... dus mss moet ik gwn de slide plug-in uitbreiden/hacken/vervangen
%   13:22 <reinout> ik zou een container plug-in maken
%   13:22 <reinout> gelijk de slide
%   13:22 <reinout> maar dan onzichtbaar
%   13:23 <reinout> want containers kan je nesten
%   13:23 <reinout> dus een slide kan bv uw layout container bevatten, die dan de children een layout geeft
%   13:23 <reinout> maar op die manier is uw layout ding bruikbaar buiten slides
%   13:24 <reinout> (alternatief was uw layout stuff in de slide plug-in steken)
%   13:31 <omega> oeh, cool idee indeed, beter dan de slide plug-in abusen


 % vim:ts=1:et:nospell:spelllang=en_gb:ft=tex

 \chapter{Implementation}

  To implement the \emph{ppt2mxp} conversion tool that is the subject of this
  thesis, we chose the Java programming language \citep{gosling-1}, version 8.
  Although the author has significant experience with lots of other, more
  interesting, more compelling, more fun languages, several reasons pushed us
  towards Java, the least of them being its ease of use. Of course, Java
  \emph{is} easy to use --- it would not have become as popular as it is
  nowadays if it wasn't. It has a fairly clear and logical syntax, a consistent
  structure, and an extensive standard library. At conception in 1995, its
  performance was abysmal, but through the years it has steadily improved and
  somewhere between Java 5 and 6 it became an industry standard.

  Quite a number of IDEs have been created to further improve developers'
  experience working with Java. Netbeans, Eclipse and IntelliJ come to mine,
  although there are many others, and of course you can still write Java using
  a standard (or advanced) text editor such as Notepad or VIM. While the author
  usually prefers the latter for any kind of text editing --- this very
  document was written entirely using VIM --- the weapon of choice when it
  comes to Java is currently IntelliJ. The way IntelliJ practically writes more
  than half of the code automatically for you is something no other IDE has
  been able to match. Naturally, this is the author's personal opinion and
  should not be seen as fact, but if you're looking for a new Java IDE, it's
  definitely worth checking out. The prospect of using IntelliJ for this thesis
  has definitely contributed to the decision of using Java. It should be noted
  that, had another Java IDE been required, this thesis might never have seen
  the light of day.

  The vast and extensive amount of libraries available for Java was obviously
  one of the more important reasons to make this choice. The existence of the
  Apache POI library (see section \ref{poi}) was a huge help in reaching our
  goal; without it, we would have had to figure out the very obfuscated .ppt
  file format structure, which undoubtedly would have taken up more time than
  was available to us. Other libraries like Spring, which allows the programmer
  to use and reuse components without writing complex systems to instantiate
  them, further increased our resolve to make Java our primary technology
  choice.

  However, Java is not the only technology used here. \mxp is written entirely
  in HTML5, so any tool that somehow relates to \mxp sooner or later needs to
  use HTML5 as well. The widely accepted HTML5 standard makes \mxp
  presentations highly portable and runnable on any device with a recent web
  browser, including smartphones and tablets \citep{roels-1}.

  In the following sections we discuss how the various technologies were used
  to create the \emph{ppt2mxp} tool.

  \section{Taking \ppt apart}
   \label{poi}

   When converting one file format into another, the first part of the process
   involves getting the data you need out of the original file. This can be
   very complicated, as some --- usually proprietary --- file formats are
   deliberately designed to discourage this. They obfuscate data, encrypt it,
   and structure it in illogical and unexpected ways, amongst other techniques.
   The \ppt file format unfortunately is such a format, as Microsoft wouldn't
   want to risk other companies making software that would work with \ppt
   files. Of course, over the years people have managed to crack the format,
   enabling the conversion of \ppt presentations into other formats, although
   the conversion does not usually guarantee to yield results that mimic the
   original version perfectly. Luckily, we don't want a perfect conversion, we
   want a better one.

   We found Apache POI library very helpful in this part of the implementation.
   The POI Library --- formerly ''Poor Obfuscation Implementation''
   \citep{sundaram-1} --- is a Java library that provides an API to access
   Microsoft document formats. The most mature (and most popular) part of it is
   HSSF, which stands for Horrible SpreadSheet Format, and which is used by
   Java developers worldwide to access Microsoft Excel spreadsheet data, as
   well as export data into Excel spreadsheets.

   For our purposes, we relied on HSLF (''Horrible SLideshow Format''), which
   gave us access to a \ppt presentation's contents in many ways. We could
   access all images at once, or every bit of text from the whole presentation,
   but the most interesting to us was the ability to access contents on a
   per-slide basis. This allowed us to loop over the presentation's slides,
   converting them one by one, by placing the contents of each slide in a \mxp
   slide equivalent.

   That was sadly not the end of it. While HSLF does give us access to all the
   text in a presentation, or per slide, it does not distinguish between
   'normal' text and bullet lists, for example. This was a difference we had to
   detect ourselves somehow. TODO find out and explain how we do this.

   Another challenge was dealing with animations and other ways people managed
   to put way more content on one slide than would be advisable. The animations
   could not be transferred to \mxp since \mxp has its own set of transitions.
   It would technically be possible to implement additional animations as a
   separate plugin for \mxp to provide the equivalents of the animations in
   \ppt, but that is beyond the scope of this thesis. So we could not provide
   the same animations, but some people use those animations not just to show
   off but to actually show multiple pictures and blocks of text, one after the
   other, on the same slide. Without animations, this content would either not
   be visible or it would become a serious layout issue in \mxp. Our solution
   proposes to limit the amount of objects one slide can contain, and any
   additional content should be put on extra slides automatically. A downside
   of this is that we currently have no way of guessing the correct order in
   which the content should appear, so what may have been an intrinsic
   choreography of pictures in \ppt may become an incoherent jumble of images
   in \mxp. Another solution would be to scale all content until it all fits
   next to each other on one slide, and then rely on the zoomable interface to
   show the pictures one by one, but in this case the same problem with order
   of appearance manifests itself. In the end, we decided it would be best to
   accept that no conversion algorithm is going to be perfect, and the author
   can always manually change the order around after the conversion is done.

  \section{Generating \mxp}

   TODO generating

   \subsection{Plain HTML5}

    Since the \mxp compiler was not functional during most of this thesis'
    implementation, we decided to generate an html file much like the \mxp
    compiler would, including the \mxp JavaScript library and plugins. This
    required us to first find out how \mxp works on the inside, which proved to
    be a steep learning curve but gave us more insight into the software than
    we would've gotten if we only had to generate \mxp XML and leave the rest
    to the compiler.

   \subsection{\mxp XML}

    TODO XML

  \section{Creating layouts}

   TODO layout

   \subsection{Using constraints}

    TODO constraints

   \subsection{Other ways}

    TODO other ways



 % vim:ts=1:et:nospell:spelllang=en_gb:ft=tex

 \chapter{Conclusions and Future Work}

  \ppt* remains the most popular presentation tool worldwide. We believe this
  is mostly because people are generally afraid of change, and would rather
  stick to their habits. Since all of their existing work is stored in the \ppt
  format, they keep using \ppt to access that content as well as create new
  presentations. On top of that, the process of creating a presentation is
  still heavily burdened by the layout of the content we want to present.
  Layout is an important and vastly underrated aspect of presentations in
  general, which uses up an astonishing amount of time during the creation of a
  presentation.

  \section{Contribution}

   We proposed an approach for converting existing \ppt presentations into \mxp
   presentations, along with a way for the author to relinquish control over
   layout to the computer and improve upon flawed human design by
   programmatically calculating ideal content placement and size. We delivered
   a proof of concept implementation that puts this approach into practice,
   first letting us show a \ppt presentation's content in \mxp, then showing us
   the possibility of applying an automated layout algorithm to that an any
   other \mxp content at will.

   Considering the first part of this thesis, which consists of the conversion
   between \ppt and \mxp, we can conclude that conversion from any other
   presentation format into \mxp is a feasible concept. Closed-source formats
   will obviously be more of a challenge than their open-sourced cousins,
   especially if no API has been created for them as we had the fortune with
   \ppt and Apache's POI/HSLF implementation. That said, open-source formats
   may be more easy to take apart but if no API exists for them it would still
   require a substantial amount of effort. Having an existing API readily
   available has definitely helped us a great deal in our efforts.

   Converting these other formats into \mxp remains an important goal in the
   endeavour to raise awareness and increase popularity of \mxp's features and
   possibilities. Our implementation provides a way to convert \ppt\ .ppt
   presentations, but \ppt* has switched to using the Office Open XML-based
   \ .pptx format in recent years, so newer \ppt presentations cannot currently
   be imported into the \mxp system.

   Our implementation is written in a way that should make it straightforward
   to adapt for other formats, provided there is a way to get the separate
   components out of those formats. If slide-based conversion is desired, then
   obviously a way to extract the components on a per-slide basis is also
   required. Additionally, our approach using an intermediary form during the
   conversion process allows for adaptation of the tool to generate other
   output formats as well. The current implementation generates HTML5 which
   includes the \mxp standard library and a set of plugins, but the original
   goal of generating \mxp XML files should be easily attainable; the only
   reason we did not implement this was the unavailability of the \mxp
   compiler, which made it impossible for us to test the generated XML files.

   As for the second part of this thesis, concerning the automated generation
   of presentation layouts, we have discussed why this is necessary. When
   creating traditional slideware, as well as using more advanced and modern
   presentation tools, layout remains a problem that for many presenters
   becomes the biggest timesink in their work. On top of that, the layout they
   create is not always a good one, and bad layout has been proven to have
   negative impact on the effectiveness of a presentation. As such, having a
   way to automatically generate a layout would save a lot of time while also
   improving the information transfer effectiveness of presentations.

   We have demonstrated such an automated layout mechanism based on theory and
   research found in related works, which we adapted and improved upon for our
   purposes. Our constraint-based approach considers every component
   separately, to combine all components into a layout where no overlapping
   exists, components can be grouped together, clear margins are put in between
   content and surrounding limits in the form of slides and other fixed-size
   containers are respected.

   In the future, the implementation of this layout mechanism may still be
   improved upon in several ways, which will be discussed in section
   \ref{future}. However, it currently does provide the most basic form of automated
   layout, which may not always succeed in generating an aesthetically pleasing
   layout but at least attempts to combine content in a way that makes the
   content easy to focus on, thus increasing effectiveness of the presentation.
   It also succeeds in letting the presenter focus on the content rather than
   the layout. It thus reduces stress and arguably increases quality of
   presentations, especially when we look at time spent creating the
   presentation versus its effectiveness.

  \section{Future Work}
   \label{future}

   In this thesis we have presented a proof-of-concept implementation of both a
   tool to convert \ppt presentations into \mxp, and an algorithm for
   generating an objectively effective layout. Due to the limited time
   available for this thesis, we were not able to go into the finer details of
   these tools, and the results of our tools could still be improved. However, within this
   limited timeframe we did deliver a solid core containing the most important
   features, in a way that allows future research to improve upon it and easily
   add any missing details.

   \subsection{Other formats}

    Our conversion tool currently allows to convert \ppt\ .ppt files into \mxp
    presentations, bypassing the \mxp compiler. The tool internally uses an
    intermediary structure to store the presentation's content, and this
    facilitates the implementation of conversion tools for other formats. As
    such, it might be a good idea to extend the tool to convert other popular
    formats like \ppt\ .pptx files, Apple Keynote presentations and many others.

    It would also be a good idea to change the output of the conversion tool to
    generate \mxp XML files to compile further using the \mxp compiler. In and
    of itself this would not seem advantageous, but with \mxp IDEs and other
    editing tools in mind it would be better to have XML files which would be
    editable using those tools, rather than raw HTML5 which presumably would
    not be readily available in any IDE.

   \subsection{Integration}

    Speaking of editors, it would be interesting to integrate the conversion
    tool into such an editor. This would allow \mxp users to just open their
    \ppt files in the \mxp editor, immediately providing access to its contents
    and letting the user edit the presentation as if it had always been an \mxp
    presentation. This would greatly improve usability of the conversion tool
    as well, since it currently does not have a graphical user interface and
    thus needs to be invoked from the command line.

   \subsection{Improving the automated layout}

    There are many ways in which the automated layout algorithm may yet be
    improved. Jock Mackinlay's work \citep{mackinlay-1} includes significant
    research on how to use artificial intelligence to create effective
    graphical visualisations. Combining his work with ours could potentially
    improve the results of our algorithm. An interesting angle here might be
    the use of a learning AI, which can be trained on sets of good and bad
    layouts, or observe the user's actions and try to mimic their behaviour.



 \newpage

 \bibliographystyle{IEEEtranN}
 \bibliography{db}

\end{document}

