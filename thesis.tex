% vim:ts=1:et

\documentclass[a4paper,12pt]{report}
%\documentclass[a4paper,12pt]{book}

% The following makes latex use nicer postscript fonts.
%\usepackage{times}
\usepackage[english]{babel}
\usepackage{subfig}
\usepackage[usenames]{color}
\usepackage{multicol}
%\usepackage[colorlinks,urlcolor=blue,linkcolor=blue]{hyperref}
\usepackage[%ps2pdf,
            bookmarks=true,
            bookmarksnumbered=false,
            bookmarksopen=false,
            colorlinks=true,
            linkcolor=webred]{hyperref}
\definecolor{webgreen}{rgb}{0, 0.5, 0} % less intense green
\definecolor{webblue}{rgb}{0, 0, 0.5} % less intense blue
\definecolor{webred}{rgb}{0.5, 0, 0} % less intense red
\usepackage[round,comma,authoryear]{natbib}

%\hyphenation{administrative argument arguments assignments complex evaluates functions happening however machine understand unreliable variable variables whenever}

\usepackage{vubtitlepage}
\author{Joris Vandermeersch}
\title{Content Migration and Layout for the \mxp Presentation Tool}

%\promotortitle{Promotor/Promotors}
\promotor{Prof. Dr. Beat Signer}
\advisors{Reinout Roels}
\advisortitle{Begeleider}
\faculty{Faculteit Wetenschappen}
\department{Departement Informatica\\en Toegepaste Informatica}
\reason{Proefschrift ingediend met het oog op het behalen\\van de graad
van Master in de Toegepaste Informatica}
\date{Mei 2015}

% vim:ts=2:et:spelllang=en

\usepackage{xspace}

\newcommand\code{\texttt}
\newcommand\ppt{Microsoft PowerPoint\xspace}
\newcommand\mxp{MindXpres\xspace}
\newcommand\latex{\LaTeX\xspace}



\begin{document}

 % First dutch TitlePage
 \maketitlepage

 \faculty{Faculty of Science}
 \advisortitle{Advisor}
 \department{Department of Computer Science\\ and Applied Computer Science}
 \reason{Graduation thesis submitted in partial fulfillment of the\\
              requirements for the degree of Master in Applied Computer Science}

 \date{May 2015}

 % Then english TitlePage
 \maketitlepage

 \chapter*{Abstract}

  \ppt continues to be used worldwide in staggering numbers. We try to provide
  an alternative with \mxp, facilitating the switch by converting existing \ppt
  presentations into \mxp presentations, and automatically fixing the layout in
  the process.

 \chapter*{Acknowledgements}

  \emph{``Simplicity is a great virtue,\\
  but it requires hard work to achieve it and education to appreciate it.\\
  And to make matters worse: complexity sells better.''}

  \hfill\emph{--- Edsger W. Dijkstra}

%   I'd like to thank:
%  \begin{itemize}
%   \item Tania, my fianc\'ee, \emph{for all of the patience and sex she continues to have with me.}
%   \item Peter, my brother, \emph{for helping me realize life should be fun.}
%   \item Reinout, my advisor and friend, \emph{for giving me the opportunity to get this over with.}
%   \item my friends \emph{for frequently providing the necessary distraction from this awful ordeal.}
%   \item my grandparents \emph{for the moral and financial support which gave me the opportunity to start and finish this, even though I'd have liked to give up ages ago.}
%   \item my parents \emph{for the genes, the upbringing and the moral support, even if they generally have no clue what I'm talking about.}
%   \item my employers and colleagues at Roots Software \emph{for putting up with my quirks, my seemingly never-ending studies and all of the inconveniences it has brought along through the years, as well as for the steady paycheck that allows me to have a life in which the academic world has nor needs a place.}
%   \item all professors, assistants and other academic personnel at the VUB \emph{for making me realize I never want to be a part of their world. Seriously.}
%  \end{itemize}

 \tableofcontents

% TODO
%14:52 <reinout> thesis is wel een ok begin, but it all needs a lot of padding :p
%14:52 <reinout> bijvoorbeeld, you should spend ~5pg on mindxpres
%14:52 <reinout> waarom bestaat het, hoe werk dat plug-in gedoe, hoe werkt dat xml gedoe
%14:55 <reinout> basically alles wat in deze paper staat:
%14:55 <reinout> https://www.academia.edu/4186970/An_Extensible_Presentation_Tool_for_Flexible_Human-Information_Interaction
%14:55 <reinout> nee wacht
%14:55 <reinout> deze:
%14:56 <reinout> https://www.academia.edu/7719770/MindXpres_An_Extensible_Content-driven_Cross-Media_Presentation_Platform
%14:56 <reinout> explain ALL the MindXpres
%14:57 <reinout> en dan kan je verdergaan, "mensen hebben nu hun content, maar nu willen we dat ze die content in MindXpres kunnen gebruiken"

 \chapter{Introduction}

  For over 25 years, \ppt has been the market leader in digital prsentations.
  Admittedly, it was a revolutionary software package when it was first
  introduced, and its ease-of-use combined with its supreme graphical
  capabilities -- at least compared to other software in the same era --
  quickly made it one of the most popular software packages in history. 25
  years later, \ppt can claim over 90\% market share in presentation software,
  and on average 30 million \ppt presentations are created every day.

  In this time, \ppt has gotten many new features, and certainly improved and
  grew with every new version, but it never really changed its core approach.
  It started out mimicking the then-popular and widespread use of dia and
  overhead projection slides, which was at the time a good way to convince
  people of its purpose, allowing them to feel comfortable with a familiar
  format instead of alienating potential customers with a new and potentially
  confusing interface.

  However, this interface is quite restricting, and in recent years different
  approaches have seen the light of day. The zoomable user interface of Prezi
  is probably the most well-known, but apart from abandoning the traditional
  slide format it does little to improve or extend the concept of presenting
  information to an audience.
  
  This is where \mxp comes in. Its extensible plugin system allows anyone with
  some knowledge of programming to create new functionality to use in
  presentations. Examples are interactivity with the audiencer through various
  means, controlling the presentation from another device -- or several! -- and
  (re)modelling data while presenting it, based on feedback from the audience.

  While this is obviously a big improvement on the traditional presentation
  model of \ppt and the likes, it remains hard to convince the general public
  of its merits. People are generally afraid of change, and it is important to
  make the transition as smooth as possible. On top of that, people are often
  worried that the work they did in the past may be lost -- or worse,
  irrelevant -- after switching to something new. This alone may be a huge
  factor in deciding wether or not to start using new software, or to stick
  with what they know.

  That is where the subject of this thesis comes in. We aim to provide a way
  for people to convert their existing \ppt presentations into \mxp
  presentations, allowing them to take their previous work with them in their
  switch to \mxp. This way, we lower the treshold for them to make the decision
  to start using \mxp as their presentation software of choice.  Once all their
  existing \ppt content is available, usable and editable in \mxp, it should be
  obvious to anyone why \mxp is the better option for their presentations.

  Another common problem with \ppt presentations is the way they look. This is
  not necessarily the fault of the software; most people just are not trained
  in graphical design, and as such they know very little about proper layout,
  color choices, or slide content limits. Everyone has probably encountered
  slides with full paragraphs of text, too small to read and / or too much to
  process in the short time the slide is visible --- (too) many people have
  made those slides themselves.

  When we say this is not the fault of the software, that is mostly true, as
  the creators of these slides obviously made a conscious choice to make their
  content appear like that. It could be said however that \ppt and other
  presentation tools are guilty through inaction. We believe it is possible to
  have software either warn its users against these choices and practices, or
  -- even better -- have the software fix these problems automatically.

  One of the primary goals of \mxp is to provide automatic layout, much like
  \latex does, ensuring that the content creator only has to worry about the
  actual content, while the software takes care of layout. In practice, both
  \latex and \mxp currently use template-based layouts, where the contents'
  position is predefined in the template and not related to or based on its
  size, shape or nature. In the end, everyone who has ever used \latex knows
  that sooner or later you will struggle to get a certain image incorporated in
  the text correctly, ending up doing the layout yourself anyway, because the
  predefined template just doesn't work properly for you specific content.

  As such, the second part of this thesis focuses on implementing true
  automatic layout in \mxp. Again primarily to convince \ppt users to switch,
  showing that their presentations actually could look better in \mxp, while
  thus also providing new functionality to existing \mxp users.

 \chapter{Problem statement}

  \section{Terminology}

  \section{\ppt}

   \ppt was officially released in 1990, with Windows 3.0 \citep{austin-1}. It
   had originally been developed as Presenter, but trademark issues caused a
   name change early on. It was also originally build for the Macintosh, which
   may seem surprising nowadays but was actually common practice back then
   since the Macintosh was widely regarded as a better development environment,
   more mature, more stable and capable of far better performance and
   visualisations. Some may argue this still rings true today.

   Since then, it has grown to be the world's most popular slide show
   presentation program, alledgedly having been installed on over 1 billion
   computers worldwide, and being used on average 350 times \emph{per second}
   \citep{parks-1}. In 2012, it had a market share of 95\%, leaving the other
   5\% to be shared by alternatives such as Apple's Keynote, Prezi, SlideRocket
   and others. While this number is declining, it may not be going as fast as
   many people think. As most readers of this thesis have heard before, over 30
   million \ppt presentations are created every day, for all kinds of purposes,
   with good and bad results both presentation-wise and goal-wise.

   TODO this probably needs more content.

  \section{\mxp}

   \emph{This chapter's content is largely based on ``MindXpres: An Extensible
   Content-driven Cross-Media Presentation Platform'' \citep{roels-1}.}

   \subsection{Introduction}

% -- BEGIN Reinout's awesome stuff
    The importance of digital presentations in this day and age cannot be
    understated. Millions of presentations are created every day, supporting
    the oral transfer of knowledge and playing an important role in educational
    settings. Their origins as tools for creating physical media such as
    photographic slides or transparencies for overhead projectors are still
    reflected in the underlying concepts and principles of slide-based
    presentation tools. The rectangular boundaries of a slide, and the linear
    navigation between slides, are still restrictions we face today in digital
    presentations. Tufte argues that these concepts of slideware have a
    negative impact on the effectiveness of knowledge transfer \citep{tufte-1}.
    While the presenter is compelled to squeeze complex ideas into a linear
    sequence of slides, those ideas are rarely sequential by nature, resulting
    in a loss of relations, overview and details. An initial approach to
    address these issues might involve creating minimalistic presentations or
    introducing some structure via a table of contents.
%## TODO ##
    Unfortunately, this
    does not work in the domain of learning, where complex knowledge or other
    pieces of rich information need to be presented “as is” \citep{farkas-1}.

    It is important to point out the monolithic nature of slideware
    presentations where content is spread over many self-contained presentation
    files. In order to “reuse” previous work, the presenter has to switch
    between files while giving a presentation or duplicate some slides in the
    new presentation. Note that the issue is not limited to reusing single
    slides since there is a wealth of resources available, spread over a wide
    spectrum of distribution channels and formats. The inclusion of content by
    reference or transclusion \citep{nelson-1} might help to cross the
    boundaries between different types of media and be beneficial in the
    context of modern cross-media presentation tools.

    There also seems to be an imbalance between the functionality for the
    authoring and visualisation of content. The main authoring views consist of
    toolbars and buttons to specify how content should be visualised while
    there is less support for the authoring of the content itself. While we
    have seen the addition of basic multimedia types such as videos to modern
    slideware, most content is still rather static. During a presentation we
    can, for example, not easily change from a bar chart to a pie chart data
    visualisation or dynamically change some values to see the immediate
    effect, which could be beneficial for knowledge transfer
    \citep{holzinger-1}. Finally, the audience can be more actively involved
    via audience response and classroom connectivity systems which provide
    multi-device interfaces for sharing knowledge and results during as well as
    after a presentation. The evolution of presentations can be compared with
    the Web 2.0 movements where users have become contributors, content is more
    dynamic and interactive and where we have a decentralisation of content via
    service-oriented architectures.

    The rapid prototyping and evaluation of new concepts for the
    representation, visualisation and interaction with content is essential in
    order to move a step towards the next generation of cross-media
    presentation tools. After introducing existing slideware solutions, we
    discuss the requirements for next generation presentation tools. This is
    followed by a description of the extensible MindXpres architecture and its
    plug-in mechanism. The web technology-based implementation of MindXpres is
    validated based on a number of use cases and MindXpres plug-ins and
    followed by a discussion of future work.

% -- END Reinout's awesome stuff

% -- BEGIN my crappy stuff
   \mxp tries to be an alternative unlike other alternatives, stepping away
   from the classic slide format and introducing a plugin system to allow
   literally anything you can think of. The obvious choices are visualisations,
   animations, layout and the embedding of various media such as video, but it
   can also provide interaction with the audience, allow different presenters
   to take control from their own device, arbitrarily and ad-hoc change the way
   data is presented allowing to play into the audience's reactions unhindered
   by design decisions made beforehand, and so many other novel ways of
   presenting that captivate the audience instead of lulling them to sleep.

   \mxp was first created by my advisor, Reinout Roels, in 2012 as part of his
   Master thesis, and has since been built upon by himself and other students.
% -- END my crappy stuff

 \chapter{Approach}

  \section{Compilation process}

   The first part of the approach is fairly straightforward in its basic
   explanation: we had to convert \ppt presentations into \mxp presentations.
   This involves finding out how \ppt presentations are structured, getting the
   parts wee need out of that structure, and then putting those parts together
   in de \mxp structure. Since the author of this thesis has a small background
   in compilers \citep{vandermeersch-1} it did not take long to see the
   resemblance of this process to that of a compiler.
   
   A compiler takes source code and transforms it into a working program with
   the semantics described by that source code. The compilation process
   consists of several steps. First the source code is tokenized, which means
   the symbols in the code are identified one by one and classified in certain
   categories.
   
   Then the tokens are processed by a parser into an intermediary form called a
   parse tree. A parser looks for certain predefined patterns in the source
   code. These patterns are part of the source code's language syntax. As such,
   these two steps analyse and validate the source code's syntax. If part of
   the code does not match any pattern, the parser and the compilation process
   stop and the user gets a message saying the code's syntax is invalid.

   When a parse tree is constructed, the compilation process can alter it, to
   improve it. Certain patterns in the parse tree may be replaceable by
   different patterns with the same outcome, but with more optimal execution.
   This part of the compilation process is optional, and is called compiler
   optimization. Optimizations can consist of many things, depending on the
   language. For example, some languages guarantee tail call optimization,
   where infinite loops can be constructed by letting a function call itself as
   its last statement without causing a stack overflow. This is something the
   compiler (or interpreter) can optimize during this part of the compilation
   process.

   After this, the parse tree can be written out to produce the desired output.
   Every node in the tree has a well-defined equivalent in the target
   language's syntax. The target language can be Assembly, which consists of
   the exact instructions a CPU needs to carry out a program, or it can be
   another programming language. Many compilers of higher-level languages
   translate their language into C, for several reasons: the C compilers that
   translate C into Assembly have been optimized so much that it is easier to
   rely on them than to put an enormous amount of effort into optimizing
   another language; C compilers exist for most -- if not all -- CPU
   architectures, which means translating a language into C makes it compatible
   with all those architectures, while it would cost a lot more effort to write
   different compilers for every architecture you would want to make your
   language available on.

   The conversion tool that is the purpose of this thesis, can be described in
   a similar succession of steps. First, we take a \ppt presentation and
   tokenize and parse it into an intermediary structure that allows us to
   perform other operations, or `optimizations', on it. The intermediary form
   consists of a `parse tree' containing the components of the original
   presentation --- a component tree, if you will.

   With this structure, we can construct a \mxp presentation containing the
   same components in the same place, essentially creating a `program' with the
   same semantic meaning as the original `source'.

  \section{Compiler optimizations}

   Since the conversion process resembles that of a compiler, it seemed logical
   at first to make automatic layout a part of that process, as some kind of
   `compiler optimization'.

   At first, we tried to traverse the component tree, giving its objects new
   coordinates and sizes so that they would fit together on every slide as well
   as possible. This seemed an easy solution, but the results were sub-optimal.
   On top of that, we soon realised that we were in essence creating another
   template out of which a presentation would be made, which was exactly the
   opposite of what we were trying to do. As such, we abandoned this approach.
   
   We then switched to a different method: defining constraints for every
   component, in the form of margins, maximum sizes and other limits, and then
   calculating a way to satisfy all constraints while fitting content together
   on each slide. While this is clearly a better method, it turned out the
   compiler optimization phase was not the best place in the process to take
   care of this.
   
   In the end, we decided to take a different approach, relying on the layout
   engine of \mxp itself and enhancing that engine to create the automatic
   layout we wanteD.

  \section{Using \mxp}

   \mxp already takes care of layout for you, since that is one of its primary
   goals. The way it does this currently is however heavily based on templates,
   while we wanted a layout engine that could take any content and put in in an
   appropriate layout without any directions from the user. As such, we had to
   enhance \mxp's layout engine to use constraints, based on the size of the
   content, and try to find an optimal position for every component it is
   given.

 \chapter{Implementation}

  \section{Taking \ppt apart}
   
   We found Apache POI library very helpful in this part of the implementation.
   The POI Library -- formerly ''Poor Obfuscation Implementation''
   \citep{sundaram-1} -- is a Java library that provides an API to access
   Microsoft document formats. The most mature (and most popular) part of it is
   HSSF, which stands for Horrible SpreadSheet Format, and which is used by
   Java developers worldwide to access Microsoft Excel spreadsheet data. 

   For our purposes, we relied on HSLF (''Horrible SLideshow Format''), which
   gave us access to a \ppt presentation's contents in many ways. We could
   access all images at once, or every bit of text from the whole presentation,
   but the most interesting to us was the ability to access contents on a
   per-slide basis.

   This allowed us to loop over the presentation's slides, converting them one
   by one, by placing the contents of each slide in a \mxp slide equivalent.

  \section{Generating \mxp}

   \subsection{Plain HTML5}

    Since the \mxp compiler was not functional during most of this thesis'
    implementation, we decided to generate an html file much like the \mxp
    compiler would, including the \mxp JavaScript library and plugins. This
    required us to first find out how \mxp works on the inside, which proved to
    be a steep learning curve but gave us more insight into the software than
    we would've gotten if we only had to generate \mxp XML and leave the rest
    to the compiler.

   \subsection{\mxp XML}

  \section{Creating layouts}

   \subsection{Using constraints}

   \subsection{Other ways}

 \chapter{Conclusions and Future Work}

  \section{Contribution}

  \section{Future Work}

 \newpage

 \bibliographystyle{IEEEtranN}
 \bibliography{db}

\end{document}

