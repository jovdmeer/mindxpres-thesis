% vim:ts=1:et:spelllang=en

%\documentclass[a4paper,12pt]{thesis}
\documentclass[a4paper,12pt]{report}
%\documentclass[a4paper,12pt]{book}

% The following makes latex use nicer postscript fonts.
%\usepackage{times}
\usepackage[english]{babel}
\usepackage{graphicx}
\graphicspath{ {img/} }
\usepackage[usenames]{color}
\usepackage{multicol}
%\usepackage[colorlinks,urlcolor=blue,linkcolor=blue]{hyperref}
\usepackage[%ps2pdf,
            bookmarks=true,
            bookmarksnumbered=false,
            bookmarksopen=false,
            colorlinks=false,
%            colorlinks=true,
            linkcolor=webred]{hyperref}
\definecolor{webgreen}{rgb}{0, 0.5, 0} % less intense green
\definecolor{webblue}{rgb}{0, 0, 0.5} % less intense blue
\definecolor{webred}{rgb}{0.5, 0, 0} % less intense red
\usepackage[round,comma,authoryear]{natbib}

%\hyphenation{administrative argument arguments assignments complex evaluates functions happening however machine understand unreliable variable variables whenever}

\usepackage{vubtitlepage}
\author{Joris Vandermeersch}
\title{Content Migration and Layout for the \mxp Presentation Tool}

%\promotortitle{Promotor/Promotors}
\promotor{Prof. Dr. Beat Signer}
\advisors{Reinout Roels}
\advisortitle{Begeleider}
\faculty{Faculteit Wetenschappen}
\department{Departement Informatica\\
            en Toegepaste Informatica}
\reason{Proefschrift ingediend met het oog op het behalen\\
        van de graad van Master in de Toegepaste Informatica}
\date{Mei 2015}

% vim:ts=2:et:spelllang=en

\usepackage{xspace}

\newcommand\code{\texttt}
\newcommand\ppt{Microsoft PowerPoint\xspace}
\newcommand\mxp{MindXpres\xspace}
\newcommand\latex{\LaTeX\xspace}



\begin{document}

% First dutch TitlePage
 \maketitlepage

 \faculty{Faculty of Science}
 \advisortitle{Advisor}
 \department{Department of Computer Science\\
             and Applied Computer Science}
 \reason{Graduation thesis submitted in partial fulfillment of the\\
         requirements for the degree of Master in Applied Computer Science}

 \date{May 2015}

% Then english TitlePage
 \maketitlepage

 % vim:ts=1:et:nospell:spelllang=en_gb:ft=tex

 \chapter*{Abstract}

  \ppt* continues to be used worldwide in staggering numbers. We try to provide
  an alternative with \mxp, facilitating the switch by converting existing \ppt
  presentations into \mxp presentations, and automatically fixing the layout in
  the process.

  TODO judging by other thesises, this should be longer



 % vim:ts=1:et:nospell:spelllang=en_gb:ft=tex

 \chapter*{Acknowledgements}

  \emph{\makebox[0pt][r]{``}Simplicity is a great virtue,\\
  but it requires hard work to achieve it and education to appreciate it.\\
  And to make matters worse: complexity sells better.''}

  \hfill\emph{--- Edsger W. Dijkstra}\\

%%% TODO thank people

%  First of all, I'd like to thank my fianc\'ee, Tania, for her love, for her
%  support, and for her patience. This thesis has been the last hurdle of a race
%  I've been running for way too long, for which I've had to put both of our
%  lives on hold way too often. It's been stressful, and I'm sure she's looking
%  forward to the load being lifted as much as I am.

%  02:24 <reinout> ik eis een dodentocht vermelding
%  02:25 <reinout> "and I would like to expres my hatred towards my supervisor reinout roels for dragging me along for the worst 20km of my life"
%  Reinout, my advisor and friend, deserves my eternal gratitude for giving me
%  the chance to get this over with, and dragging me through it as well. I
%  could never have finished this without his help, his guidance and his
%  enthousiasm.

%  This section would not be complete without mentioning Peter, my brother, who
%  showed me the fun side of studying, the more significant parts of life, and
%  the importance of procrastination. I'm always proud to call myself his big
%  brother.

%  This thesis was further brought to you with the support of many other people,
%  all of whom really deserve more than just being mentioned in my
%  acknowledgements section, and if they ever need any help they'll know to find
%  me:
%  \begin{itemize}
%   \item My friends, for frequently providing the necessary distraction from this awful ordeal, as well as for giving me many more fun distractions to look forward to whenever there seemed to be no end to this.
%   \item My grandparents, for the moral and financial support which gave me the opportunity to start and finish this, even though I wanted to give up ages ago.
%   \item My parents, for the genes, the upbringing and the moral support, even if they often have no clue what I'm talking about.
%   \item Infected Mushroom, for providing the soundtrack to many late-night caffeine-fueled hack- and write sessions.
%   \item RedBull and Nalu, for providing the fuel for those same sessions.
%   \item My colleagues at Roots Software, for putting up with my quirks, but more importantly for showing me the world beyond academics.
%   \item All professors, assistants and other academic personnel at the VUB, for making me realize a long time ago that I never want to be a part of the academic world. Seriously.
%  \end{itemize}



 \tableofcontents

% TODO
%14:52 <reinout> thesis is wel een ok begin, but it all needs a lot of padding :p
%14:52 <reinout> bijvoorbeeld, you should spend ~5pg on mindxpres
%14:52 <reinout> waarom bestaat het, hoe werk dat plug-in gedoe, hoe werkt dat xml gedoe
%14:55 <reinout> basically alles wat in deze paper staat:
%14:55 <reinout> https://www.academia.edu/4186970/An_Extensible_Presentation_Tool_for_Flexible_Human-Information_Interaction
%14:55 <reinout> nee wacht
%14:55 <reinout> deze:
%14:56 <reinout> https://www.academia.edu/7719770/MindXpres_An_Extensible_Content-driven_Cross-Media_Presentation_Platform
%14:56 <reinout> explain ALL the MindXpres
%14:57 <reinout> en dan kan je verdergaan, "mensen hebben nu hun content, maar nu willen we dat ze die content in MindXpres kunnen gebruiken"

 % vim:ts=1:et:nospell:spelllang=en_gb:ft=tex

 \chapter{Introduction}

  For over 25 years, \ppt* has been the market leader in digital prsentations.
  Admittedly, it was a revolutionary software package when it was first
  introduced, and its ease-of-use combined with its supreme graphical
  capabilities --- at least compared to other software in the same era --
  quickly made it one of the most popular software packages in history. 25
  years later, \ppt* can claim over 90\% market share in presentation software,
  and on average 30 million \ppt presentations are created every day.

  In this time, \ppt* has gotten many new features, and certainly improved and
  grew with every new version, but it never really changed its core approach.
  It started out mimicking the then-popular and widespread use of dia and
  overhead projection slides, which was at the time a good way to convince
  people of its purpose, allowing them to feel comfortable with a familiar
  format instead of alienating potential customers with a new and potentially
  confusing interface.

  However, this interface is quite restricting, and in recent years different
  approaches have seen the light of day. The zoomable user interface of Prezi
  is probably the most well-known, but apart from abandoning the traditional
  slide format it does little to improve or extend the concept of presenting
  information to an audience.

  This is where \mxp comes in. Its extensible plugin system allows anyone with
  some knowledge of programming to create new functionality to use in
  presentations. Examples are interactivity with the audiencer through various
  means, controlling the presentation from another device --- or several! ---
  and (re)modelling data while presenting it, based on feedback from the
  audience.

  While this is obviously a big improvement on the traditional presentation
  model of \ppt* and the likes, it remains hard to convince the general public
  of its merits. People are generally afraid of change, and it is important to
  make the transition as smooth as possible. On top of that, people are often
  worried that the work they did in the past may be lost --- or worse,
  irrelevant --- after switching to something new. This alone may be a huge
  factor in deciding wether or not to start using new software, or to stick
  with what they know.

  That is where the subject of this thesis comes in. We aim to provide a way
  for people to convert their existing \ppt presentations into \mxp
  presentations, allowing them to take their previous work with them in their
  switch to \mxp. This way, we lower the treshold for them to make the decision
  to start using \mxp as their presentation software of choice. Once all their
  existing \ppt content is available, usable and editable in \mxp, it should be
  obvious to anyone why \mxp is the better option for their presentations.

  TODO argue that automatic layout helps liberate the content from the confines
  of slides, not just fix bad layout

  Another common problem with \ppt presentations is the way they look. This is
  not necessarily the fault of the software; most people just are not trained
  in graphical design, and as such they know very little about proper layout,
  color choices, or slide content limits. Everyone has probably encountered
  slides with full paragraphs of text, too small to read and / or too much to
  process in the short time the slide is visible --- (too) many people have
  made those slides themselves.

  When we say this is not the fault of the software, that is mostly true, as
  the creators of these slides obviously made a conscious choice to make their
  content appear like that. It could be said however that \ppt* and other
  presentation tools are guilty through inaction. We believe it is possible to
  have software either warn its users against these choices and practices, or
  --- even better --- have the software fix these problems automatically.

  One of the primary goals of \mxp is to provide automatic layout, much like
  \latex does, ensuring that the content creator only has to worry about the
  actual content, while the software takes care of layout. In practice, both
  \latex and \mxp currently use template-based layouts, where the contents'
  position is predefined in the template and not related to or based on its
  size, shape or nature. In the end, everyone who has ever used \latex knows
  that sooner or later you will struggle to get a certain image incorporated in
  the text correctly, ending up doing the layout yourself anyway, because the
  predefined template just doesn't work properly for your specific content.

  As such, the second part of this thesis focuses on implementing true
  automatic layout in \mxp. Again primarily to convince \ppt* users to switch,
  showing that their presentations actually could look better in \mxp, while
  thus also providing new functionality to existing \mxp users.

  TODO moar?


 % vim:ts=1:et:nospell:spelllang=en_gb:ft=tex

 \chapter{Related work}

  \emph{This chapter's content is largely based on ``MindXpres: An Extensible
  Content-driven Cross-Media Presentation Platform'' \citep{roels-1}.}

  \section{Background}

   The importance of digital presentations in this day and age cannot be
   understated. Millions of presentations are created every day, supporting the
   oral transfer of knowledge and playing an important role in educational
   settings. Their origins as tools for creating physical media such as
   photographic slides or transparencies for overhead projectors are still
   reflected in the underlying concepts and principles of slide-based
   presentation tools. The rectangular boundaries of a slide, and the linear
   navigation between slides, are still restrictions we face today in digital
   presentations. Tufte argues that these concepts of slideware have a negative
   impact on the effectiveness of knowledge transfer \citep{tufte-1}. While
   the presenter is compelled to squeeze complex ideas into a linear sequence
   of slides, those ideas are rarely sequential by nature, resulting in a loss
   of relations, overview and details. An initial approach to address these
   issues might involve creating minimalistic presentations or introducing some
   structure via a table of contents. Sadly, when complex knowledge or other
   pieces of rich information need to be presented “as is” \citep{farkas-1} ---
   as in the domain of learning --- this does not work.

   One of the main issues with traditional slideware presentations is their
   monolithic nature, especially when content is spread over many
   self-contained presentation files. ``Reusing'' previous work involves either
   switching between files while giving a presentation or duplicating some
   slides in the new presentation. It should be noted that this issue is not
   limited to the reuse of single slides: there is an ever increasing wealth of
   resources available for reuse, spread over a wide spectrum of distribution
   channels and formats. The possibility to include content by reference or
   transclusion \citep{nelson-1} may contribute in crossing the boundaries
   between different types of media and prove beneficial in the context of
   modern cross-media presentation tools.

   The difference in functionality between the authoring of content and its
   visualisation is striking as well. The primary editors consist of mostly
   toolbars and buttons used for selecting and specifying the way content
   should be visualised, while support for authoring the content itself is not
   quite as extensive. Modern slideware has grown to include basic multimedia
   types such as videos, but most content is still rather static. It is, for
   example, not possible during a presentation to easily switch from a bar
   chart to a pie chart data visualisation, or to dynamically change some
   values in the represented data and immediately see the effect in the graph,
   which could be beneficial for knowledge transfer \citep{holzinger-1}. The
   audience could also be more actively involved in the presentation, through
   audience response and classroom connectivity systems providing multi-device
   interfaces allowing to share knowledge and results during as well as after a
   presentation. The evolution of presentations is reminiscent of the Web2.0
   movements where users have switched roles from purely consuming content to
   contributing as well, content has become more dynamic and interactive, and
   service-oriented architectures (``The Cloud'') have ensured decentralisation
   of content.

   In order to move a step towards the next generation of cross-media
   presentation tools, it is essential to allow the rapid prototyping and
   evaluation of new concepts for the representation, visualisation and
   interaction with content.

   Before discussing the requirements for a new generation of presentation
   tools, we briefly introduce existing slideware solutions. Afterwards, we
   describe the architecture of \mxp, its extensible nature and its plug-in
   mechanism. The HTML5-based implementation of \mxp is then discussed through
   demonstration of several use cases and \mxp plug-ins.

   A specific issue with slideware we'd like to focus on in this thesis, is the
   trouble with layout in presentations. It can be hard to display the content
   you want in a way that's clear, informative and nice to look at. The vast
   majority of layouts created today is mostly done by hand: a human graphic
   designer or ``layout expert'' makes most, if not all, of the decisions about
   the position and size of the objects to be presented \citep{lok-1}. Most
   software offers some templates, allowing you to drop pictures and text into
   predefined slots and places on a slide, but then those templates have been
   defined by someone else too. Computer-generated layout is rare and usually
   not quite up to the task.

   \mxp is among the software packages offering templates, in that layout is
   handled by whichever plug-in you choose, but so far no plug-ins have defined
   dynamic layout algorithms, rather sticking to predefined ways to put text
   and pictures on slides. But as \mxp does not constrain us to the limits of
   slides, this should be seen as an opportunity to offer dynamic layout as
   well. After all, if we're not limited to a certain area within which our
   content should fit, it should be much easier to put content next to each
   other in a way that makes sense.

  \section{Existing solutions}

   Since digital slideware was first introduced, their influence, advantages
   and disadvantages have been studied extensively. There have been studies
   acknowledging the benefits of slideware as a teaching asset
   \citep{holzinger-1}, while others have been less positive. Tufte
   \citeyearpar{tufte-1} heavily criticises slideware for its infatuation with
   outdated concepts. He discusses the many ramifications of dimensional and
   structural limitations as well as linear navigation, and points out the
   discrepancy with how the human mind works. Amongst Tufte's conclusions, and
   also confirmed by Adams \citep{adams-1}, is the suggestion that slide-based
   presentations are not appropriate for every kind of knowledge transfer and
   especially not in a scientific context. Recent work shows the importance
   towards the learning process of integrating content into the bigger picture,
   both structurally and visually \citep{gross-1}, which is affected by the
   navigation and visualisation.

   Several approaches have been proposed to offer non-linear navigation.
   CounterPoint \citep{good-1}, Fly \citep{lichtschlag-1} and Prezi, provide
   Zoomable User Interfaces (ZUIs) which offer virtually unlimited space.
   Microsoft has experimented with this concept as well in pptPlex. Other
   approaches to escape the confines of the slide have been noticed, like
   MultiPresenter \citep{lanir-1} or tiling slideshows \citep{chen-1}.
   PaperPoint \citep{signer-1} and Palette \citep{nelson-2} additionally
   facilitate the non-linear navigation of digital presentations consisting of
   slide selection through augmented paper-based interfaces. Lastly, a category
   of authoring tools exists which use hypermedia to implement varying paths
   through a set of slides. NextSlidePlease \citep{spicer-1} enables users to
   define a weighted graph of slides, and tries to suggest navigational paths
   based on the link weights and the remaining presentation time. Microsoft
   cultivates this idea in their HyperSlides \citep{edge-1} project. Garcia
   \citep{garcia-1} has additionally explored the potential of \ppt* as an
   authoring tool for hypermedia-based presentations.

   \ppt* was officially released in 1990, with Windows 3.0 \citep{austin-1}. It
   had originally been developed as Presenter, but trademark issues caused a
   name change early on. It was also originally build for the Macintosh, which
   may seem surprising nowadays but was actually common practice back then
   since the Macintosh was widely regarded as a better development environment,
   more mature, more stable and capable of far better performance and
   visualisations. Some may argue this still rings true today.

   Since then, it has grown to be the world's most popular slide show
   presentation program, alledgedly having been installed on over 1 billion
   computers worldwide, and being used on average 350 times \emph{per second}
   \citep{parks-1}. In 2012, it had a market share of 95\%, leaving the other
   5\% to be shared by alternatives such as Apple's Keynote, Prezi, SlideRocket
   and others. While this number is declining, it may not be going as fast as
   many people think. As most readers of this thesis have heard before, over 30
   million \ppt presentations are created every day, for all kinds of purposes,
   with good and bad results both presentation-wise and goal-wise.

   To reuse content in existing presentation tools, that content needs to be
   duplicated, which results in a multitude of redundant copies that need to be
   kept consistent with each other: if one copy is changed, all the others must
   be changed in the same way to prevent inconsistencies and mistakes. While
   some attempts have been made to solve this problem, there is still a long
   way to go. When looking for document formats designed to server more general
   educational purposes, we find formats such as the Learning Material Markup
   Language (LMML) \citep{suss-1}, the Connexions Markup Language (CNXML) and
   the eLesson Markup Language (eLML) \citep{fisler-1}. All of these formats
   share their focus on the reuse of content, but all of them attempt this at a
   relatively high granularity level. Content can be organised in lessons or
   modules, and users are encouraged to use these, as a whole, in their
   teaching. When we investigated the formats more closely, we observed that
   outgoing links to external content were supported, but
   transclusion\footnote{The inclusion of content via references} was not. In
   relation to presentations, Microsoft's Slide Libraries exist as central
   repositories that store slides to enable slide sharing and reuse within an
   organisation. The dependency on SharePoint might represent a hurdle for some
   users, as not everyone has the ability and opportunity to set up such a
   server. A more significant issue is the fact that slides still need to be
   searched and manually copied into presentations. Keeping slides in the
   repository and in other presentations is the responsibility of the authors
   of those slides and those presentations, as no automatic update system is
   provided. SlideRocket and SlideShare are both similar tools showing
   intentions and providing functionality for content reuse. The SliDL
   \citep{canos-1} research framework works much like Microsoft's Slide
   Libraries, in that it allows for storage and tagging of slides in a database
   for reuse, but also in that it shares the same shortcomings. The ALOCOM
   \citep{verbert-1} framework aimed at flexible content reuse is built upon a
   content ontology and a (de)composition framework for legacy documents
   including \ppt documents, Wikipedia pages and SCORM content packages.
   However, ALOCOM may be too rigid for evolving presentation formats, and it
   currently only supports the authoring phase, although the tool does succeed
   in decomposing legacy documents as advertised.

   Aside from the similarities in the Web's and presentation environments'
   evolution, some of the problems mentioned in this section can find their
   solutions in the context of the Web. It should not come as a surprise then,
   that web technologies are being used more often recently in the realisation
   of presentation solutions. The Simple Standards-based Slide Show System
   (S5)\footnote{http://meyerweb.com/eric/tools/s5/} is an XHTML-based
   slideshow file format that enforces the standard slideware model. The W3C's
   Slidy \citep{raggett-1} initiative offers another presentation format based
   on the classical slideware model. Both of these formats have some valuable
   properties. They encourage a clean separation of content and visualisation
   through the use of CSS themes. The design is resolution independent, and the
   layout and font size adjust to the available screen real estate. Last but
   not least, some more recent HTML5-based presentation solutions such as
   impress.js, deck.js, Shower or reveal.js. Cross-device support is one of the
   most important advantages to leverage when using a well-known open standard
   such as HTML. However, as all of these solutions display some restrictions
   in terms of visualisation, navigation, and cross-media support, they are
   unfortunately too limited for our needs.

   The tools and projects discussed in this section mostly focus on
   distinguishing novel ideas for presentations. Nevertheless, the different
   concepts introduced in these tools don't offer interoperability between
   them. One project may focus on the authoring, another one fixates on novel
   content types and a third solution supplies radically new navigation
   mechanisms. Slideware tools may often allow third-party extensions but the
   API exposed to plug-in developers is usually limited by the software's
   underlying model. As an illustration, \ppt supports interaction from
   plug-ins with the presentation model, but the model dictates that a
   presentation consists of a sequence of slides. Many existing web-based
   presentation formats share this flaw. Because of this, we see a need for an
   open presentation platform such as \mxp to support innovation by
   contributing the necessary modularity and interoperability \citep{bush-1}.

   It is perhaps surprising that, to our knowledge, currently no tools exist to
   calculate dynamic layouts of content in slideware. Existing solutions
   include template systems, sometimes very fine-grained like \latex allowing
   you to define templates for every single layout choice, usually more coarse
   like \ppt* or Apple Keynote using Master Slides to define different layouts
   on a per-slide basis, and always with the option of letting the user
   customise the layout by hand, literally manually moving the content to the
   exact place where we want it, unhindered by style guides, good practices or
   common sense. This has resulted in mindboggling layout choices involving
   enormous amounts of tiny text crammed onto one slide, or pictures strewn
   across a slide overwhelming the audience with too much information at once.

  \section{New solutions}

   Here we propose a set of requirements to establish a wide range of
   presentation styles and visualisations. This set has been compiled based on
   a review of the more recent presentation solutions discussed in the previous
   section.

     \paragraph{Non-linear Navigation} As we mentioned before, traversing
      slides in a linear fashion is a remnant of the way early photographic
      slides worked. Over the years, people have grown used to this form of
      navigation despite the inconveniences. If the presenter unexpectedly
      needs to show anything other than the next or previous slide (e.g. to
      answer a question from the audience), they either need a considerable
      amount of time to scroll forwards or backwards, or they have to switch to
      the slide sorter view, to find the desired slide. Also troubling is the
      lack of any functionality allowing a single slide to be included multiple
      times throughout the presentation without duplicating the slide in
      question, meaning if any change has to be made to that slide the same
      change has to be performed on all copies. This poses the risk of
      overlooking some copies, introducing inconsistencies and facilitating
      mistakes. There are several manners in which this lack of flexible
      navigation might be addressed, including the possibility to define
      non-linear navigation paths \citep{spicer-1,edge-1} or zoomable user
      interfaces (ZUIs) \citep{good-1,lichtschlag-1,haller-1}.

     \paragraph{Separation of Content and Presentation} Similar to the approach
      of \latex and other professional typesetting systems, content should be
      written in a standardised way with visualisation being handled
      automatically by the typesetting system. The clear separation between
      content and presentation makes the presentation tool handle the
      visualisation, allowing the authors of a presentation to focus on the
      content. Additionally, this facilitates experimentation with different
      visualisations. \latex does have a document class called Beamer which was
      designed specifically for presentations, but while we were inspired by
      its structured and content-driven approach, the content-related
      functionality and the visualisation possibilities are too limited to be
      considered as a basis for an extensible presentation tool.

     \paragraph{Extensibility} Rapidly prototyping innovative navigation and
      visualisation techniques, but also new content types and presentation
      formats, should be easy in order for a presentation tool to be successful
      as an experimental platform for new presentation concepts. It should be
      possible to add or replace specific components without requiring changes
      in the core. A presentation tool should provide a modular architecture
      with loosely coupled components to be truly extensible. Note that this
      type of extensibility should be offered on the level of content types as
      well as for the visualisation engine and content structures.

     \paragraph{Cross-Media Content Reuse} We have previously briefly mentioned
      the lack of content reuse in existing presentation tools. Even though
      there is a wealth of open education material available, it is rather
      difficult to use this content in presentations. However, the concept of
      transclusion does work well for digital documents and parts of the Web
      (e.g. via the HTML img tag). The seamless integration of external
      cross-media content as implemented in these environments should also be
      supported by any modern presentation tool. This includes several
      different mechanisms that enable including parts of other presentations
      (e.g. slides), transcluding content from third-party document formats,
      and including content from open learning repositories.

     \paragraph{Connectivity}

      Connectivity for multi-device input and output has become more relevant
      in relation to presentation tools with the rise of social and mobile
      technologies. Multi-directional connectivity needs to be supported for
      several reasons. First, it is a requirement to enable the previously
      mentioned cross-media transclusion from external resources. Second,
      multi-directional connectivity is the basis of audience feedback via
      real-time response or voting systems \citep{dufresne-1} as well as other
      forms of multi-device interfaces.

     \paragraph{Interactivity}

      As mentioned earlier, content can be more interactive and the
      extensibility requirement addresses this issue since the intended
      architecture should support dynamic or interactive content and
      visualisations. However, traditional human input devices might not
      suffice for components offering a high level of interaction. As such, a
      presentation tool should facilitate the integration of other forms of
      input like gesture-based interaction using Microsoft's Kinect controller
      or digital pen interaction \citep{signer-2} as implemented by the
      PaperPoint \citep{signer-1} presentation tool.

     \paragraph{Post-Presentation Phase}

      Slide decks often play an important role as study or reference material,
      even if that was never their original goal. It is a trivial act to share
      traditional slide decks after a presentation, but this changes when the
      previously mentioned requirements are taken into account. The nonlinear
      navigation allows presenters to go through their content in a non-obvious
      order, and input from the audience might drive parts of a presentation,
      amongst other possible variables. Special attention should therefore be
      paid to the post-presentation phase. Playing back a presentation using
      the original navigational path, annotations and audience input should be
      made easy, while the content should also be made discoverable and
      reusable. With the rise and popularity of modern social media, there is a
      definite possibility to include the social aspect in a post-presentation
      phase through a content discussion mechanism.

  \section{\mxp Platform}
   \label{mxp-platform}

   This section presents the global architecture of the
   \mxp\footnote{http://mindxpres.com} crossmedia presentation platform as
   outlined in \figref{roels-1-fig-1}, which addresses the requirements
   presented in the previous section.

   \fig{roels-1-fig-1}{\mxp architecture}

   \subsection{Document Format and Authoring Language}

    A dedicated \mxp document format is used to store, structure and and
    reference content. Content is stored, structured and referenced in Each
    individual \mxp document contains the presentation's content itself and may
    also refer to some external content to be included. A new \mxp document can
    be constructed by hand similar to how \latex is authored, or --- in the
    near future --- it may be generated via a graphical authoring tool.
    Contrary to other presentation formats such as Slidy, S5 or OOXML, the \mxp
    authoring language abandons unnecessary HTML and XML specifics and focuses
    on a semantically more meaningful vocabulary. The syntax of the authoring
    language is almost entirely defined by plug-ins that enable the inclusion
    and visualisation of various media types and structures. To allow users
    some freedom in the way they present their information, the core \mxp
    presentation engine does little more than providing a runtime environment
    for plug-ins and lets them define the media types (e.g. video or source
    code) as well as structures (e.g. slides or graph-based content layouts).

    This also becomes apparent in the document format as every plug-in extends the
    available syntax. Any visual styling including different fonts,
    colours or backgrounds is achieved by applying specific themes to the
    underlying content.

   \subsection{Compiler}

    The compiler generates a self-contained portable \mxp presentation bundle
    based on a \mxp document. Although a \mxp document could be directly
    interpreted at visualisation time, we decided to create this intermediary
    step for a number of reasons. First, the compiler enables the creation of
    different types of presentations from the same \mxp document instance.
    This lets us not only create dynamic and interactive presentations but also
    more static output formats such as PDF documents for printing. Second, it
    is unwise to assume that there will always be an Internet connection
    available when giving a presentation. To overcome this possible issue, the
    compiler might create an offline version of a presentation with all
    necessary content pre-downloaded and included in the \mxp presentation
    bundle. Last but not least, the compiler might resolve incompatibility
    issues by, for instance, converting unsupported video formats or including
    certain HTML5 libraries.

   \subsection{\mxp Presentation Bundle}

    The dynamic \mxp presentation bundle contains the compiled content along
    with a portable cross-platform presentation runtime engine which enables
    more interactive and networked presentations. Resembling the original
    document, the compiled presentation content still consists of integrated
    content as well as references to external resources, such as online content
    that will be retrieved when the presentation is visualised. However, it
    should be noted that the content might have been modified by the compiler
    and, for example, been converted or extracted from other document formats
    that the runtime engine cannot process. References to external content may
    have been dereferenced by the compiler for offline viewing.

    A presentation bundle's core runtime engine consists of the three modules
    shown in \figref{roels-1-fig-1}. The \emph{content engine} is responsible
    for processing the content and linking it to the corresponding
    visualisation plug-ins. The \emph{graphics engine} provides all
    rendering-related functionality. For instance, some presenters may prefer a
    zoomable user interface because it provides a better overview of their
    content \citep{reuss-1}. This graphical functionality is also available to
    the plug-ins, which can make use of the provided abstractions. The
    \emph{communication engine} exposes a communication API which can also be
    used by plug-ins. It implements some basic functionality for fetching
    external content while also offering the possibility to form networks
    between multiple \mxp presentation instances as well as to connect to
    third-party hardware such as digital pens or clicker systems.

    Finally, the presentation bundle also contains a collection of
    \emph{themes} and \emph{plug-ins} as referenced by the presentation
    content. Themes may define visual styling on a global as well as on a
    plug-in level. The content engine encounters different content types and
    hands them over to the matching plug-in, which in turn uses the graphics
    engine to visualise the content.

   \subsection{Plug-in Types}

    \fig{roels-1-fig-2}{Structure plug-in examples}

    In order to attain the required flexibility, all non-core modules have been
    implemented as plug-ins. Even the most basic content types including text,
    images and bullet lists are handled through plug-ins. We distinguish
    between three major categories of plug-ins:

    \begin{itemize}

     \item \emph{Components} are the smallest elements of a presentation. The
     component plug-ins handle the visualisation for specific content types
     such as text, images, bullet lists, graphs or videos. The content engine
     invokes the corresponding plug-ins in order to visualise the content.

     \item \emph{Containers} are used to group and organise components in a
     specific way. One example of such a container is a slide, where each slide
     contains different content but also some recurring elements. For instance,
     every slide of a presentation can contain certain elements such as a
     title, a slide number and the author's name, all of which can be
     abstracted at a higher level. Another example is an image container that
     visualises its content as a horizontally scrollable list of images. It's
     important to observe that \mxp does not restrict us to the slide format
     and content can be laid out in many alternative ways.

     \item \emph{Structures} are high-level structures and layouts for
     components and containers. They may scatter content in a graph-like
     structure or they may clearly group it in sections like in a book. These
     are radically different ways of visualising and navigating content but the
     plug-in abstraction allows the user to easily switch between different
     presentation styles like the ones shown in \figref{roels-1-fig-2}.
     Structures differ from containers in that they do not restrict media types
     of their child elements in any way while they may influence the default
     navigational path through the content.

    \end{itemize}

   \subsection{Implementation}

    HTML5 and its related web technologies were chosen as the backbone for the
    \mxp presentation platform. Alternatives such as JavaFX, Flash or game
    engines were investigated as well, but HTML5 appeared to be the best
    option. The widely accepted HTML5 standard makes \mxp presentations highly
    portable, as any device with a recent web browser can display them,
    including smartphones and tablets. Moreover, HTML5 offers rich
    visualisation functionality by design and the inclusion of Cascading Style
    Sheets (CSS) and third-party JavaScript libraries makes it a powerful
    visualisation platform.

    \subsubsection{Document Format and Authoring Language}

     The \mxp document format that enables the simple expression of a
     presentation's content, structure and references is based on the
     eXtensible Markup Language (XML). \lstref{xml-1} shows a simple example
     of a presentation defined in our XML-based authoring language. The set of
     valid tags and their structure, apart from the \code{presentation} root
     tag, consists of what is provided by the available plug-ins.

     \begin{figure}[h!]
      \begin{lstxml}{xml-1}{Authoring a simple \mxp presentation}
<presentation>
  <slide title="Vannevar Bush">
    <bulletlist>
      <item>March 11, 1890 - June 28, 1974</item>
      <item>American Engineer, founder of Raytheon</item>
    </bulletlist>
    <image source="bush.jpg"/>
  </slide>
</presentation>
      \end{lstxml}
     \end{figure}

    \subsubsection{Compiler}

     The compiler has been implemented as a Node.js application. Not only does
     this accomodate the use of the compiler via a web interface or as a web
     service, but projects such as node-webkit also enable the compiler to be
     executed as a local offline desktop application. The decision to use
     server-side JavaScript was influenced by the fact that Node.js has the
     ability to bridge web and desktop technologies. On one hand, the framework
     facilitates interaction with other web services and allows us to work with
     HTML, JSON, XML and JavaScript visualisation libraries during compilation.
     On the other hand, the framework can carry out tasks for which web
     technologies are usually not suitable, including video conversion, legacy
     document format access, file system access or TCP/IP connectivity.

     To enable validation of a \mxp document in the XML format described above,
     an XML Schema exists which is augmented with additional constraints
     provided by the plug-ins. After validating the document, it is parsed and
     any discovered tags might trigger preprocessor actions defined by the
     plug-ins, such as the extraction of data from referenced legacy document
     formats (e.g. \ppt or Excel) or the conversion of an unsupported video
     format. Each tag is then converted to HTML5 and all information is encoded
     in the attributes of a \code{div} element. The HTML5 standard allows
     custom attributes that start with a \code{data-} prefix. \lstref{xml-2}
     hilights converted parts of the original XML document we saw in
     \lstref{xml-1}. Observe that no visualisation-specific information is
     included in the transformation, which merely results in a valid HTML5
     document ready to bundle into a self-contained package together with the
     presentation engine.

     \begin{figure}[h!]
      \begin{lstxml}{xml-2}{Transformed HTML5 presentation content}
<div data-type="presentation">
  <div data-type="slide" data-title="Vannevar Bush">
    <div data-type="bulletlist">
      ...
      \end{lstxml}
     \end{figure}

    \subsubsection{Presentation Engine}

     The presentation engine's main purpose is to create a visually appealing
     and interactive presentation based on the compiled HTML content. As
     \figref{roels-1-fig-1} shows, the presentation engine consists of several
     smaller components which enable plug-ins to implement powerful features
     with minimal effort. The combination of these components allows for rapid
     prototyping and evaluation of innovative visualisation ideas. A resulting
     \mxp presentation combining various structure, container and component
     plug-ins is shown in \figref{roels-1-fig-3}.

     \fig{roels-1-fig-3}{A \mxp presentation}

     \paragraph{Content Engine} The content engine is the first component that
      is activated when a presentation is loaded. It uses the well-known
      jQuery JavaScript library to process the content of the HTML
      presentation. Whenever a \code{div} element is discovered, the
      \code{data-type} attribute is read and the corresponding plug-ins are
      triggered in order to visualise the content.

     \paragraph{Graphics Engine} The graphics engine accomodates interesting
      new visualisation and navigation styles. Apart from some basic helper
      functions, it provides efficient panning, scaling and rotation via CSS3
      transformations and supports zoomable user interfaces as well as the more
      traditional navigation approaches.

     \paragraph{Communication Engine} The communication engine offers
      abstractions that enable plug-ins to retrieve external content at runtime.
      It also provides the architectural foundation to form networks
      between different \mxp instances and to integrate third-party hardware
      \citep{roels-2}.
      For the \mxp prototype, a small Intel Next Unit
      of Computing Kit (NUC) was used with high-end WiFi and Bluetooth modules to act as
      a central access point and provide the underlying network support. \mxp presentation
      instances use WebSockets to communicate with other \mxp instances via the
      access point. The access point also acts as a container for data
      adapters that translate input from third-party devices
      into a generic representation that can be used by the \mxp instances in
      the network. In order to transcend simple broadcast-based communication,
      a routing mechanism was implemented based on the publish-subscribe
      pattern, allowing plug-ins to subscribe to specific events or publish
      information. The communication engine supplies the foundation for audience
      response systems \citep{roels-2} or even full classroom communication
      systems where the creativity of plug-in developers is the only limit.

     \paragraph{Plug-ins} Plug-ins are implemented as JavaScript bundles
      consisting of a folder containing JavaScript files and other resources
      such as CSS files, images or other JavaScript libraries. As a first
      convention, a plug-in should contain a manifest file with a predefined
      name. The manifest describes the plug-in using metadata such as the
      plug-in name and version but also a list of tags it provides and handles
      to be used in a presentation. The plug-in claims unique ownership for
      these tags and is solely responsible for their visualisation if the
      content engine encounters them. As a second convention, a plug-in must
      implement at least one JavaScript object providing certain methods, one
      of them being the \code{init()} method which is called when the plug-in
      is loaded by the presentation engine. The plug-in may decide to load
      additional JavaScript or CSS via the provided dependency injection
      functionality. A second method it needs to implement is the
      \code{process()} method which is invoked with a pointer to the
      corresponding DOM node as a parameter by the content engine when it
      encounters a corresponding tag. A plug-in is free to modify the DOM tree
      and may also register callbacks to enable future interaction with the
      content.

     \paragraph{Themes} CSS is currently the driving technology behind a basic
      templating system. These themes offer styling either on a global or on a
      plug-in level. It has always been the intention to replace this system
      with a more advanced layout engine that steers away from templates, and
      part of that is incidentally one of the goals of this thesis. The
      intention is to provide layout functionality far beyond what can be
      attained using templates. However, the styling functionality will still
      be handled by the current system. In the future this system may still be
      replaced or enhanced to allow more dynamic styling using JavaScript.

   \subsection{Use Cases}

    To prove the merits of the architectural and technological choices, we
    demonstrate the extensibility and feasibility of \mxp as a rapid
    prototyping platform through demonstration of a number of content- and
    navigation-specific plug-ins that have been developed so far. Additional
    plug-ins for audience-driven functionality such as real-time polls, screen
    mirroring and navigational takeover can be found in \citep{roels-2}.

    \subsubsection{Structured Overview Plug-in}

     In Section \ref{mxp-platform} we described how structure plug-ins may
     alter and influence the way presentations are visualised and navigated. In
     order to illustrate this, a structure plug-in called \emph{structured
     layout} was implemented, to combine a zoomable user interface with the
     ability to group content into sections. The resulting visualisation of the
     \emph{structured layout} plug-in is displayed in \figref{roels-1-fig-3}.
     Whenever a new section is reached while navigating through the
     presentation, the view is zoomed out to provide an overview of the content
     within the section and to convey a sense of progress.

    \subsubsection{Slide Plug-in}

     Even though one of the main intentions of \mxp is to discard the concept
     of slides with all their limitations, it was deemed necessary to include
     support for this concept as well. As such, a slide-like container plug-in
     was created. While the benefits and issues of using slides with a fixed
     size are debatable, this plug-in was implemented as a testament to the
     framework's versatility. The main purpose of the slide plug-in is to
     produce a rectangular styleable component container with an optional title
     and some other information. Containers may also contribute functionality
     to layout their content. In this instance, the slide plug-in implements a
     quick and easy layout mechanism that allows the presenter to partition the
     slide into rows and columns. Slide containers are then assigned to these
     slots in the order that they are discovered. The use of the slide plug-in
     together with the resulting visualisation is depicted in
     \lstref{roels-1-fig-4}. This demonstrates the use of the image plug-in (a
     component plug-in) as well, which introduces a simple form of cross-media
     transclusion. A visualised external image can be cropped and filters (e.g.
     colour correction) may be applied without duplicating or modifying the
     original source.

     \begin{figure}[h!]
      \begin{subfigure}{0.53\textwidth}
       \vspace{-1em}
       \begin{lstxml*}
<slide layout="\60\40" title="Vannevar Bush">
 <bulletlist>
  <item>About
   <item>March 11, 1890 - June 28, 1974</item>
   <item>American Engineer</item>
   <item>Founder of Raytheon</item>
  </item>
  <item>...
  </item>
 </bulletlist>
 <image source="http://example.com/bush.jpg">
  <crop bounds="10%, 5%, -10%, -20%" />
 </image>
</slide>
       \end{lstxml*}
      \end{subfigure}
      \hfill
      \begin{subfigure}{0.43\textwidth}
       \fignl{roels-1-fig-4}
      \end{subfigure}
      \renewcommand{\figurename}{Listing}
      \renewcommand{\figureshortname}{Lst.}
      \cl{roels-1-fig-4}{Slide plug-in}
     \end{figure}

    \subsubsection{Enhanced Video Plug-in}

     When showing a video in an educational setting, we often need more
     functionality than the average video player can provide
     \citep{reuss-1}. \mxp offers an enhanced video plug-in as demonstrated in
     \lstref{roels-1-fig-5}, adding the possibility to overlay a video with text
     or arbitrary shapes. This overlay functionality can be used as a basic
     captioning system as well as to highlight items of interest during
     playback.

     Furthermore, there is an option to trigger certain events at specified
     times. One may mark certain points where a video should automatically
     pause at, highlighting an object and then continuing playback after a
     specified amount of time. Other features include the bookmarking of
     certain positions in a video for direct access and the possibility to
     display multiple videos in a synchronised manner. The enhanced video
     plug-in leverages the default HTML5 video player and overlays it with a
     transparent \code{div} element for augmentation. Currently it utilizes the
     HTML5 video API to synchronise the creation and removal of overlays but a
     SMIL-based implementation might be used in the future.

     \begin{figure}[h!]
      \begin{subfigure}{0.53\textwidth}
       \vspace{-1em}
       \begin{lstxml*}
<video source="vid.mp4">
 <caption start="0:00" duration="1500ms">
  Lecture 3 - Butterfly Species
 </caption>
 <pause start="0:43" duration="5s">
  <caption>
   The peacock butterfly (aglais io) ...
  </caption>
  <highlight x="30%" y="9%"
             width="35%" height="40%" />
 </pause>
</video>
       \end{lstxml*}
      \end{subfigure}
      \hfill
      \begin{subfigure}{0.43\textwidth}
       \fignl{roels-1-fig-5}
      \end{subfigure}
      \renewcommand{\figurename}{Listing}
      \renewcommand{\figureshortname}{Lst.}
      \cl{roels-1-fig-5}{Enhanced video plug-in}
     \end{figure}

 \newpage

    \subsubsection{Source Code Visualisation Plug-in}

     We have previously mentioned the issues involved with visualising complex
     resources such as source code. Our \mxp source code plug-in exports a
     \code{code} tag allowing the presenter to paste their code into a
     presentation and have \mxp visualise it nicely through syntax highlighting
     using the
     SyntaxHighlighter\footnote{http://alexgorbatchev.com/SyntaxHighlighter/}
     JavaScript library. Whenever the content engine encounters a \code{code}
     tag, this plug-in is invoked to beautify the code. It also automatically
     adds vertical scrollbars for larger segments of source code as illustrated
     in \lstref{roels-1-fig-6}.

     \begin{figure}[h!]
      \begin{subfigure}{0.53\textwidth}
       \vspace{-1em}
       \begin{lstxml*}
<code>
 <publications>
  <publication type="inproceedings">
   <title>An Architecture for Open Cross-Media
          Annotation Services</title>
   <author>
     <surname>Signer</surname>
     <forename>Beat</forename>
   </author>
   <author>
     <surname>Norrie</surname>
     <forename>Moira</forename>
   ...
</code>
       \end{lstxml*}
      \end{subfigure}
      \hfill
      \begin{subfigure}{0.43\textwidth}
       \fignl{roels-1-fig-6}
      \end{subfigure}
      \renewcommand{\figurename}{Listing}
      \renewcommand{\figureshortname}{Lst.}
      \cl{roels-1-fig-6}{Source code visualisation}
     \end{figure}

   \subsection{Discussion and Future Work}

    \mxp currently supports transclusion and cross-media content reuse through
    the use of plug-ins. For example, the image or video plug-in can visualise
    (and enhance) external resources, a dictionary plug-in could retrieve
    definitions on demand via a web service or we might create a plug-in that
    lets us import content (e.g. \ppt slides) from legacy documents at compile
    time. Nevertheless, the introduction of generic reuse tags in our document
    format is actively being investigated. This would allow the presenter to
    transclude arbitrary parts of other \mxp presentations. While the focus has
    been on the cross-media aspect of resources that can be used in a
    presentation, the cross-media publishing aspect might also be considered in
    the future via alternative compiler output formats.

   %<TODO rework>
    We are aware that the current authoring of \mxp presentations has some
    usability issues. The average presenter cannot be expected to construct an
    XML document or any CSS themes. In order to tackle this issue and further
    evaluate \mxp in real-life settings, we are currently developing a
    graphical \mxp authoring tool. We further intend to provide a central
    plug-in repository which would make it easy for novice users to find,
    install and use new plug-ins via the authoring tool. In the long run, we
    intend to revise the use of monolithic documents and move towards
    repositories of semantically linked information based on the RSL hypermedia
    metamodel \citep{signer-3}. This would not only promote content reuse and
    sharing, but also create opportunities for context-aware as well as
    semi-automated presentation authoring where relevant content is recommended
    by the authoring tool.
   %</TODO rework>

  \section{Layout}

   Proper layout is incredibly important when trying to transfer knowledge and
   information through written and visual media. Layout can help clarify
   boundaries and relations between pieces of information, by grouping and
   separating them appropriately. Layout is one component of a presentation's
   design, that --- combined with other decisions --- determines the number and
   nature of the visual representations of the information the creator wants to
   communicate, along with its format\footnote{The way the visual objects are
   realised (e.g. as text, graphics, UI widgets\ldots), and their attributes
   (e.g. color, texture, font\ldots)}. The layout of a presentation can have a
   tremenduous influence on its effectiveness in communicating information to,
   and obtaining information from, the audience it is meant to interact with.
   The importance of individual objects can be emphasised or minimised, and the
   connection between obects can be clarified or blurred. A well laid out
   presentation can provide a narrative for the viewer to discover, inferring
   correct links between the objects along the way, and to accomplish tasks
   quickly and correctly, increasing the presentation's effectiveness.

   Creating a good layout is almost never easy. People often spend more time on
   the layout of their presentation than the content. Most, if not all,
   decisions in layout are made by human beings. Some of them are professional
   designers who spend years learning and figuring out how to create effective
   layouts, and even then they may take hours or days to create even a single
   screen of a presentation. In fact, the more someone knows about proper
   layout and design, the more time they may spend perfectioning their work.
   However, sometimes time-critical information must be communicated and the
   layout process is too expensive and too slow to address these situations.
   This can be a serious problem (see also section \ref{nasa}). Many software
   packages have been developed to make this process easier, to get better
   results, to give more or less control to the creators. Many different
   approaches have been taken, and yet most of them still involve having a
   human being make the final decisions on the layout.

   There are tools like \ppt, which give you some guidelines and some templates
   but generally let you do your own thing. If your own thing is entirely
   different from any best practices on layout, nothing will stop you. Other
   tools like \latex give you complete control over every aspect of the layout,
   while setting some sensible defaults so that you can get a good-looking
   layout without much effort, while still letting you do whatever you want
   once you overcome the steep learning curve that separates the casual users
   from the experts. There are tools that combine the power of \latex with the
   comfort of WYSIWYG editors, bringing the casuals a bit closer to the
   experts. But all of those tools have one thing in common: every aspect of
   every layout they create has, at some point, been designed and decided upon
   by a human being.

   Aesthetics are a natural phenomenon, and the creation of aesthetically
   pleasing layouts is therefore a manifestation of our instincts. As with most
   instincts, it has proven difficult to translate this into a concept that can
   be understood by a computer. Moreover, it is still difficult to explain it
   in human terms, which --- according to a popular quote often attributed to
   Albert Einstein --- proves we don't fully understand it ourselves.

   When we look to other technologies, we do find some automated layout
   implementations. For example, the web has had to adapt to mobile devices
   with small screens over the past few years, and has done this gracefully by
   creating the concept of responsive design. In short, this allows websites to
   adapt their layout to any screen, no matter the size. While this is often a
   hard-coded difference, where effectively two or more versions of the same
   webpage are created aimed at different screen sizes, some websites take a
   more dynamic approach based on constraints. As the space the page is to be
   displayed on gets smaller, the layout algorithm may decide to display
   content below other content instead of side-by-side, it may scale images to
   fit the screen, it may even switch fonts and font sizes if necessary.

   This constraint-based technique is described in a few papers
   \citep{lok-1,hurst-1}, but has --- to our knowledge --- not been applied in
   any presentation software so far. This is surprising, as presentations often
   look like they could use some of this magic. A proper constraint-based
   layout algorithm could allow any user to drop content onto a slide, without
   worrying about clarity or even legibility, and the algorithm could take care
   of the rest. Of course, there are some limits in traditional slideware that
   may hinder this approach: if a user decides to put more content on a slide
   than there is physical room available, the algorithm could either make the
   content smaller or split it across several slides, but either solution may
   bring its own problems up. An advantage of ZUI's is that no matter how small
   the content gets, we can still zoom in to make it clear again\footnote{It
   should be noted that \mxp in its current form does not support this level of
   zooming. While the software can zoom out to provide an overview of the
   presentation while zooming in on the separate components, it is not yet
   possible to zoom in or out extremely to reveal 'hidden' parts of the
   presentation. This is something we encourage to look into and change,
   because it can greatly improve both our layout solution as well as the whole
   \mxp experience in general.}.

% maybe TODO more stuff from papers about automated layout goes here

   \subsection{Algorithms}
    \label{related-algorithms}

    When researching layout algorithms, one will often come across the very
    active field of graph layout \citep{battista-1}. We will not go into the
    specifics of this field, as most of the issues with which it is concerned
    are specific to problems caused by the explicit visual representation of
    graph edges --- for example, the minimisation of edge crossing
    \citep{battista-2, shahrokhi-1}. The same applies to automated layout as
    referring to automated circuit layout for VLSI chip fabrication
    \citep{hu-1, lengauer-1} as well as automated placement of pieces to be cut
    from a bolt of cloth used to produce clothing \citep{milenkovic-1}.
    Contrary to presentation layouts (including graph layouts), these layouts
    are designed to meet the requirements of a fabrication process, rather than
    to make them understandable to humans. While some techniques used therein
    definitely apply to our more general problem of automated presentation
    layout (e.g. general constraint solvers) others decidedly do not (e.g.
    bin-packing techniques \citep{hofri-1} that result in minimal area layouts
    at the expense of maintaining visually obvious relationships between
    objects).



 % vim:ts=1:et:nospell:spelllang=en_gb:ft=tex

 \chapter{Problem statement}

  \section{Terminology}

  \section{Stuff we want to solve}

   TODO change title



 % vim:ts=1:et:nospell:spelllang=en_gb:ft=tex

 \chapter{Approach}

  In this chapter we explain the different approaches we tried in order to
  reach our goal and find a solution for the problem we described. As you will
  see, this was not immediately a straightforward process but rather one of
  trial and error. The goal was clear, the starting point was clear as well,
  but as often in computer science, there is more than one way to get from
  point A to point B, and it is not always clear which way is the best,
  easiest, most efficient or most effective.
 
  Since we're talking about the approach here, and not the implementation (for
  that, see chapter \ref{implementation}), we start by describing in broad
  terms what needs to be done and how this should be done, then we refine until
  we have a full set of specifications ready for implementation, where the last
  details will be ironed out.

  Unfortunately it is possible to refine an approach until it is ready for
  implementation, and only find out during implementation that the approach
  you've chosen will not work. This happened during our work on creating an
  automated layout system. Luckily we still had time to go back to the drawing
  board, and we did not have to restart from scratch; large parts of our
  approach were correct, the basic layout process we thought out was still a
  viable part of the approach, but it turned out we would have to split up the
  conversion and layout parts into two separate processes, rather than
  implementing them as two steps of the same process.
  
  Specifically, we had thought at first to figure out the ideal layout during
  conversion, when we would have all the separate components, by immediately
  putting them in the right place. This idea was partly conceived after looking
  at the HTML code generated by the \mxp compiler, thinking we would generate
  the same HTML code in our conversion process. It turned out we could bypass
  the \mxp compiler this way, but that wouldn't be necessary: we could just as
  well generate \mxp XML and have the compiler take care of the rest for us.
 
  We also found during implementation that generating a layout in Java would
  not easily give us the results we were hoping for. However, at this point we
  had realized generating \mxp XML would be a better option, so we could have
  \mxp take care of the layout for us. Except \mxp didn't do fully automated
  layout yet, the layout system was mostly template-based, so we decided we
  would need to write our own \mxp plug-in that would solve this problem for
  us.

  \section{Conversion process}

   The first part of the approach is fairly straightforward in its basic
   explanation: we had to convert \ppt presentations into \mxp presentations.
   This involved finding out how \ppt presentations are structured, getting the
   parts wee need out of that structure, and then putting those parts together
   in de \mxp structure.

   It appeared soon enough to us that the nature of this process resembled that
   of a compilation process. A compiler takes source code and transforms it
   into a working program with the semantics described by that source code. The
   compilation process consists of several steps. First the source code is
   tokenized, which means the symbols in the code are identified one by one and
   classified in certain categories.

   Then the tokens are processed by a parser into an intermediary form called a
   parse tree. A parser looks for certain predefined patterns in the source
   code. These patterns are part of the source code's language syntax. As such,
   these two steps analyse and validate the source code's syntax. If part of
   the code does not match any pattern, the parser and the compilation process
   stop and the user gets a message saying the code's syntax is invalid.

   When a parse tree is constructed, the compilation process can alter it, to
   improve it. Certain patterns in the parse tree may be replaceable by
   different patterns with the same outcome, but with more optimal execution.
   This part of the compilation process is optional, and is called compiler
   optimization. Optimizations can consist of many things, depending on the
   language. For example, some languages guarantee tail call optimization,
   where infinite loops can be constructed by letting a function call itself as
   its last statement without causing a stack overflow. This is something the
   compiler (or interpreter) can optimize during this part of the compilation
   process.

   After this, the parse tree can be written out to produce the desired output.
   Every node in the tree has a well-defined equivalent in the target
   language's syntax. The target language can be Assembly, which consists of
   the exact instructions a CPU needs to carry out a program, or it can be
   another programming language. Many compilers of higher-level languages
   translate their language into C, for several reasons: the C compilers that
   translate C into Assembly have been optimized so much that it is easier to
   rely on them than to put an enormous amount of effort into optimizing
   another language; C compilers exist for most --- if not all --- CPU
   architectures, which means translating a language into C makes it compatible
   with all those architectures, while it would cost a lot more effort to write
   different compilers for every architecture you would want to make your
   language available on.

   The conversion tool that is the purpose of this thesis, can be described in
   a similar succession of steps. As a first step, we take a \ppt presentation
   and take it apart into its components, effectively walking over each
   component, classifying them and registering their content type, original
   position and size, and any other specific properties. This can be seen as
   the tokenization phase, after which we end up with a series of `tokens' or,
   in our case, presentation components.
  
   We then turn this series of `tokens' into a `parse tree', an intermediary
   structure that reflects the relation between the components and the
   hierarchy of the presentation, which may consist of chapters, sections,
   slides and component groups. In \ppt this structure is fairly simple, so the
   creation of this `parse tree' is a straightforward process.
  
   However, in \mxp we are not limited to the rigid hierarchy of sections and
   slides, so at this point we can actually start manipulating our tree and
   improve upon it, for example by moving parts around, nesting components in
   different ways, grouping them in other ways than they originally were, etc.
   In compilation terms, this is the optimization phase, where the compiler can
   manipulate the program to run more efficiently, to replace parts of it with
   other functionality, or to add features the source didn't explicitly specify
   (e.g. garbage collection, but also spyware components \citep{scahill-1}). 

   As we discuss in section \ref{compiler-optimizations}, this seemed like the
   right time to incorporate automated layout generation into the conversion
   process. As we see later in section \ref{mxp-plug-in}, it turned out it
   wasn't. In the end, no significant `optimizations' or manipilation of the
   tree structure were implemented. Later on we would utilize this optimization
   phase to enable automated layout in another way, without actually performing
   the layout here, but at this point it would not affect the end result in any
   way.
  
   To finish the conversion process, we can traverse our component tree and
   generate a \mxp presentation from it. This can be done in several ways,
   since our intermediary form is in no way dependant on or bound to a specific
   format. Since the \mxp compiler was unavailable for a long time during our
   research and implementation, we decided it would be best to go straight to
   HTML5, so that we could test the conversion process without relying on the
   \mxp compiler. This worked out fairly well, although manually constructing
   HTML5 to work with the \mxp JavaScript library proved difficult. We ran into
   several issues, often mostly due to our lack of knowledge of the inner
   workings of \mxp, but we managed to get a presentable result that emulated
   the original \ppt presentation quite well.

   Afterwards, we altered our conversion tool to generate \mxp XML instead,
   which was a lot simpler since we would rely on \mxp to provide our layout
   and other things for us through the \mxp compiler. This approach allowed us
   to use the full power of \mxp, including our own plug-in for automated
   layout. At this point, the optimization phase was also revisited, and
   leveraged to introduce specific XML tags around component groups that would
   trigger our automated layout plug-in.

  \section{Compiler optimizations}
   \label{compiler-optimizations}

   Since the conversion process resembles that of a compiler, it seemed logical
   at first to make automated layout a part of that process, as some kind of
   `compiler optimization'. During this phase in the process, the component
   tree would be manipulated and altered, with the express purpose to improve
   upon its structure and properties, so as to get a better end result. Our
   improvements in this case would then consist of the automated layout.

   As a first attempt, we tried to traverse the component tree, giving each
   object new coordinates and sizes based on their original coordinates and
   sizes, as well as the coordinates and sizes of objects around them, so that
   they would fit together on every slide as well as possible. This seemed an
   easy solution, but the results were sub-optimal. On top of that, we soon
   realised that we were in essence creating another template-based system that
   would generate slides and presentations based on predefined ratios and
   rules, which was exactly the opposite of what we were trying to do. As such,
   we abandoned this approach in favor of a constraint-based algorithm as
   described in section \ref{related-algorithms}.

   This involved a technique that at first sight may seem like yet another
   template system, but actually is completely different: defining constraints
   for every component, in the form of margins, maximum sizes and other limits,
   and then calculating a way to satisfy all constraints while fitting content
   together on each slide. The similarities with template-based systems exist
   in the presence of predefined constraints, ratios and rules, but the
   important difference is that these constraints are defined relative to the
   component itself, without specifying anything absolute about location or
   size. For example, we would retain the aspect ratio of an image, without
   specifying its size, so that the image may be scaled to accomodate other
   components in a dynamic layout. As another example, we might specify there
   needs to be a certain distance between a component and any other components,
   relative to its size. We could also specify a certain relation between
   components, ensuring components stay in each others vicinity, one should
   always be left of the other, no other components may be placed between them,
   etc. Using these rules, we would then programmatically calculate the best
   layout using those components, but without any other bias. These constraints
   would be based only on the original situation, never on any suggestions from
   us or other developers or authors, which makes all the difference with
   traditional template-based layouts.
 
   While this is clearly a better method, it turned out the compiler
   optimization phase was not the best place in the process to take care of
   this. While we had the necessary data to calculate the layout, we would have
   had to generate the layout along with the \mxp presentation, after which the
   presentation could not be altered anymore without breaking the layout. This
   defeated the purpose of exporting to \mxp, which was to allow the presenter
   to edit, extend and improve their presentation further using \mxp. What we
   needed was a way to get \mxp itself to generate the layout, even if we
   wanted to add components to the presentation afterwards, and even if we
   wanted to create a new \mxp presentation instead of starting from \ppt.
   After all, how would we convince people to drop \ppt for \mxp's automated
   layout capabilities if they could only use that functionality by starting
   from \ppt?

   In the end, we decided to change our approach again. We took the automated
   layout out of the conversion process, instead opting at this point in the
   process to only add the necessary layout triggers in the form of an
   enclosing XML tag around the components that would need to be included in
   the automated layout. As such, the generated \mxp XML would include those
   tags, and a plug-in (described in section \ref{mxp-plug-in}) would then
   generate the layout at runtime.
  
  \section{Using \mxp}

   One of the primary goals of \mxp is to separate content from layout,
   allowing the author of a presentation to focus on the content while \mxp
   takes care of the layout. The way it does this is currently mostly through
   the compiler, which decides the width, height and coordinates of content,
   relative to the container the content belongs to. The plug-ins responsible
   for handling components and containers currently don't mess with those
   settings, but technically, they could. The compiler decides the measurements
   and coordinates based on templates. The solution we were looking for was a
   layout engine that could take any content and put it in an appropriate
   layout without any directions from the user. As such, we had to enhance
   \mxp's layout engine to use constraints, based on the size of the content,
   and try to find an optimal position for every component it is given.

   \subsection{A \mxp plug-in}
    \label{mxp-plug-in}

    We did this by creating an invisible container plug-in. Containers are a
    way of grouping components, other containers, etc. in \mxp. This means they
    have control over their child elements, which gives us the opportunity to
    override the layout of those elements. A container plug-in thus allows us
    to implement our own layout system. Since it's a new element, it doesn't
    override existing elements as it would have done if we had, for example,
    rewritten the `slide' plug-in. The user can decide for themself whether or
    not to use it, and it can be used anywhere in the presentation: wrap the
    whole presentation in it, or just a small part, whichever works best for
    your purposes. It also won't break existing presentations that don't use
    it, while those presentations can very easily be altered to take advantage
    of it.

    An important aspect of this is that containers can be nested. This means we
    can create slide-based presentations, which can contain our
    \code{autolayout} container, which then contains the slide's contents, thus
    creating an optimal layout of the content per-slide. Another way of using
    it could be without slides, throwing all content together in one
    \code{autolayout} container, and letting it take care of the layout for the
    whole presentation at once. It should be noted here that the
    \code{autolayout} container makes each of its child nodes focusable
    separately, to compensate for arbitrary resizing it may perform on large
    objects in order to fit them next to other content, by using the focus
    functionality to automatically zoom into these components when necessary.
  
    We call it an \emph{invisible} container plug-in because it does not
    introduce any visual content, shape or indication for itself. Compare with
    the \code{slide} plug-in which obviously puts some kind of slide-look
    around the content it encompasses, and it becomes clear what we mean by
    this: although the content within is obviously affected by our plug-in,
    there is no visible indication of its presence to the audience.

    The plug-in uses the compiler's numbers to decide relative locations
    between components, as well as size ratios, and then finds a way to display
    those components in a way that the display order makes sense (or at least
    matches the intended order as closely as possible), that no overlapping
    occurs (since we don't have the animations that \ppt might have used to
    display one piece of information and then another on top of it), and
    resizing everything if necessary in order to fit within the specified
    container. While this may seem like a bad idea since content can get
    illegibly small this way, keep in mind that we can rely on the ZUI to focus
    on each component separately, or on groups of components, while \ppt
    obviously can only display the whole slide at once.

    In this manner we would generate \mxp presentations that were immediately
    usable, while also being adjustable; and on top of this, we would allow the
    automated layout process to be used in other \mxp presentations that were
    not originally converted from \ppt slides. The goal of this plug-in would
    thus be to provide automated layout functionality to \mxp presentations,
    and to allow any \mxp author to use it simply by putting an
    \code{<autolayout></autolayout>} container around the components they want
    the plug-in to act on. This approach has the additional advantage that the
    container can be used multiple times throughout the presentation, while
    also allowing other parts of the presentation to have a manual layout.
  
    Since it is possible to nest containers, which means a number of components
    could be grouped together in an \code{autolayout} container, then the
    result could be put into another \code{autolayout} container together with
    other components --- other \code{autolayout} containers, perhaps --- to
    generate an automated layout for an overview of the different groups.
    Compare it with traditional slideware, where components are grouped
    together in slides, then the slides might be put next to each other in an
    overview --- except our approach drops the slide boundaries, while still
    maintaining the ability to group components together to show a relation or
    link between them.

   \subsection{An automated layout algorithm}

    As discussed earlier, our first approach included an algorithm where
    content would be placed on slides according to certain rules, trying to
    attain a mythical `perfect' layout based on the golden ratio, symmetry,
    centering and other general guidelines we would find in advice on creating
    presentations. It turns out that, while following those guidelines as a
    human being is generally a good idea, a computer has different ways of
    calculating a good layout. The issue here can be compared to other problems
    in computer science; for example, people in the robotics department have
    tried for decades to create a robot that behaves exactly like a human
    being, and people in artificial intelligence have tried to create an AI
    that thinks like us. However, we've found time and time again that
    computers simply aren't very good at acting, thinking or being human, just
    like we aren't good at being computers. Making a computer act like a human
    makes it disadvantaged --- almost by definition, just like we are severely
    handicapped when we try to perform typically automatable, repetitive and/or
    math-intensive tasks. A computer's true power only shows when you let it do
    what it's good at, which is the repetitive stuff, the mathematically
    complex stuff, etc. Trying to make it generate presentation layouts like a
    human would, is asking for subpar results.

    If we approach this problem keeping in mind a computer's strength and
    weaknesses, we arrive at a different approach. This involves calculating
    sizes, ratio's, positions, margins and other numbers, of which the formulas
    are actually not too hard to come up with as a human, but which the
    execution is definitely more of a computer task. We start off by checking
    each component, and noting its original location and size. We then try to
    find components that are in proximity of each other, and figure out their
    original layout: above/below each other, next to each other, overlapping...
    Then, we try to put them together, possibly resizing them to match each
    other's sizes, and trying to match their original relative locations while
    introducing a certain rigidity, or consistency, by aligning them properly
    and puzzling them together as neatly as possible.
   
    This last part may sound weird, but it really is something to take into
    consideration, especially when the amount of components might be much
    bigger than what should fit on an average traditional slide. You could put
    all components in a row, just displaying them side-by-side, but that is not
    very aesthetically pleasing. Instead, we opted to try and keep components
    close to each other. This was finally achieved by finding the location
    closest to the starting point that would fit the component being
    considered, while still taking into account the earlier constraints about
    relative location and size. Thanks to the ZUI in \mxp, this makes for
    interesting layouts that still remain manageable, and provide a nice
    overview of all content when zoomed out.


 % vim:ts=1:et:nospell:spelllang=en_gb:ft=tex

 \chapter{Implementation}
  \label{implementation}

  To implement the \emph{ppt2mxp} conversion tool that is the subject of this
  thesis, we chose the Java programming language \citep{gosling-1}, version 8.
  Although the author has significant experience with lots of other, more
  interesting, more compelling, more fun languages, several reasons pushed us
  towards Java, the least of them being its ease of use. Of course, Java
  \emph{is} easy to use --- it would not have become as popular as it is
  nowadays if it wasn't. It has a fairly clear and logical syntax, a consistent
  structure, and an extensive standard library. At conception in 1995, its
  performance was abysmal, but through the years it has steadily improved and
  somewhere between Java 5 and 6 it became an industry standard.

  Quite a number of IDEs have been created to further improve developers'
  experience working with Java. Netbeans, Eclipse and IntelliJ come to mine,
  although there are many others, and of course you can still write Java using
  a standard (or advanced) text editor such as Notepad or VIM. While the author
  usually prefers the latter for any kind of text editing --- this very
  document was written entirely using VIM --- the weapon of choice when it
  comes to Java is currently IntelliJ. The way IntelliJ practically writes more
  than half of the code automatically for you is something no other IDE has
  been able to match. Naturally, this is the author's personal opinion and
  should not be seen as fact, but if you're looking for a new Java IDE, it's
  definitely worth checking out. The prospect of using IntelliJ for this thesis
  has definitely contributed to the decision of using Java. It should be noted
  that, had another Java IDE been required, this thesis might never have seen
  the light of day.

  The vast and extensive amount of libraries available for Java was obviously
  one of the more important reasons to make this choice. The existence of the
  Apache POI library (see section \ref{poi}) was a huge help in reaching our
  goal; without it, we would have had to figure out the very obfuscated .ppt
  file format structure, which undoubtedly would have taken up more time than
  was available to us. Other libraries like Spring, which allows the programmer
  to use and reuse components without writing complex systems to instantiate
  them, further increased our resolve to make Java our primary technology
  choice.

  However, Java is not the only technology used here. \mxp is written entirely
  in HTML5, so any tool that somehow relates to \mxp sooner or later needs to
  use HTML5 as well. The widely accepted HTML5 standard makes \mxp
  presentations highly portable and runnable on any device with a recent web
  browser, including smartphones and tablets \citep{roels-1}.

  In the following sections we discuss how the various technologies were used
  to create the \emph{ppt2mxp} tool.

  \section{Taking \ppt apart}
   \label{poi}

   When converting one file format into another, the first part of the process
   involves getting the data you need out of the original file. This can be
   very complicated, as some --- usually proprietary --- file formats are
   deliberately designed to discourage this. They obfuscate data, encrypt it,
   and structure it in illogical and unexpected ways, amongst other techniques.
   The \ppt file format unfortunately is such a format, as Microsoft wouldn't
   want to risk other companies making software that would work with \ppt
   files. Of course, over the years people have managed to crack the format,
   enabling the conversion of \ppt presentations into other formats, although
   the conversion does not usually guarantee to yield results that mimic the
   original version perfectly. Luckily, we don't want a perfect conversion, we
   want a better one.

   We found Apache POI library very helpful in this part of the implementation.
   The POI Library --- formerly ''Poor Obfuscation Implementation''
   \citep{sundaram-1} --- is a Java library that provides an API to access
   Microsoft document formats. The most mature (and most popular) part of it is
   HSSF, which stands for Horrible SpreadSheet Format, and which is used by
   Java developers worldwide to access Microsoft Excel spreadsheet data, as
   well as export data into Excel spreadsheets.

   For our purposes, we relied on HSLF (''Horrible SLideshow Format''), which
   gave us access to a \ppt presentation's contents in many ways. We could
   access all images at once, or every bit of text from the whole presentation,
   but the most interesting to us was the ability to access contents on a
   per-slide basis. This allowed us to loop over the presentation's slides,
   converting them one by one, by placing the contents of each slide in a \mxp
   slide equivalent.

   That was sadly not the end of it. While HSLF does give us access to all the
   text in a presentation, or per slide, it does not distinguish between
   'normal' text and bullet lists, for example. This was a difference we had to
   detect ourselves somehow. % TODO find out and explain how we do this.

   Another challenge was dealing with animations and other ways people managed
   to put way more content on one slide than would be advisable. The animations
   could not be transferred to \mxp since \mxp has its own set of transitions.
   It would technically be possible to implement additional animations as a
   separate plug-in for \mxp to provide the equivalents of the animations in
   \ppt*, but that is beyond the scope of this thesis. So we could not provide
   the same animations, but some people use those animations not just to show
   off but to actually show multiple pictures and blocks of text, one after the
   other, on the same slide. Without animations, this content would either not
   be visible or it would become a serious layout issue in \mxp. Our solution
   proposes to limit the amount of objects one slide can contain, and any
   additional content should be put on extra slides automatically. A downside
   of this is that we currently have no way of guessing the correct order in
   which the content should appear, so what may have been an intrinsic
   choreography of pictures in \ppt may become an incoherent jumble of images
   in \mxp. Another solution would be to scale all content until it all fits
   next to each other on one slide, and then rely on the zoomable interface to
   show the pictures one by one, but in this case the same problem with order
   of appearance manifests itself. In the end, we decided it would be best to
   accept that no conversion algorithm is going to be perfect, and the author
   can always manually change the order around after the conversion is done.

  \section{Generating \mxp}

%   TODO generating

   \subsection{Plain HTML5}

    Since the \mxp compiler was not functional during most of this thesis'
    implementation, we decided to generate an html file much like the \mxp
    compiler would, including the \mxp JavaScript library and plug-ins. This
    required us to first find out how \mxp works on the inside, which proved to
    be a steep learning curve but gave us more insight into the software than
    we would've gotten if we only had to generate \mxp XML and leave the rest
    to the compiler.

   \subsection{\mxp XML}

%    TODO XML

  \section{Creating layouts}

%   TODO layout

   \subsection{Using constraints}

%    TODO constraints

   \subsection{Other ways}

%    TODO other ways

%   15:50 <omega> zeg, ik zit nu al een hele tijd thesis te schrijven en de laatste paar dagen vooral te zeveren over layout, maar intussen doe ik nog ni echt iets van layout, met t gedacht van ik schrijf daar binnenkort ne mindxpres plug-in voor en klaar
%   15:51 <omega> maar wordt layout momenteel eig ni mostly door de compiler gedaan?
%   15:52 <omega> ben zo eens naar de presentation.js libs en code gaan kijken, en ik zie ni direct een manier om ne plug-in layout te laten doen, aangezien plug-ins mostly component-specifiek zijn en ni alle componenten kunnen aansturen
%   15:53 <omega> dus klopt het dat ik ofwel de compiler moet aanpassen, ofwel presentation.js hacken om dat soort plug-ins toe te laten?
%   15:53 <omega> of laat het dat soort plug-ins al toe maar zijn er gewoon nog geen?
%   16:30 <omega> de 'structured' plug-in doet wel layout van slides, maar binnen die slides zie ik niet meteen een systeem dat layout regelt, met templates of otherwise, het pakt gewoon de coordinaten en afmetingen die de compiler bepaald heeft
%   16:31 <omega> al zou die slide plug-in wel *kunnen* prutsen met die layout... dus mss moet ik gwn de slide plug-in uitbreiden/hacken/vervangen
%   13:22 <reinout> ik zou een container plug-in maken
%   13:22 <reinout> gelijk de slide
%   13:22 <reinout> maar dan onzichtbaar
%   13:23 <reinout> want containers kan je nesten
%   13:23 <reinout> dus een slide kan bv uw layout container bevatten, die dan de children een layout geeft
%   13:23 <reinout> maar op die manier is uw layout ding bruikbaar buiten slides
%   13:24 <reinout> (alternatief was uw layout stuff in de slide plug-in steken)
%   13:31 <omega> oeh, cool idee indeed, beter dan de slide plug-in abusen


 % vim:ts=1:et:nospell:spelllang=en_gb:ft=tex

 \chapter{Conclusions and Future Work}

%  TODO conclusion

  \section{Contribution}

%   TODO why is this useful at all

   Converting \ppt presentations into \mxp is now possible and easy.

   Automated layout has not been available in presentations until now. \mxp is
   the first presentation system that doesn't require predefined templates, nor
   manual layout tweaking by the enduser, instead letting people focus on the
   content while the software takes care of the rest.

  \section{Future Work}

%   TODO what should the next thesis slave still fix

   Other formats: extend the convertor tool to convert Keynote, prezzi, pptx,
   \ldots

   Integrate the convertor into an \mxp editor

   Improve the automated layout algorithm beyond constraints using learning AI,
   training it on good/bad layouts, neural network...

   Implement extreme zooming into \mxp using the CSS3 perspective property
   along with translation along the z-axis to utilize the full power of the 3d
   space.



 \newpage

 \bibliographystyle{IEEEtranN}
 \bibliography{db}

\end{document}

