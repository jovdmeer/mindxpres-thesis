% vim:ts=1:et:spelllang=en

%\documentclass[a4paper,12pt]{thesis}
\documentclass[a4paper,12pt]{report}
%\documentclass[a4paper,12pt]{book}

% The following makes latex use nicer postscript fonts.
%\usepackage{times}
\usepackage[english]{babel}
\usepackage{graphicx}
\graphicspath{ {img/} }
\usepackage[usenames]{color}
\usepackage{multicol}
%\usepackage[colorlinks,urlcolor=blue,linkcolor=blue]{hyperref}
\usepackage[%ps2pdf,
            bookmarks=true,
            bookmarksnumbered=false,
            bookmarksopen=false,
            colorlinks=false,
%            colorlinks=true,
            linkcolor=webred]{hyperref}
\definecolor{webgreen}{rgb}{0, 0.5, 0} % less intense green
\definecolor{webblue}{rgb}{0, 0, 0.5} % less intense blue
\definecolor{webred}{rgb}{0.5, 0, 0} % less intense red
\usepackage[round,comma,authoryear]{natbib}

%\hyphenation{administrative argument arguments assignments complex evaluates functions happening however machine understand unreliable variable variables whenever}

\usepackage{vubtitlepage}
\author{Joris Vandermeersch}
\title{Content Migration and Layout for the \mxp Presentation Tool}

%\promotortitle{Promotor/Promotors}
\promotor{Prof. Dr. Beat Signer}
\advisors{Reinout Roels}
\advisortitle{Begeleider}
\faculty{Faculteit Wetenschappen}
\department{Departement Informatica\\
            en Toegepaste Informatica}
\reason{Proefschrift ingediend met het oog op het behalen\\
        van de graad van Master in de Toegepaste Informatica}
\date{Mei 2015}

% vim:ts=2:et:spelllang=en

\usepackage{xspace}

\newcommand\code{\texttt}
\newcommand\ppt{Microsoft PowerPoint\xspace}
\newcommand\mxp{MindXpres\xspace}
\newcommand\latex{\LaTeX\xspace}



\begin{document}

 % First dutch TitlePage
 \maketitlepage

 \faculty{Faculty of Science}
 \advisortitle{Advisor}
 \department{Department of Computer Science\\
             and Applied Computer Science}
 \reason{Graduation thesis submitted in partial fulfillment of the\\
         requirements for the degree of Master in Applied Computer Science}

 \date{May 2015}

 % Then english TitlePage
 \maketitlepage

 % vim:ts=1:et:nospell:spelllang=en_gb:ft=tex

 \chapter*{Abstract}

  \ppt* continues to be used worldwide in staggering numbers. We try to provide
  an alternative with \mxp, facilitating the switch by converting existing \ppt
  presentations into \mxp presentations, and automatically fixing the layout in
  the process.

  TODO judging by other thesises, this should be longer



 % vim:ts=1:et:nospell:spelllang=en_gb:ft=tex

 \chapter*{Acknowledgements}

  \emph{\makebox[0pt][r]{``}Simplicity is a great virtue,\\
  but it requires hard work to achieve it and education to appreciate it.\\
  And to make matters worse: complexity sells better.''}

  \hfill\emph{--- Edsger W. Dijkstra}\\

%%% TODO thank people

%  First of all, I'd like to thank my fianc\'ee, Tania, for her love, for her
%  support, and for her patience. This thesis has been the last hurdle of a race
%  I've been running for way too long, for which I've had to put both of our
%  lives on hold way too often. It's been stressful, and I'm sure she's looking
%  forward to the load being lifted as much as I am.

%  02:24 <reinout> ik eis een dodentocht vermelding
%  02:25 <reinout> "and I would like to expres my hatred towards my supervisor reinout roels for dragging me along for the worst 20km of my life"
%  Reinout, my advisor and friend, deserves my eternal gratitude for giving me
%  the chance to get this over with, and dragging me through it as well. I
%  could never have finished this without his help, his guidance and his
%  enthousiasm.

%  This section would not be complete without mentioning Peter, my brother, who
%  showed me the fun side of studying, the more significant parts of life, and
%  the importance of procrastination. I'm always proud to call myself his big
%  brother.

%  This thesis was further brought to you with the support of many other people,
%  all of whom really deserve more than just being mentioned in my
%  acknowledgements section, and if they ever need any help they'll know to find
%  me:
%  \begin{itemize}
%   \item My friends, for frequently providing the necessary distraction from this awful ordeal, as well as for giving me many more fun distractions to look forward to whenever there seemed to be no end to this.
%   \item My grandparents, for the moral and financial support which gave me the opportunity to start and finish this, even though I wanted to give up ages ago.
%   \item My parents, for the genes, the upbringing and the moral support, even if they often have no clue what I'm talking about.
%   \item Infected Mushroom, for providing the soundtrack to many late-night caffeine-fueled hack- and write sessions.
%   \item RedBull and Nalu, for providing the fuel for those same sessions.
%   \item My colleagues at Roots Software, for putting up with my quirks, but more importantly for showing me the world beyond academics.
%   \item All professors, assistants and other academic personnel at the VUB, for making me realize a long time ago that I never want to be a part of the academic world. Seriously.
%  \end{itemize}



 \tableofcontents

% TODO
%14:52 <reinout> thesis is wel een ok begin, but it all needs a lot of padding :p
%14:52 <reinout> bijvoorbeeld, you should spend ~5pg on mindxpres
%14:52 <reinout> waarom bestaat het, hoe werk dat plug-in gedoe, hoe werkt dat xml gedoe
%14:55 <reinout> basically alles wat in deze paper staat:
%14:55 <reinout> https://www.academia.edu/4186970/An_Extensible_Presentation_Tool_for_Flexible_Human-Information_Interaction
%14:55 <reinout> nee wacht
%14:55 <reinout> deze:
%14:56 <reinout> https://www.academia.edu/7719770/MindXpres_An_Extensible_Content-driven_Cross-Media_Presentation_Platform
%14:56 <reinout> explain ALL the MindXpres
%14:57 <reinout> en dan kan je verdergaan, "mensen hebben nu hun content, maar nu willen we dat ze die content in MindXpres kunnen gebruiken"

 % vim:ts=1:et:nospell:spelllang=en_gb:ft=tex

 \chapter{Introduction}

  For over 25 years, \ppt* has been the market leader in digital prsentations.
  Admittedly, it was a revolutionary software package when it was first
  introduced, and its ease-of-use combined with its supreme graphical
  capabilities --- at least compared to other software in the same era --
  quickly made it one of the most popular software packages in history. 25
  years later, \ppt* can claim over 90\% market share in presentation software,
  and on average 30 million \ppt presentations are created every day.

  In this time, \ppt* has gotten many new features, and certainly improved and
  grew with every new version, but it never really changed its core approach.
  It started out mimicking the then-popular and widespread use of dia and
  overhead projection slides, which was at the time a good way to convince
  people of its purpose, allowing them to feel comfortable with a familiar
  format instead of alienating potential customers with a new and potentially
  confusing interface.

  However, this interface is quite restricting, and in recent years different
  approaches have seen the light of day. The zoomable user interface of Prezi
  is probably the most well-known, but apart from abandoning the traditional
  slide format it does little to improve or extend the concept of presenting
  information to an audience.

  This is where \mxp comes in. Its extensible plugin system allows anyone with
  some knowledge of programming to create new functionality to use in
  presentations. Examples are interactivity with the audiencer through various
  means, controlling the presentation from another device --- or several! ---
  and (re)modelling data while presenting it, based on feedback from the
  audience.

  While this is obviously a big improvement on the traditional presentation
  model of \ppt* and the likes, it remains hard to convince the general public
  of its merits. People are generally afraid of change, and it is important to
  make the transition as smooth as possible. On top of that, people are often
  worried that the work they did in the past may be lost --- or worse,
  irrelevant --- after switching to something new. This alone may be a huge
  factor in deciding wether or not to start using new software, or to stick
  with what they know.

  That is where the subject of this thesis comes in. We aim to provide a way
  for people to convert their existing \ppt presentations into \mxp
  presentations, allowing them to take their previous work with them in their
  switch to \mxp. This way, we lower the treshold for them to make the decision
  to start using \mxp as their presentation software of choice. Once all their
  existing \ppt content is available, usable and editable in \mxp, it should be
  obvious to anyone why \mxp is the better option for their presentations.

  TODO argue that automatic layout helps liberate the content from the confines
  of slides, not just fix bad layout

  Another common problem with \ppt presentations is the way they look. This is
  not necessarily the fault of the software; most people just are not trained
  in graphical design, and as such they know very little about proper layout,
  color choices, or slide content limits. Everyone has probably encountered
  slides with full paragraphs of text, too small to read and / or too much to
  process in the short time the slide is visible --- (too) many people have
  made those slides themselves.

  When we say this is not the fault of the software, that is mostly true, as
  the creators of these slides obviously made a conscious choice to make their
  content appear like that. It could be said however that \ppt* and other
  presentation tools are guilty through inaction. We believe it is possible to
  have software either warn its users against these choices and practices, or
  --- even better --- have the software fix these problems automatically.

  One of the primary goals of \mxp is to provide automatic layout, much like
  \latex does, ensuring that the content creator only has to worry about the
  actual content, while the software takes care of layout. In practice, both
  \latex and \mxp currently use template-based layouts, where the contents'
  position is predefined in the template and not related to or based on its
  size, shape or nature. In the end, everyone who has ever used \latex knows
  that sooner or later you will struggle to get a certain image incorporated in
  the text correctly, ending up doing the layout yourself anyway, because the
  predefined template just doesn't work properly for your specific content.

  As such, the second part of this thesis focuses on implementing true
  automatic layout in \mxp. Again primarily to convince \ppt* users to switch,
  showing that their presentations actually could look better in \mxp, while
  thus also providing new functionality to existing \mxp users.

  TODO moar?


 % vim:ts=1:et:nospell:spelllang=en_gb:ft=tex

 \chapter{Problem statement}

  \section{Terminology}

  \section{Stuff we want to solve}

   TODO change title



 % vim:ts=1:et:nospell:spelllang=en_gb:ft=tex

 \chapter{Approach}

  In this chapter we explain the different approaches we tried in order to
  reach our goal and find a solution for the problem we described. As you will
  see, this was not immediately a straightforward process but rather one of
  trial and error. The goal was clear, the starting point was clear as well,
  but as often in computer science, there is more than one way to get from
  point A to point B, and it is not always clear which way is the best,
  easiest, most efficient or most effective.
 
  Since we're talking about the approach here, and not the implementation (for
  that, see chapter \ref{implementation}), we start by describing in broad
  terms what needs to be done and how this should be done, then we refine until
  we have a full set of specifications ready for implementation, where the last
  details will be ironed out.

  Unfortunately it is possible to refine an approach until it is ready for
  implementation, and only find out during implementation that the approach
  you've chosen will not work. This happened during our work on creating an
  automated layout system. Luckily we still had time to go back to the drawing
  board, and we did not have to restart from scratch; large parts of our
  approach were correct, the basic layout process we thought out was still a
  viable part of the approach, but it turned out we would have to split up the
  conversion and layout parts into two separate processes, rather than
  implementing them as two steps of the same process.
  
  Specifically, we had thought at first to figure out the ideal layout during
  conversion, when we would have all the separate components, by immediately
  putting them in the right place. This idea was partly conceived after looking
  at the HTML code generated by the \mxp compiler, thinking we would generate
  the same HTML code in our conversion process. It turned out we could bypass
  the \mxp compiler this way, but that wouldn't be necessary: we could just as
  well generate \mxp XML and have the compiler take care of the rest for us.
 
  We also found during implementation that generating a layout in Java would
  not easily give us the results we were hoping for. However, at this point we
  had realized generating \mxp XML would be a better option, so we could have
  \mxp take care of the layout for us. Except \mxp didn't do fully automated
  layout yet, the layout system was mostly template-based, so we decided we
  would need to write our own \mxp plug-in that would solve this problem for
  us.

  \section{Conversion process}

   The first part of the approach is fairly straightforward in its basic
   explanation: we had to convert \ppt presentations into \mxp presentations.
   This involved finding out how \ppt presentations are structured, getting the
   parts wee need out of that structure, and then putting those parts together
   in de \mxp structure.

   It appeared soon enough to us that the nature of this process resembled that
   of a compilation process. A compiler takes source code and transforms it
   into a working program with the semantics described by that source code. The
   compilation process consists of several steps. First the source code is
   tokenized, which means the symbols in the code are identified one by one and
   classified in certain categories.

   Then the tokens are processed by a parser into an intermediary form called a
   parse tree. A parser looks for certain predefined patterns in the source
   code. These patterns are part of the source code's language syntax. As such,
   these two steps analyse and validate the source code's syntax. If part of
   the code does not match any pattern, the parser and the compilation process
   stop and the user gets a message saying the code's syntax is invalid.

   When a parse tree is constructed, the compilation process can alter it, to
   improve it. Certain patterns in the parse tree may be replaceable by
   different patterns with the same outcome, but with more optimal execution.
   This part of the compilation process is optional, and is called compiler
   optimization. Optimizations can consist of many things, depending on the
   language. For example, some languages guarantee tail call optimization,
   where infinite loops can be constructed by letting a function call itself as
   its last statement without causing a stack overflow. This is something the
   compiler (or interpreter) can optimize during this part of the compilation
   process.

   After this, the parse tree can be written out to produce the desired output.
   Every node in the tree has a well-defined equivalent in the target
   language's syntax. The target language can be Assembly, which consists of
   the exact instructions a CPU needs to carry out a program, or it can be
   another programming language. Many compilers of higher-level languages
   translate their language into C, for several reasons: the C compilers that
   translate C into Assembly have been optimized so much that it is easier to
   rely on them than to put an enormous amount of effort into optimizing
   another language; C compilers exist for most --- if not all --- CPU
   architectures, which means translating a language into C makes it compatible
   with all those architectures, while it would cost a lot more effort to write
   different compilers for every architecture you would want to make your
   language available on.

   The conversion tool that is the purpose of this thesis, can be described in
   a similar succession of steps. As a first step, we take a \ppt presentation
   and take it apart into its components, effectively walking over each
   component, classifying them and registering their content type, original
   position and size, and any other specific properties. This can be seen as
   the tokenization phase, after which we end up with a series of `tokens' or,
   in our case, presentation components.
  
   We then turn this series of `tokens' into a `parse tree', an intermediary
   structure that reflects the relation between the components and the
   hierarchy of the presentation, which may consist of chapters, sections,
   slides and component groups. In \ppt this structure is fairly simple, so the
   creation of this `parse tree' is a straightforward process.
  
   However, in \mxp we are not limited to the rigid hierarchy of sections and
   slides, so at this point we can actually start manipulating our tree and
   improve upon it, for example by moving parts around, nesting components in
   different ways, grouping them in other ways than they originally were, etc.
   In compilation terms, this is the optimization phase, where the compiler can
   manipulate the program to run more efficiently, to replace parts of it with
   other functionality, or to add features the source didn't explicitly specify
   (e.g. garbage collection, but also spyware components \citep{scahill-1}). 

   As we discuss in section \ref{compiler-optimizations}, this seemed like the
   right time to incorporate automated layout generation into the conversion
   process. As we see later in section \ref{mxp-plug-in}, it turned out it
   wasn't. In the end, no significant `optimizations' or manipilation of the
   tree structure were implemented. Later on we would utilize this optimization
   phase to enable automated layout in another way, without actually performing
   the layout here, but at this point it would not affect the end result in any
   way.
  
   To finish the conversion process, we can traverse our component tree and
   generate a \mxp presentation from it. This can be done in several ways,
   since our intermediary form is in no way dependant on or bound to a specific
   format. Since the \mxp compiler was unavailable for a long time during our
   research and implementation, we decided it would be best to go straight to
   HTML5, so that we could test the conversion process without relying on the
   \mxp compiler. This worked out fairly well, although manually constructing
   HTML5 to work with the \mxp JavaScript library proved difficult. We ran into
   several issues, often mostly due to our lack of knowledge of the inner
   workings of \mxp, but we managed to get a presentable result that emulated
   the original \ppt presentation quite well.

   Afterwards, we altered our conversion tool to generate \mxp XML instead,
   which was a lot simpler since we would rely on \mxp to provide our layout
   and other things for us through the \mxp compiler. This approach allowed us
   to use the full power of \mxp, including our own plug-in for automated
   layout. At this point, the optimization phase was also revisited, and
   leveraged to introduce specific XML tags around component groups that would
   trigger our automated layout plug-in.

  \section{Compiler optimizations}
   \label{compiler-optimizations}

   Since the conversion process resembles that of a compiler, it seemed logical
   at first to make automated layout a part of that process, as some kind of
   `compiler optimization'. During this phase in the process, the component
   tree would be manipulated and altered, with the express purpose to improve
   upon its structure and properties, so as to get a better end result. Our
   improvements in this case would then consist of the automated layout.

   As a first attempt, we tried to traverse the component tree, giving each
   object new coordinates and sizes based on their original coordinates and
   sizes, as well as the coordinates and sizes of objects around them, so that
   they would fit together on every slide as well as possible. This seemed an
   easy solution, but the results were sub-optimal. On top of that, we soon
   realised that we were in essence creating another template-based system that
   would generate slides and presentations based on predefined ratios and
   rules, which was exactly the opposite of what we were trying to do. As such,
   we abandoned this approach in favor of a constraint-based algorithm as
   described in section \ref{related-algorithms}.

   This involved a technique that at first sight may seem like yet another
   template system, but actually is completely different: defining constraints
   for every component, in the form of margins, maximum sizes and other limits,
   and then calculating a way to satisfy all constraints while fitting content
   together on each slide. The similarities with template-based systems exist
   in the presence of predefined constraints, ratios and rules, but the
   important difference is that these constraints are defined relative to the
   component itself, without specifying anything absolute about location or
   size. For example, we would retain the aspect ratio of an image, without
   specifying its size, so that the image may be scaled to accomodate other
   components in a dynamic layout. As another example, we might specify there
   needs to be a certain distance between a component and any other components,
   relative to its size. We could also specify a certain relation between
   components, ensuring components stay in each others vicinity, one should
   always be left of the other, no other components may be placed between them,
   etc. Using these rules, we would then programmatically calculate the best
   layout using those components, but without any other bias. These constraints
   would be based only on the original situation, never on any suggestions from
   us or other developers or authors, which makes all the difference with
   traditional template-based layouts.
 
   While this is clearly a better method, it turned out the compiler
   optimization phase was not the best place in the process to take care of
   this. While we had the necessary data to calculate the layout, we would have
   had to generate the layout along with the \mxp presentation, after which the
   presentation could not be altered anymore without breaking the layout. This
   defeated the purpose of exporting to \mxp, which was to allow the presenter
   to edit, extend and improve their presentation further using \mxp. What we
   needed was a way to get \mxp itself to generate the layout, even if we
   wanted to add components to the presentation afterwards, and even if we
   wanted to create a new \mxp presentation instead of starting from \ppt.
   After all, how would we convince people to drop \ppt for \mxp's automated
   layout capabilities if they could only use that functionality by starting
   from \ppt?

   In the end, we decided to change our approach again. We took the automated
   layout out of the conversion process, instead opting at this point in the
   process to only add the necessary layout triggers in the form of an
   enclosing XML tag around the components that would need to be included in
   the automated layout. As such, the generated \mxp XML would include those
   tags, and a plug-in (described in section \ref{mxp-plug-in}) would then
   generate the layout at runtime.
  
  \section{Using \mxp}

   One of the primary goals of \mxp is to separate content from layout,
   allowing the author of a presentation to focus on the content while \mxp
   takes care of the layout. The way it does this is currently mostly through
   the compiler, which decides the width, height and coordinates of content,
   relative to the container the content belongs to. The plug-ins responsible
   for handling components and containers currently don't mess with those
   settings, but technically, they could. The compiler decides the measurements
   and coordinates based on templates. The solution we were looking for was a
   layout engine that could take any content and put it in an appropriate
   layout without any directions from the user. As such, we had to enhance
   \mxp's layout engine to use constraints, based on the size of the content,
   and try to find an optimal position for every component it is given.

   \subsection{A \mxp plug-in}
    \label{mxp-plug-in}

    We did this by creating an invisible container plug-in. Containers are a
    way of grouping components, other containers, etc. in \mxp. This means they
    have control over their child elements, which gives us the opportunity to
    override the layout of those elements. A container plug-in thus allows us
    to implement our own layout system. Since it's a new element, it doesn't
    override existing elements as it would have done if we had, for example,
    rewritten the `slide' plug-in. The user can decide for themself whether or
    not to use it, and it can be used anywhere in the presentation: wrap the
    whole presentation in it, or just a small part, whichever works best for
    your purposes. It also won't break existing presentations that don't use
    it, while those presentations can very easily be altered to take advantage
    of it.

    An important aspect of this is that containers can be nested. This means we
    can create slide-based presentations, which can contain our
    \code{autolayout} container, which then contains the slide's contents, thus
    creating an optimal layout of the content per-slide. Another way of using
    it could be without slides, throwing all content together in one
    \code{autolayout} container, and letting it take care of the layout for the
    whole presentation at once. It should be noted here that the
    \code{autolayout} container makes each of its child nodes focusable
    separately, to compensate for arbitrary resizing it may perform on large
    objects in order to fit them next to other content, by using the focus
    functionality to automatically zoom into these components when necessary.
  
    We call it an \emph{invisible} container plug-in because it does not
    introduce any visual content, shape or indication for itself. Compare with
    the \code{slide} plug-in which obviously puts some kind of slide-look
    around the content it encompasses, and it becomes clear what we mean by
    this: although the content within is obviously affected by our plug-in,
    there is no visible indication of its presence to the audience.

    The plug-in uses the compiler's numbers to decide relative locations
    between components, as well as size ratios, and then finds a way to display
    those components in a way that the display order makes sense (or at least
    matches the intended order as closely as possible), that no overlapping
    occurs (since we don't have the animations that \ppt might have used to
    display one piece of information and then another on top of it), and
    resizing everything if necessary in order to fit within the specified
    container. While this may seem like a bad idea since content can get
    illegibly small this way, keep in mind that we can rely on the ZUI to focus
    on each component separately, or on groups of components, while \ppt
    obviously can only display the whole slide at once.

    In this manner we would generate \mxp presentations that were immediately
    usable, while also being adjustable; and on top of this, we would allow the
    automated layout process to be used in other \mxp presentations that were
    not originally converted from \ppt slides. The goal of this plug-in would
    thus be to provide automated layout functionality to \mxp presentations,
    and to allow any \mxp author to use it simply by putting an
    \code{<autolayout></autolayout>} container around the components they want
    the plug-in to act on. This approach has the additional advantage that the
    container can be used multiple times throughout the presentation, while
    also allowing other parts of the presentation to have a manual layout.
  
    Since it is possible to nest containers, which means a number of components
    could be grouped together in an \code{autolayout} container, then the
    result could be put into another \code{autolayout} container together with
    other components --- other \code{autolayout} containers, perhaps --- to
    generate an automated layout for an overview of the different groups.
    Compare it with traditional slideware, where components are grouped
    together in slides, then the slides might be put next to each other in an
    overview --- except our approach drops the slide boundaries, while still
    maintaining the ability to group components together to show a relation or
    link between them.

   \subsection{An automated layout algorithm}

    As discussed earlier, our first approach included an algorithm where
    content would be placed on slides according to certain rules, trying to
    attain a mythical `perfect' layout based on the golden ratio, symmetry,
    centering and other general guidelines we would find in advice on creating
    presentations. It turns out that, while following those guidelines as a
    human being is generally a good idea, a computer has different ways of
    calculating a good layout. The issue here can be compared to other problems
    in computer science; for example, people in the robotics department have
    tried for decades to create a robot that behaves exactly like a human
    being, and people in artificial intelligence have tried to create an AI
    that thinks like us. However, we've found time and time again that
    computers simply aren't very good at acting, thinking or being human, just
    like we aren't good at being computers. Making a computer act like a human
    makes it disadvantaged --- almost by definition, just like we are severely
    handicapped when we try to perform typically automatable, repetitive and/or
    math-intensive tasks. A computer's true power only shows when you let it do
    what it's good at, which is the repetitive stuff, the mathematically
    complex stuff, etc. Trying to make it generate presentation layouts like a
    human would, is asking for subpar results.

    If we approach this problem keeping in mind a computer's strength and
    weaknesses, we arrive at a different approach. This involves calculating
    sizes, ratio's, positions, margins and other numbers, of which the formulas
    are actually not too hard to come up with as a human, but which the
    execution is definitely more of a computer task. We start off by checking
    each component, and noting its original location and size. We then try to
    find components that are in proximity of each other, and figure out their
    original layout: above/below each other, next to each other, overlapping...
    Then, we try to put them together, possibly resizing them to match each
    other's sizes, and trying to match their original relative locations while
    introducing a certain rigidity, or consistency, by aligning them properly
    and puzzling them together as neatly as possible.
   
    This last part may sound weird, but it really is something to take into
    consideration, especially when the amount of components might be much
    bigger than what should fit on an average traditional slide. You could put
    all components in a row, just displaying them side-by-side, but that is not
    very aesthetically pleasing. Instead, we opted to try and keep components
    close to each other. This was finally achieved by finding the location
    closest to the starting point that would fit the component being
    considered, while still taking into account the earlier constraints about
    relative location and size. Thanks to the ZUI in \mxp, this makes for
    interesting layouts that still remain manageable, and provide a nice
    overview of all content when zoomed out.


 % vim:ts=1:et:nospell:spelllang=en_gb:ft=tex

 \chapter{Implementation}
  \label{implementation}

  To implement the \emph{ppt2mxp} conversion tool that is the subject of this
  thesis, we chose the Java programming language \citep{gosling-1}, version 8.
  Although the author has significant experience with lots of other, more
  interesting, more compelling, more fun languages, several reasons pushed us
  towards Java, the least of them being its ease of use. Of course, Java
  \emph{is} easy to use --- it would not have become as popular as it is
  nowadays if it wasn't. It has a fairly clear and logical syntax, a consistent
  structure, and an extensive standard library. At conception in 1995, its
  performance was abysmal, but through the years it has steadily improved and
  somewhere between Java 5 and 6 it became an industry standard.

  Quite a number of IDEs have been created to further improve developers'
  experience working with Java. Netbeans, Eclipse and IntelliJ come to mine,
  although there are many others, and of course you can still write Java using
  a standard (or advanced) text editor such as Notepad or VIM. While the author
  usually prefers the latter for any kind of text editing --- this very
  document was written entirely using VIM --- the weapon of choice when it
  comes to Java is currently IntelliJ. The way IntelliJ practically writes more
  than half of the code automatically for you is something no other IDE has
  been able to match. Naturally, this is the author's personal opinion and
  should not be seen as fact, but if you're looking for a new Java IDE, it's
  definitely worth checking out. The prospect of using IntelliJ for this thesis
  has definitely contributed to the decision of using Java. It should be noted
  that, had another Java IDE been required, this thesis might never have seen
  the light of day.

  The vast and extensive amount of libraries available for Java was obviously
  one of the more important reasons to make this choice. The existence of the
  Apache POI library (see section \ref{poi}) was a huge help in reaching our
  goal; without it, we would have had to figure out the very obfuscated .ppt
  file format structure, which undoubtedly would have taken up more time than
  was available to us. Other libraries like Spring, which allows the programmer
  to use and reuse components without writing complex systems to instantiate
  them, further increased our resolve to make Java our primary technology
  choice.

  However, Java is not the only technology used here. \mxp is written entirely
  in HTML5, so any tool that somehow relates to \mxp sooner or later needs to
  use HTML5 as well. The widely accepted HTML5 standard makes \mxp
  presentations highly portable and runnable on any device with a recent web
  browser, including smartphones and tablets \citep{roels-1}.

  In the following sections we discuss how the various technologies were used
  to create the \emph{ppt2mxp} tool.

  \section{Taking \ppt apart}
   \label{poi}

   When converting one file format into another, the first part of the process
   involves getting the data you need out of the original file. This can be
   very complicated, as some --- usually proprietary --- file formats are
   deliberately designed to discourage this. They obfuscate data, encrypt it,
   and structure it in illogical and unexpected ways, amongst other techniques.
   The \ppt file format unfortunately is such a format, as Microsoft wouldn't
   want to risk other companies making software that would work with \ppt
   files. Of course, over the years people have managed to crack the format,
   enabling the conversion of \ppt presentations into other formats, although
   the conversion does not usually guarantee to yield results that mimic the
   original version perfectly. Luckily, we don't want a perfect conversion, we
   want a better one.

   We found Apache POI library very helpful in this part of the implementation.
   The POI Library --- formerly ''Poor Obfuscation Implementation''
   \citep{sundaram-1} --- is a Java library that provides an API to access
   Microsoft document formats. The most mature (and most popular) part of it is
   HSSF, which stands for Horrible SpreadSheet Format, and which is used by
   Java developers worldwide to access Microsoft Excel spreadsheet data, as
   well as export data into Excel spreadsheets.

   For our purposes, we relied on HSLF (''Horrible SLideshow Format''), which
   gave us access to a \ppt presentation's contents in many ways. We could
   access all images at once, or every bit of text from the whole presentation,
   but the most interesting to us was the ability to access contents on a
   per-slide basis. This allowed us to loop over the presentation's slides,
   converting them one by one, by placing the contents of each slide in a \mxp
   slide equivalent.

   That was sadly not the end of it. While HSLF does give us access to all the
   text in a presentation, or per slide, it does not distinguish between
   'normal' text and bullet lists, for example. This was a difference we had to
   detect ourselves somehow. % TODO find out and explain how we do this.

   Another challenge was dealing with animations and other ways people managed
   to put way more content on one slide than would be advisable. The animations
   could not be transferred to \mxp since \mxp has its own set of transitions.
   It would technically be possible to implement additional animations as a
   separate plug-in for \mxp to provide the equivalents of the animations in
   \ppt*, but that is beyond the scope of this thesis. So we could not provide
   the same animations, but some people use those animations not just to show
   off but to actually show multiple pictures and blocks of text, one after the
   other, on the same slide. Without animations, this content would either not
   be visible or it would become a serious layout issue in \mxp. Our solution
   proposes to limit the amount of objects one slide can contain, and any
   additional content should be put on extra slides automatically. A downside
   of this is that we currently have no way of guessing the correct order in
   which the content should appear, so what may have been an intrinsic
   choreography of pictures in \ppt may become an incoherent jumble of images
   in \mxp. Another solution would be to scale all content until it all fits
   next to each other on one slide, and then rely on the zoomable interface to
   show the pictures one by one, but in this case the same problem with order
   of appearance manifests itself. In the end, we decided it would be best to
   accept that no conversion algorithm is going to be perfect, and the author
   can always manually change the order around after the conversion is done.

  \section{Generating \mxp}

%   TODO generating

   \subsection{Plain HTML5}

    Since the \mxp compiler was not functional during most of this thesis'
    implementation, we decided to generate an html file much like the \mxp
    compiler would, including the \mxp JavaScript library and plug-ins. This
    required us to first find out how \mxp works on the inside, which proved to
    be a steep learning curve but gave us more insight into the software than
    we would've gotten if we only had to generate \mxp XML and leave the rest
    to the compiler.

   \subsection{\mxp XML}

%    TODO XML

  \section{Creating layouts}

%   TODO layout

   \subsection{Using constraints}

%    TODO constraints

   \subsection{Other ways}

%    TODO other ways

%   15:50 <omega> zeg, ik zit nu al een hele tijd thesis te schrijven en de laatste paar dagen vooral te zeveren over layout, maar intussen doe ik nog ni echt iets van layout, met t gedacht van ik schrijf daar binnenkort ne mindxpres plug-in voor en klaar
%   15:51 <omega> maar wordt layout momenteel eig ni mostly door de compiler gedaan?
%   15:52 <omega> ben zo eens naar de presentation.js libs en code gaan kijken, en ik zie ni direct een manier om ne plug-in layout te laten doen, aangezien plug-ins mostly component-specifiek zijn en ni alle componenten kunnen aansturen
%   15:53 <omega> dus klopt het dat ik ofwel de compiler moet aanpassen, ofwel presentation.js hacken om dat soort plug-ins toe te laten?
%   15:53 <omega> of laat het dat soort plug-ins al toe maar zijn er gewoon nog geen?
%   16:30 <omega> de 'structured' plug-in doet wel layout van slides, maar binnen die slides zie ik niet meteen een systeem dat layout regelt, met templates of otherwise, het pakt gewoon de coordinaten en afmetingen die de compiler bepaald heeft
%   16:31 <omega> al zou die slide plug-in wel *kunnen* prutsen met die layout... dus mss moet ik gwn de slide plug-in uitbreiden/hacken/vervangen
%   13:22 <reinout> ik zou een container plug-in maken
%   13:22 <reinout> gelijk de slide
%   13:22 <reinout> maar dan onzichtbaar
%   13:23 <reinout> want containers kan je nesten
%   13:23 <reinout> dus een slide kan bv uw layout container bevatten, die dan de children een layout geeft
%   13:23 <reinout> maar op die manier is uw layout ding bruikbaar buiten slides
%   13:24 <reinout> (alternatief was uw layout stuff in de slide plug-in steken)
%   13:31 <omega> oeh, cool idee indeed, beter dan de slide plug-in abusen


 % vim:ts=1:et:nospell:spelllang=en_gb:ft=tex

 \chapter{Conclusions and Future Work}

%  TODO conclusion

  \section{Contribution}

%   TODO why is this useful at all

   Converting \ppt presentations into \mxp is now possible and easy.

   Automated layout has not been available in presentations until now. \mxp is
   the first presentation system that doesn't require predefined templates, nor
   manual layout tweaking by the enduser, instead letting people focus on the
   content while the software takes care of the rest.

  \section{Future Work}

%   TODO what should the next thesis slave still fix

   Other formats: extend the convertor tool to convert Keynote, prezzi, pptx,
   \ldots

   Integrate the convertor into an \mxp editor

   Improve the automated layout algorithm beyond constraints using learning AI,
   training it on good/bad layouts, neural network...

   Implement extreme zooming into \mxp using the CSS3 perspective property
   along with translation along the z-axis to utilize the full power of the 3d
   space.



 \newpage

 \bibliographystyle{IEEEtranN}
 \bibliography{db}

\end{document}

