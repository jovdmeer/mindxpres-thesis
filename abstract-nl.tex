% vim:ts=1:et:nospell:spelllang=en_gb:ft=tex

 \chapter*{Samenvatting}

  Klassieke diavoorstellingen worden wereldwijd gebruikt om kennis over te
  brengen en te delen. \ppt* is het populairste en meest bekende programma in
  dit gebied. Spijtig genoeg heeft deze software weinig evolutie gekend sinds
  ze het licht zag in de jaren 1980, ondanks een ongelooflijke stortvloed aan
  vernieuwingen in bijna elk ander aspect van informatie- en
  communicatietechnologie. Het basisconcept van digitale presentaties is steeds
  hetzelfde gebleven, gebaseerd op de oorspronkelijke fysieke overhead- en
  diaprojecties. De beperkingen in grootte zorgen vaak voor een minder dan
  ideale weergave van informatie, waardoor het moeilijk wordt voor de
  presentator om alles goed te kunnen uitleggen, alsook voor het publiek om
  alles te begrijpen.

  Presentatoren ondervinden vaak, tot zelfs gewoonlijk, problemen met het
  schikken van hun inhoud binnen dit formaat op een manier die duidelijk,
  overzichtelijk en esthetisch verantwoord is. Het grootste deel van de tijd
  nodig om een presentatie te maken wordt verspild aan het zoeken van een goede
  layout voor de voorziene informatie, vaak dan nog met weinig positief
  resultaat. Professionele designers werken jarenlang aan templates voor dit
  soort presentatiesoftware, in een poging om een alles passende oplossing te
  voorzien voor onderdelen die --- weinig verrassend --- meestal niet voldoen
  aan de regels die erop toegepast worden.

% TODO
  \mxp is a presentation tool that brings a shift of paradigms in authoring and
  delivering presentations. It provides an extensible platform that allows a
  presenter to focus on the content of their presentation, while \mxp takes
  care of the visualisation. It consists almost entirely of plug-ins to process
  and visualise various content types, and allows the addition of new plug-ins
  to introduce new functionality as needed. The only limit in this system is
  the plug-in developer's imagination.

  In this thesis, we propose a tool and an approach to convert existing \ppt
  presentations into \mxp presentations, in the hopes of convincing \ppt users
  to switch to \mxp by showing them the possibilities of \mxp using their own
  content. As a second goal, we try to replace the default template-based
  layout system of \mxp with a layout engine that generates an ideal layout
  based on the content of a presentation.

%  TODO translate from abstract-en

