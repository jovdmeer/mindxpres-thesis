% vim:ts=1:et:nospell:spelllang=en_gb:ft=tex

 \chapter*{Samenvatting}

  Klassieke presentaties worden wereldwijd gebruikt om kennis over te
  brengen en te delen. \ppt* is het populairste en meest bekende programma in
  dit gebied. Spijtig genoeg heeft deze software weinig evolutie gekend sinds
  ze het licht zag in de jaren 1980, ondanks een ongelooflijke stortvloed aan
  vernieuwingen in bijna elk ander aspect van informatie- en
  communicatietechnologie. Het basisconcept van digitale presentaties is steeds
  hetzelfde gebleven, gebaseerd op de oorspronkelijke fysieke overhead- en
  diaprojecties. De beperkingen in grootte zorgen vaak voor een minder dan
  ideale weergave van informatie, waardoor het moeilijk wordt voor de
  presentator om alles goed te kunnen uitleggen, alsook voor het publiek om
  alles te begrijpen.

  Presentatoren ondervinden vaak, tot zelfs gewoonlijk, problemen met het
  schikken van hun inhoud binnen dit formaat op een manier die duidelijk,
  overzichtelijk en esthetisch verantwoord is. Het grootste deel van de tijd
  nodig om een presentatie te maken wordt verspild aan het zoeken van een goede
  layout voor de voorziene informatie, vaak dan nog met weinig positief
  resultaat. Professionele designers werken jarenlang aan templates voor dit
  soort presentatiesoftware, in een poging om een alles passende oplossing te
  voorzien voor onderdelen die --- weinig verrassend --- meestal niet voldoen
  aan de regels die erop toegepast worden.

  \mxp is een programma dat een paradigmaverschuiving in het bewerken en geven
  van presentaties met zich meebrengt. Het voorziet een uitbreidbaar platform
  dat een presentator toelaat zich te focussen op de inhoud van zijn
  presentatie, terwijl \mxp de visualisatie voor zijn rekening neemt. Het
  geheel bestaat bijna volledig uit plug-ins die verschillende soorten
  informatie verwerken en visualiseren, en maakt het mogelijk om nieuwe
  plug-ins toe te voegen om nieuwe functionaliteit beschikbaar te maken wanneer
  nodig. De verbeelding van de plug-in ontwikkelaar is de enige grens aan dit
  systeem.

  In deze thesis stellen we een programma en aanpak voor waarmee bestaande \ppt
  presentaties kunnen omgezet worden in \mxp presentaties, met de bedoeling om
  \ppt gebruikers te overtuigen om \mxp te gaan gebruiken door hen de
  mogelijkheden van \mxp te tonen met hun eigen presentatie-inhoud. Als
  bijkomend --- maar niet minder belangrijk --- doel vervangen we het bestaande
  layout systeem van \mxp, gebaseerd op templates, door een layout algoritme
  dat een ideale layout genereert gebaseerd op de onderdelen van een
  presentatie.

