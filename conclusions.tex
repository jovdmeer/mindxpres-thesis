% vim:ts=1:et:nospell:spelllang=en_gb:ft=tex

 \chapter{Conclusions and Future Work}

  \ppt* remains the most popular presentation tool worldwide. We believe this
  is mostly because people are generally afraid of change, and would rather
  stick to their habits. Since all of their existing work is stored in the \ppt
  format, they keep using \ppt to access that content as well as create new
  presentations. On top of that, the process of creating a presentation is
  still heavily burdened by the layout of the content we want to present.
  Layout is an important and vastly underrated aspect of presentations in
  general, which uses up an astonishing amount of time during the creation of a
  presentation.

  \section{Contribution}

   We proposed an approach for converting existing \ppt presentations into \mxp
   presentations, along with a way to take control over layout away from the
   author and improve upon flawed human design by programmatically calculating
   ideal content placement and size. We delivered a proof of concept
   implementation that puts this approach into practice, first letting us show
   a \ppt presentation's content in \mxp, then showing us the possibility of
   applying an automated layout algorithm to that an any other \mxp content at
   will.

   Considering the first part of this thesis, which consists of the conversion
   between \ppt and \mxp, we can conclude that conversion from any other
   presentation format into \mxp is a feasible concept. Closed-source formats
   will obviously be more of a challenge than their open-sourced cousins,
   especially if no API has been created for them as we had the fortune with
   \ppt and Apache's POI/HSLF implementation. That said, open-source formats
   may be more easy to take apart but if no API exists for them it would still
   require a substantial amount of effort. Having an existing API readily
   available has definitely helped us a great deal in our efforts.

   Converting these other formats into \mxp remains an important goal in the
   endeavour to raise awareness and increase popularity of \mxp's features and
   possibilities. Our implementation provides a way to convert \ppt .ppt
   presentations, but \ppt* has switched to using the Office Open XML-based
   .pptx format in recent years, so newer \ppt presentations cannot currently
   be imported into the \mxp system.

   Our implementation is written in a way that should make it straightforward
   to adapt for other formats, provided there is a way to get the separate
   components out of those formats. If slide-based conversion is desired, then
   obviously a way to extract the components on a per-slide basis is also
   required. Additionally, our approach using an intermediary form during the
   conversion process allows for adaptation of the tool to generate other
   output formats as well. The current implementation generates HTML5 which
   includes the \mxp standard library and a set of plugins, but the original
   goal of generating \mxp XML files should be easily attainable; the only
   reason we did not implement this was the unavailability of the \mxp
   compiler, which made it impossible for us to test the generated XML files.

   As for the second part of this thesis, concerning the automated generation
   of presentation layouts, we have discussed why this is necessary. When
   creating traditional slideware, as well as using more advanced and modern
   presentation tools, layout remains a problem that for many presenters
   becomes the biggest timesink in their work. On top of that, the layout they
   create isn't always a good one, and bad layout has been proven to have
   negative impact on the effectiveness of a presentation. As such, having a
   way to automatically generate a layout would save a lot of time while also
   improving the information transfer effectiveness of presentations.

   We have demonstrated such an automated layout mechanism based on theory and
   research found in related works, which we adapted and improved upon for our
   purposes. Our constraint-based approach considers every component
   separately, to combine all components into a layout where no overlapping
   exists, components can be grouped together, clear margins are put in between
   content and surrounding limits in the form of slides and other fixed-size
   containers are respected.

   The implementation of this mechanism is far from complete, and may still be
   improved upon in several ways, which will be discussed in section
   \ref{future}. It does however provide the most basic form of automated
   layout, which may not always succeed in generating an aesthetically pleasing
   layout but at least attempts to combine content in a way that makes the
   content easy to focus on, thus increasing effectiveness of the presentation.
   It also succeeds in letting the presenter focus on the content rather than
   the layout. It thus reduces stress and arguably increases quality of
   presentations, especially when we look at time spent creating the
   presentation versus its effectiveness.

  \section{Future Work}
   \label{future}

   In this thesis we have presented a proof-of-concept implementation of both a
   tool to convert \ppt presentations into \mxp, and an algorithm for
   generating an objectively effective layout. Due to the limited time
   available for this thesis, we were not able to go into the finer details of
   these tools, and the result can seem rather unpolished. However, within this
   limited timeframe we did deliver a solid core containing the most important
   features, in a way that allows future research to improve upon it and easily
   add any missing details.

   \subsection{Other formats}

    Our conversion tool currently allows to convert \ppt .ppt files into \mxp
    presentations, bypassing the \mxp compiler. The tool internally uses an
    intermediary structure to store the presentation's content, and this
    facilitates the implementation of conversion tools for other formats. As
    such, it might be a good idea to extend the tool to convert other popular
    formats like \ppt .pptx files, Apple Keynote presentations and many others.

    It would also be a good idea to change the output of the conversion tool to
    generate \mxp XML files to compile further using the \mxp compiler. In and
    of itself this would not seem advantageous, but with \mxp IDE's and other
    editing tools in mind it would be better to have XML files which would be
    editable using those tools, rather than raw HTML5 which presumably would
    not be readily available in any IDE.

   \subsection{Integration}

    Speaking of editors, it would be interesting to integrate the conversion
    tool into such an editor. This would allow \mxp users to just open their
    \ppt files in the \mxp editor, immediately providing access to its contents
    and letting the user edit the presentation as if it had always been an \mxp
    presentation. This would greatly improve usability of the conversion tool
    as well, since it currently does not have a graphical user interface and
    thus needs to be invoked from the command line.

   \subsection{Improving the automated layout}

    There are many ways in which the automated layout algorithm may yet be
    improved. Jock Mackinlay's work \citep{mackinlay-1} includes significant
    research on how to use artificial intelligence to create effective
    graphical visualisations. Combining his work with ours could potentially
    improve the results of our algorithm. An interesting angle here might be
    the use of a learning AI, which can be trained on sets of good and bad
    layouts, or observe the user's actions and try to mimic their behaviour.

