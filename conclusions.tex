% vim:ts=1:et:nospell:spelllang=en_gb:ft=tex

 \chapter{Conclusions and Future Work}

%  TODO conclusion

  We proposed an approach for converting existing \ppt presentations into \mxp
  presentations, along with a way to take control over layout away from the
  author and improve upon flawed human design by programmatically calculating
  ideal content placement and size. We delivered a proof of concept
  implementation that puts this approach into practice, first letting us show a
  \ppt presentation's content in \mxp, then showing us the possibility of
  applying an automated layout algorithm to that an any other \mxp content at
  will.

  Considering the first part of this thesis, which consists of the conversion
  between \ppt and \mxp, we can conclude that conversion from any other
  presentation format into \mxp is a feasible concept. Closed-source formats
  will obviously be more of a challenge than their open-sourced cousins,
  especially if no API has been created for them as we had the fortune with
  \ppt and Apache's POI/HSLF implementation. That said, open-source formats may
  be more easy to take apart but if no API exists for them it would still
  require a substantial amount of effort. Having an existing API readily
  available has definitely helped us a great deal in our efforts.

  Converting these other formats into \mxp remains an important goal in the
  endeavour to raise awareness and increase popularity of \mxp's features and
  possibilities. Our implementation provides a way to convert \ppt .ppt
  presentations, but \ppt* has switched to using the Office Open XML-based
  .pptx format in recent years, so newer \ppt presentations cannot currently be
  imported into the \mxp system.

  Our implementation is written in a way that should make it straightforward to
  adapt for other formats, provided there is a way to get the separate
  components out of those formats. If slide-based conversion is desired, then
  obviously a way to extract the components on a per-slide basis is also
  required. Additionally, our approach using an intermediary form during the
  conversion process allows for adaptation of the tool to generate other output
  formats as well. The current implementation generates HTML5 which includes
  the \mxp standard library and a set of plugins, but the original goal of
  generating \mxp XML files should be easily attainable; the only reason we did
  not implement this was the unavailability of the \mxp compiler, which made it
  impossible for us to test the generated XML files.

  As for the second part of this thesis, concerning the automated generation of
  presentation layouts, we have discussed why this is necessary. When creating
  traditional slideware, as well as using more advanced and modern presentation
  tools, layout remains a problem that for many presenters becomes the biggest
  timesink in their work. On top of that, the layout they create isn't always a
  good one, and bad layout has been proven to have negative impact on the
  effectiveness of a presentation. As such, having a way to automatically
  generate a layout would save a lot of time while also improving the
  information transfer effectiveness of presentations.

  We have demonstrated such an automated layout mechanism based on theory and
  research found in related works, which we adapted and improved upon for our
  purposes. Our constraint-based approach considers every component separately,
  to combine all components into a layout where no overlapping exists,
  components can be grouped together, clear margins are put in between content
  and surrounding limits in the form of slides and other fixed-size containers
  are respected.

  The implementation of this mechanism is far from complete, and may still be
  improved upon in several ways, which will be discussed in section
  \ref{future}. It does however provide the most basic form of automated
  layout, which may not always succeed in generating an aesthetically pleasing
  layout but at least attempts to combine content in a way that makes the
  content easy to focus on, thus increasing effectiveness of the presentation.
  It also succeeds in letting the presenter focus on the content rather than
  the layout. It thus reduces stress and arguably increases quality of
  presentations, especially when we look at time spent creating the
  presentation versus its effectiveness.

  \section{Contribution}

%   TODO why is this useful at all

   Converting \ppt presentations into \mxp is now possible and easy.

   Automated layout has not been available in presentations until now. \mxp is
   the first presentation system that doesn't require predefined templates, nor
   manual layout tweaking by the enduser, instead letting people focus on the
   content while the software takes care of the rest.

  \section{Future Work}
   \label{future}

%   TODO what should the next thesis slave still fix

   Other formats: extend the convertor tool to convert Keynote, prezzi, pptx,
   \ldots

   Other formats: refactor the convertor tool to separate output from the
   component classes, concentrating it in a Writer class that implements a
   Writer interface, so that the Writer class can easily be replaced with
   another one to generate a different format.

   Integrate the convertor into an \mxp editor

   Improve the automated layout algorithm beyond constraints using learning AI,
   training it on good/bad layouts, neural network...

   Implement extreme zooming into \mxp using the CSS3 perspective property
   along with translation along the z-axis to utilize the full power of the 3d
   space.

