% vim:ts=1:et:nospell:spelllang=en_gb:ft=tex

 \chapter{Implementation}
  \label{implementation}

  To implement the \emph{ppt2mxp} conversion tool that is the subject of this
  thesis, we chose the Java programming language \citep{gosling-1}, version 8.
  Although the author has significant experience with lots of other, more
  interesting, more compelling, more fun languages, several reasons pushed us
  towards Java, the least of them being its ease of use. Of course, Java
  \emph{is} easy to use --- it would not have become as popular as it is
  nowadays if it wasn't. It has a fairly clear and logical syntax, a consistent
  structure, and an extensive standard library. At conception in 1995, its
  performance was abysmal, but through the years it has steadily improved and
  somewhere between Java 5 and 6 it became an industry standard.

  Quite a number of IDEs have been created to further improve developers'
  experience working with Java. Netbeans, Eclipse and IntelliJ come to mine,
  although there are many others, and of course you can still write Java using
  a standard (or advanced) text editor such as Notepad or VIM. While the author
  usually prefers the latter for any kind of text editing --- this very
  document was written entirely using VIM --- the weapon of choice when it
  comes to Java is currently IntelliJ. The way IntelliJ practically writes more
  than half of the code automatically for you is something no other IDE has
  been able to match. Naturally, this is the author's personal opinion and
  should not be seen as fact, but if you're looking for a new Java IDE, it's
  definitely worth checking out. The prospect of using IntelliJ for this thesis
  has definitely contributed to the decision of using Java. It should be noted
  that, had another Java IDE been required, this thesis might never have seen
  the light of day.

  The vast and extensive amount of libraries available for Java was obviously
  one of the more important reasons to make this choice. The existence of the
  Apache POI library (see section \ref{poi}) was a huge help in reaching our
  goal; without it, we would have had to figure out the very obfuscated .ppt
  file format structure, which undoubtedly would have taken up more time than
  was available to us. Other libraries like Spring, which allows the programmer
  to use and reuse components without writing complex systems to instantiate
  them, further increased our resolve to make Java our primary technology
  choice.

  However, Java is not the only technology used here. \mxp is written entirely
  in HTML5, so any tool that somehow relates to \mxp sooner or later needs to
  use HTML5 as well. The widely accepted HTML5 standard makes \mxp
  presentations highly portable and runnable on any device with a recent web
  browser, including smartphones and tablets \citep{roels-1}.

  In the following sections we discuss how the various technologies were used
  to create the \emph{ppt2mxp} tool.

  \section{Taking \ppt apart}
   \label{poi}

   When converting one file format into another, the first part of the process
   involves getting the data you need out of the original file. This can be
   very complicated, as some --- usually proprietary --- file formats are
   deliberately designed to discourage this. They obfuscate data, encrypt it,
   and structure it in illogical and unexpected ways, amongst other techniques.
   The \ppt file format unfortunately is such a format, as Microsoft wouldn't
   want to risk other companies making software that would work with \ppt
   files. Of course, over the years people have managed to crack the format,
   enabling the conversion of \ppt presentations into other formats, although
   the conversion does not usually guarantee to yield results that mimic the
   original version perfectly. Luckily, we don't want a perfect conversion, we
   want a better one.

   We found Apache POI library very helpful in this part of the implementation.
   The POI Library --- formerly ''Poor Obfuscation Implementation''
   \citep{sundaram-1} --- is a Java library that provides an API to access
   Microsoft document formats. The most mature (and most popular) part of it is
   HSSF, which stands for Horrible SpreadSheet Format, and which is used by
   Java developers worldwide to access Microsoft Excel spreadsheet data, as
   well as export data into Excel spreadsheets.

   For our purposes, we relied on HSLF (''Horrible SLideshow Format''), which
   gave us access to a \ppt presentation's contents in many ways. We could
   access all images at once, or every bit of text from the whole presentation,
   but the most interesting to us was the ability to access contents on a
   per-slide basis. This allowed us to loop over the presentation's slides,
   converting them one by one, by placing the contents of each slide in a \mxp
   slide equivalent.

   That was sadly not the end of it. While HSLF does give us access to all the
   text in a presentation, or per slide, it does not distinguish between
   'normal' text and bullet lists, for example. This was a difference we had to
   detect ourselves somehow. % TODO find out and explain how we do this.

   Another challenge was dealing with animations and other ways people managed
   to put way more content on one slide than would be advisable. The animations
   could not be transferred to \mxp since \mxp has its own set of transitions.
   It would technically be possible to implement additional animations as a
   separate plug-in for \mxp to provide the equivalents of the animations in
   \ppt*, but that is beyond the scope of this thesis. So we could not provide
   the same animations, but some people use those animations not just to show
   off but to actually show multiple pictures and blocks of text, one after the
   other, on the same slide. Without animations, this content would either not
   be visible or it would become a serious layout issue in \mxp. Our solution
   proposes to limit the amount of objects one slide can contain, and any
   additional content should be put on extra slides automatically. A downside
   of this is that we currently have no way of guessing the correct order in
   which the content should appear, so what may have been an intrinsic
   choreography of pictures in \ppt may become an incoherent jumble of images
   in \mxp. Another solution would be to scale all content until it all fits
   next to each other on one slide, and then rely on the zoomable interface to
   show the pictures one by one, but in this case the same problem with order
   of appearance manifests itself. In the end, we decided it would be best to
   accept that no conversion algorithm is going to be perfect, and the author
   can always manually change the order around after the conversion is done.

  \section{Generating \mxp}

%   TODO generating

   \subsection{Plain HTML5}

    Since the \mxp compiler was not functional during most of this thesis'
    implementation, we decided to generate an html file much like the \mxp
    compiler would, including the \mxp JavaScript library and plug-ins. This
    required us to first find out how \mxp works on the inside, which proved to
    be a steep learning curve but gave us more insight into the software than
    we would've gotten if we only had to generate \mxp XML and leave the rest
    to the compiler.

   \subsection{\mxp XML}

%    TODO XML

  \section{Creating layouts}

%   TODO layout

   \subsection{Using constraints}

%    TODO constraints

   \subsection{Other ways}

%    TODO other ways

%   15:50 <omega> zeg, ik zit nu al een hele tijd thesis te schrijven en de laatste paar dagen vooral te zeveren over layout, maar intussen doe ik nog ni echt iets van layout, met t gedacht van ik schrijf daar binnenkort ne mindxpres plug-in voor en klaar
%   15:51 <omega> maar wordt layout momenteel eig ni mostly door de compiler gedaan?
%   15:52 <omega> ben zo eens naar de presentation.js libs en code gaan kijken, en ik zie ni direct een manier om ne plug-in layout te laten doen, aangezien plug-ins mostly component-specifiek zijn en ni alle componenten kunnen aansturen
%   15:53 <omega> dus klopt het dat ik ofwel de compiler moet aanpassen, ofwel presentation.js hacken om dat soort plug-ins toe te laten?
%   15:53 <omega> of laat het dat soort plug-ins al toe maar zijn er gewoon nog geen?
%   16:30 <omega> de 'structured' plug-in doet wel layout van slides, maar binnen die slides zie ik niet meteen een systeem dat layout regelt, met templates of otherwise, het pakt gewoon de coordinaten en afmetingen die de compiler bepaald heeft
%   16:31 <omega> al zou die slide plug-in wel *kunnen* prutsen met die layout... dus mss moet ik gwn de slide plug-in uitbreiden/hacken/vervangen
%   13:22 <reinout> ik zou een container plug-in maken
%   13:22 <reinout> gelijk de slide
%   13:22 <reinout> maar dan onzichtbaar
%   13:23 <reinout> want containers kan je nesten
%   13:23 <reinout> dus een slide kan bv uw layout container bevatten, die dan de children een layout geeft
%   13:23 <reinout> maar op die manier is uw layout ding bruikbaar buiten slides
%   13:24 <reinout> (alternatief was uw layout stuff in de slide plug-in steken)
%   13:31 <omega> oeh, cool idee indeed, beter dan de slide plug-in abusen
