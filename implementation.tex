% vim:ts=1:et:nospell:spelllang=en_gb:ft=tex

 \chapter{Implementation}

  \section{Taking \ppt apart}

   We found Apache POI library very helpful in this part of the implementation.
   The POI Library --- formerly ''Poor Obfuscation Implementation''
   \citep{sundaram-1} --- is a Java library that provides an API to access
   Microsoft document formats. The most mature (and most popular) part of it is
   HSSF, which stands for Horrible SpreadSheet Format, and which is used by
   Java developers worldwide to access Microsoft Excel spreadsheet data. 

   For our purposes, we relied on HSLF (''Horrible SLideshow Format''), which
   gave us access to a \ppt presentation's contents in many ways. We could
   access all images at once, or every bit of text from the whole presentation,
   but the most interesting to us was the ability to access contents on a
   per-slide basis.

   This allowed us to loop over the presentation's slides, converting them one
   by one, by placing the contents of each slide in a \mxp slide equivalent.

  \section{Generating \mxp}

   \subsection{Plain HTML5}

    Since the \mxp compiler was not functional during most of this thesis'
    implementation, we decided to generate an html file much like the \mxp
    compiler would, including the \mxp JavaScript library and plugins. This
    required us to first find out how \mxp works on the inside, which proved to
    be a steep learning curve but gave us more insight into the software than
    we would've gotten if we only had to generate \mxp XML and leave the rest
    to the compiler.

   \subsection{\mxp XML}

  \section{Creating layouts}

   \subsection{Using constraints}

   \subsection{Other ways}

