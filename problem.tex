% vim:ts=1:et:nospell:spelllang=en_gb:ft=tex

 \chapter{Problem statement}

  Computers, software and digital content are everywhere. Everything you use
  nowadays is somehow related to computers and electronics, and if it isn't, it
  probably will be soon. This may be a bit of hyperbole, but there's a core of
  truth in it. If you think about it, more and more things have become and are
  becoming some kind of computer. Coffee machines used to be simple machines
  that heated water and let it drip over coffee grounds; now there are coffee
  machines that are connected to the internet, and can be turned on remotely
  from your smartphone. That smartphone itself is an incredible evolution as
  well: just 20 years ago, phones were analog devices, and you could use them
  to call people and nothing more. Today, our phone does a lot more than that,
  so much more that calling has actually become a minor feature to most people.

  Content is going the same way. Photos used to be on a special film, and could
  be `developed' onto special paper through a proces involving a dark room and
  several chemicals. Movies existed on a projection film, newspapers were
  actually made of paper and music was available on vinyl disks with grooves
  that matched the sound waves. All of this content has been digitized since.
  This means of course that you can see or hear it using a computer, like you
  would've seen it without a computer before, but on top of that it means the
  content can be much more dynamic. You can link it to other content, you can
  make it respond to your actions, you can discuss it with people around the
  world. Digitized content allows for interactivity, so that the audience is no
  longer a passive onlooker but an active participant.

  It is no surprise, then, that slideshow presentations have evolved from the
  original dia's or overhead projection slides into a digital form as well.
  Except, until recently the evolution stopped there. Slideshows did not become
  interactive, and the audience remained passive onlookers watching a series of
  images projected on a screen or a wall. The presenter told a story, and the
  audience listened. Often during or at the end of the presentation there would
  be a chance to ask questions, but those questions could only be answered
  vocally by the presenter. If the question needed any visual explanation, the
  slideshow would not be able to help. We had digital slides, but the
  difference with the physical slides was neglectable.

  In our eyes, the culprit for this is Microsoft's \ppt. This software package
  took the world by storm, making it possible for everyone with a computer to
  make digital slideshows, which was impressive at the time. However, \ppt*
  never really evolved beyond that, except for adding some simpel animations
  allowing text to `fly' into view. Since it was --- and still is! --- the
  dominant player in the world of slideware with over 90\% market share, this
  apathy towards change firmly rooted slideware in the concepts of the past.
  Luckily, a few years ago some people realized this and decided to take
  matters into their own hands. They stepped away from the classic slide
  format, allowing for any kind of layout, combined with zoomable interfaces
  and other methods of displaying data. One such alternative is \mxp, created
  in the WISE lab at the VUB.

  \mxp is based on a plug-in architecture. Plug-ins can do anything from
  arranging data in a certain way to letting the audience control the
  slideshow. Virtually anything is possible if you only implement it, and
  implementing it is fairly simpel if you know a bit about web development as
  the whole thing is written in HTML5. Other software packages have plug-ins
  too of course, but they have a limited set of functionality available to
  them, they're not as easy to implement, and most importantly: they're bound
  by the same slide format used since overhead projections.

  However, even with the new alternatives, \ppt* remains the most-used slideshow
  software. People keep using it because it's familiar, they've used it
  hundreds of times before and as such all their existing work is viewable only
  through \ppt. Switching to a new software package is hard. This thesis aims
  to make the transition easier, by providing a way to convert existing \ppt
  presentations into \mxp. On top of that, we try to find a way to immediately
  release the transferred content from the confines of classic slides, by
  instead automatically figuring out the best possible layout for the content
  we extracted from the original \ppt file.

  \section{Terminology}

   TODO use fancy words to explain other fancy words

   \paragraph{Slideware}

  \section{Stuff we want to solve}

   TODO change this section's title

   TODO talk about how automatic layout is a real problem in today's world

