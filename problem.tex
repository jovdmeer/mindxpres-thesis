% vim:ts=1:et:nospell:spelllang=en_gb:ft=tex

 \chapter{Problem statement}

  Computers, software and digital content are everywhere. Everything we use
  nowadays is somehow related to computers and electronics, and if it isn't, it
  probably will be soon. This may be a bit of hyperbole, but there's a core of
  truth in it. If you think about it, more and more things have become and are
  becoming some kind of computer. Coffee machines used to be simple machines
  that heated water and let it drip over coffee grounds; now there are coffee
  machines that are connected to the internet, and can be turned on remotely
  from your smartphone. That smartphone itself is an incredible evolution as
  well: just 20 years ago, phones were analog devices, and you could use them
  to call people and nothing more. Today, our phone does a lot more than that,
  so much more that calling has actually become a minor feature to most people.

  Content is going the same way. Photos used to be on a special film, and could
  be `developed' onto special paper through a proces involving a dark room and
  several chemicals. Movies existed on a projection film, newspapers were
  actually made of paper and music was available on vinyl disks with grooves
  that matched the sound waves. All of this content has been digitized since.
  This means of course that you can see or hear it using a computer, like you
  would've seen it without a computer before, but on top of that it means the
  content can be much more dynamic. You can link it to other content, you can
  make it respond to your actions, you can discuss it with people around the
  world. Digitized content allows for interactivity, so that the audience is no
  longer a passive onlooker but an active participant.

  It is no surprise, then, that slideshow presentations have evolved from the
  original dia's or overhead projection slides into a digital form as well.
  Except, until recently the evolution stopped there. Slideshows did not become
  interactive, and the audience remained passive onlookers watching a series of
  images projected on a screen or a wall. The presenter told a story, and the
  audience listened. Often during or at the end of the presentation there would
  be a chance to ask questions, but those questions could only be answered
  vocally by the presenter. If the question needed any visual explanation, the
  slideshow would not be able to help. We had digital slides, but the
  difference with the physical slides was neglectable.

  In our eyes, the culprit for this is Microsoft's \ppt. This software package
  took the world by storm, making it possible for everyone with a computer to
  make digital slideshows, which was impressive at the time. However, \ppt*
  never really evolved beyond that, except for adding some simpel animations
  allowing text to `fly' into view (among others). Since it was --- and still
  is! --- the dominant player in the world of slideware with over 90\% market
  share, this apathy towards change firmly rooted slideware in the concepts of
  the past.  Luckily, a few years ago some people realized this and decided to
  take matters into their own hands. They stepped away from the classic slide
  format, allowing for any kind of layout, combined with zoomable interfaces
  and other methods of displaying data. One such alternative is \mxp, created
  in the WISE lab at the VUB.

  \mxp is based on a plug-in architecture. Plug-ins can do anything from
  arranging data in a certain way to letting the audience control the
  slideshow. Virtually anything is possible if you only implement it, and
  implementing it is fairly simpel if you know a bit about web development as
  the whole thing is written in HTML5. Other software packages have plug-ins
  too of course, but they have a limited set of functionality available to
  them, they're not as easy to implement, and most importantly: they're bound
  by the same slide format used since overhead projections.

  However, even with the new alternatives, \ppt* remains the most-used slideshow
  software. People keep using it because it's familiar, they've used it
  hundreds of times before and as such all their existing work is viewable only
  through \ppt. Switching to a new software package is hard. This thesis aims
  to make the transition easier, by providing a way to convert existing \ppt
  presentations into \mxp. On top of that, we try to find a way to immediately
  release the transferred content from the confines of classic slides, by
  instead automatically figuring out the best possible layout for the content
  we extracted from the original \ppt file.

  \section{Terminology}

%   TODO use fancy words to explain other fancy words

   The words \emph{slideshow} and \emph{presentation} are often used
   interchangeably throughout this report, although they do not quite cover the
   same meaning. By \emph{slideshow} we mean a presentation consisting of a set
   of slides, the kind \ppt* and many other presentation software provide us
   with. \emph{Presentation} then refers to the wider concept of material
   intended to be viewed and manipulated by people in order to convey
   information, usually but not necessarily from one or several presenter(s) to
   an audience.

   The term \emph{layout} refers to both the process of determining the
   position and size of each visual object that is to be displayed in a
   presentation, and the result of that process.

   \emph{Slideware} is a contraction of the words `slideshow' and `software',
   referring to software packages used to create slideshow presentations.

  \section{Stuff we want to solve}

%   TODO change this section's title

   According to several sources \citep{parker-1, drucker-1, bajaj-1}, over 30
   million \ppt presentations are being made every day. That is an enormous
   amount. Creating a \ppt presentation is easy; creating a good \ppt
   presentation, however, is not. Slides have a fixed size, and you can only
   fit so much information on one slide before the effectiveness of
   transferring that information to one's audience starts deteriorating. Over
   the years, many people have created written and unwritten guidelines to
   creating effective slideshows, specifying how much text and how many images
   should fit on one slide. Over those years, many people have failed to follow
   those guidelines. But whether you choose to follow the guidelines or
   not, one thing remains true: people who create slideshow presentations spend
   most of their time not on the \emph{content} of their presentation, but on the
   \emph{layout}.

   The layout of a presentation can have a significant impact on how well it
   communicates information to and obtains information from those who interact
   with it. The vast majority of layouts created today are done ``by hand'': a
   human graphic designer or ``layout expert'' makes most, if not all, of the
   decisions about the position and size of the objects to be presented.
   Designers typically spend years learning how to create effective layouts,
   and may take hours or days to create even a single screen of a presentation.
   Designing presentations by hand is too expensive and too slow to address
   situations in which time-critical information must be communicated.

   It may seem like an overstatement to emphasize the significance of layout
   and formatting in presentations. One could assume these issues are
   irrelevant, or that only inexperienced presenters would make these mistakes.
   The real-life example of the space shuttle Columbia illustrates that this is
   not always the case. Leading up to the tragic incident in which the shuttle
   burned up during re-entry after spending 2 weeks in orbit, Boeing
   Corporation engineers delivered three reports to {NASA} totalling 28 \ppt
   slides, to help them assess the damage caused by a piece of debris hitting
   the wing of the shuttle during launch, and the threat this damage might have
   posed.  As Edward Tufte beautifully describes in his article ``\ppt Does
   Rocket Science'' \citep{tufte-2}, the reports existed only in those slides,
   and the slides were woefully inadequate for the task at hand. Although Tufte
   likes to suggest this proves that \ppt is an inherently bad tool, what it
   really proves is that \ppt makes it easy to create bad presentations, and a
   tool that either discourages this manner of presenting information or makes
   it altogether impossible would be a great improvement.

   Since layout is such a hard skill to master, we propose to automate this
   task, letting the presenter focus on the content of the presentation and
   providing a proper layout fit for the content provided through software.

