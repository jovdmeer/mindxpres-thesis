% vim:ts=1:et:nospell:spelllang=en_gb:ft=tex

 \chapter{Introduction}

  For over 25 years, \ppt* has been the market leader in digital prsentations.
  Admittedly, it was a revolutionary software package when it was first
  introduced, and its ease-of-use combined with its supreme graphical
  capabilities --- at least compared to other software in the same era --
  quickly made it one of the most popular software packages in history. 25
  years later, \ppt* can claim over 90\% market share in presentation software,
  and on average 30 million \ppt presentations are created every day.

  In this time, \ppt* has gotten many new features, and certainly improved and
  grew with every new version, but it never really changed its core approach.
  It started out mimicking the then-popular and widespread use of dia and
  overhead projection slides, which was at the time a good way to convince
  people of its purpose, allowing them to feel comfortable with a familiar
  format instead of alienating potential customers with a new and potentially
  confusing interface.

  However, this interface is quite restricting, and in recent years different
  approaches have seen the light of day. The zoomable user interface of Prezi
  is probably the most well-known, but apart from abandoning the traditional
  slide format it does little to improve or extend the concept of presenting
  information to an audience.

  This is where \mxp comes in. Its extensible plug-in system allows anyone with
  some knowledge of programming to create new functionality to use in
  presentations. Examples are interactivity with the audiencer through various
  means, controlling the presentation from another device --- or several! ---
  and (re)modelling data while presenting it, based on feedback from the
  audience.

  While this is obviously a big improvement on the traditional presentation
  model of \ppt* and the likes, it remains hard to convince the general public
  of its merits. People are generally afraid of change, and it is important to
  make the transition as smooth as possible. On top of that, people are often
  worried that the work they did in the past may be lost --- or worse,
  irrelevant --- after switching to something new. This alone may be a huge
  factor in deciding wether or not to start using new software, or to stick
  with what they know.

  That is where the subject of this thesis comes in. We aim to provide a way
  for people to convert their existing \ppt presentations into \mxp
  presentations, allowing them to take their previous work with them in their
  switch to \mxp. This way, we lower the treshold for them to make the decision
  to start using \mxp as their presentation software of choice. Once all their
  existing \ppt content is available, usable and editable in \mxp, it should be
  obvious to anyone why \mxp is the better option for their presentations.

%  TODO argue that automatic layout helps liberate the content from the confines
%  of slides, not just fix bad layout

  Another common problem with \ppt presentations is the way they look. This is
  not necessarily the fault of the software; most people just are not trained
  in graphical design, and as such they know very little about proper layout,
  color choices, or slide content limits. Everyone has probably encountered
  slides with full paragraphs of text, too small to read and / or too much to
  process in the short time the slide is visible --- (too) many people have
  made those slides themselves.

  When we say this is not the fault of the software, that is mostly true, as
  the creators of these slides obviously made a conscious choice to make their
  content appear like that. It could be said however that \ppt* and other
  presentation tools are guilty through inaction. We believe it is possible to
  have software either warn its users against these choices and practices, or
  --- even better --- have the software fix these problems automatically.

  One of the primary purposes of \mxp is to provide automatic layout, much like
  \latex does, ensuring that the content creator only has to worry about the
  actual content, while the software takes care of layout. In practice, both
  \latex and \mxp currently use template-based layouts, where the contents'
  position is predefined in the template and not related to or based on its
  size, shape or nature. In the end, everyone who has ever used \latex knows
  that sooner or later you will struggle to get a certain image incorporated in
  the text correctly, ending up doing the layout yourself anyway, because the
  predefined template just doesn't work properly for your specific content.

  Our goal is to eradicate those situations. Automatic layout should
  dynamically adjust to any content it is given, no matter the size or aspect
  ratio. This may seem hard, if you consider the limits of slides and the fact
  that you can only fit so much content on them before they are full. this is
  where another important aspect of \mxp comes into play: we are not
  necessarily bound to the limits of slides. If we don't have to consider the
  boundaries of traditional slides, we can fit content together in an
  aesthetically pleasing way much easier, without having to scale anything.

  As such, the second part of this thesis focuses on implementing true
  automatic layout in \mxp. Again primarily to convince \ppt* users to switch,
  showing that their presentations actually could look better in \mxp, while
  thus also providing new functionality to existing \mxp users.

%  TODO moar?
